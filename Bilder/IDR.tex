% Define block styles
\tikzstyle{block} = [rectangle, draw, fill=blue!20, text width=5em, text centered, rounded corners, minimum height=4em, text width=8em]
\tikzstyle{cloud} = [draw, rectangle,fill=grey!20, node distance=2.5cm, minimum height=2em]


\begin{tikzpicture}[scale=0.6, every node/.style={scale=0.6}]
% Place nodes
    \node[cloud, scale=1.5] at (-4,0.3)      (dispo)         {Disposition};
    \node[cloud, scale=1.5] at (0.2, 0.3)      (verlader)      {Verlader} ;
    \node[cloud, scale=1.5] at (4.8,0.3)       (rangierende)   {Lokrangierführer};
    \node[block] at (-4, -1.5)  (Wagen)         {Wagen wird telefonisch angemeldet};
    \node[block] at (-4, -3.7)  (Waamf)         {Ausfüllen des Wagenan- und -abmeldungs- formulars};
    \node[block] at (-4, -6)    (RA)            {Erstellung eines Rangierauftrags im DVS};
    \node[block] at (4.8, -6)     (RA2)           {Erhält Rangierauftrag und erfüllt diesen};
    \node[block] at (0.2, -8.3)   (beladung)      {Wagen wird beladen, kontrolliert und evtl. verschoben};
    \node[block] at (0.2, -10.3)  (ueprot)        {Übertragungs- protokoll wird ausgefüllt};
    \node[block] at (-4, -12.3) (RA3)           {Erstellung eines Rangierauftrags im DVS};
    \node[block] at (4.8, -13.7)  (RA4)           {Erhält Rangierauftrag, holt Wagen ab, verweigt ihn und trägt das Gewicht ins DVS ein und bringt Wagen ins Ausfahrgleis};
    \node[block] at (-4, -15)   (Lieferdoc)     {Erstellt Lieferdokumente mit Nettogewicht, Transportmittelart, Route und Versantdbedingungen};
    \node[block] at (-4, -17.6) (Transportdoc)  {Erstellt Transportauftrag};
    \node[block] at (-4, -19.6) (abmelden)      {Meldet den Wagen ab. Damit ist die Ware im Warenausgang};
    \node[block] at (-4, -22.4) (Frachtbrief)   {Erstellt mittels Transportnummer, Liefernummer und Übertragungs- protokoll einen Frachtbrief};
    \node[block] at (4.8, -23.2)  (Zettel)        {Bezettel Wagen};
    \node[block] at (-4, -25.0) (RA5)           {Erstellung eines Rangierauftrags im DVS};
    \node[block] at (4.8, -25.3)  (RA6)           {Erhält Rangierauftrag und bringt Wagen in den Vorbahnhof};

% Draw edges
    %Kasten Dispo
    \draw[dashed] (-6, 1.2) -- (-6,-26.5);
    \draw[dashed] (-6, 1.2) -- (-2, 1.2);
    \draw[dashed] (-2, 1.2) -- (-2, -26.5);
    \draw[dashed] (-6, -26.5) -- (-2, -26.5);
    %Kasten Verlader
    \draw[dashed] (-1.75, 1.2) -- (2.05, 1.2);
    \draw[dashed] (-1.75, 1.2) -- (-1.75, -26.5);
    \draw[dashed] (-1.75, -26.5) -- (2.05, -26.5);
    \draw[dashed] (2.05, 1.2) -- (2.05, -26.5);
    %Kasten LRF
    \draw[dashed] (2.25, 1.2) -- (2.25, -26.5);
    \draw[dashed] (2.25, 1.2) -- (7.35, 1.2);
    \draw[dashed] (2.25, -26.5) -- (7.35, -26.5);
    \draw[dashed] (7.35, 1.2) -- (7.35, -26.5);
    
    %Pfeile ohne Ende
    %Pfeile mit Ende
\end{tikzpicture}
