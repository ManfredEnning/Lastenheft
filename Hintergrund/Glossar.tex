\newglossaryentry{Ablaufberg}{
    name=Ablaufberg,
    description={Der Ablaufberg ist ein in der Regel künstlich angelegter Hügel, über den ein Gleis verläuft. Ablaufberge dienen beim Abdrücken dem Ablaufen lassen von Güter- wagen, die auf diese Weise nach ihren Bestimmungsorten sortiert werden}
}
\newglossaryentry{Bremsprobe}{
    name=Bremsprobe,
    description={Eine Bremsprobe  ist ein zur Vorbereitung von Zugfahrten gehörender Vorgang, bei dem die Funktionsfähigkeit des Bremssystems der Fahrzeuge im Zugverband überprüft wird. Dabei wird im Stillstand das Anlegen und Lösen der zu prüfenden Bremsen kontrolliert}
}
\newglossaryentry{Eisenbahnverkehrsunternehmen}{
    name=Eisenbahnverkehrsunternehmen,
    description={Eisenbahnverkehrsunternehmen sind Eisenbahnen, die Eisenbahnverkehrsleistungen erbringen. \acrshort{EVU}s müssen in der Lage sein, die Zugförderung (Traktion) sicherzustellen\cite{rennsteigbahn}}
}
\newglossaryentry{Heisslaeuferortungsanlage}{
    name=Hei\ss l\" auferortungsanlage,
    description={Eine Heißläuferortungsanlage dient dazu, eine unzulässige Er- wärmung von Radsatzlagern durch Defekte bei Schienenfahrzeugen (sogenannte Heiß- läufer) rechtzeitig feststellen zu können}
}
\newglossaryentry{ep-Bremsen}{
    name=ep-Bremse,
    description={Die elektropneumatische Bremse (ep-Bremse) ist eine durch elektropneumatische Bauteile gesteuerte automatische Druckluftbremse bei Eisenbahnen. Die ep-Bremse ermöglicht das gleichzeitige Bremsen oder Lösen aller Fahrzeuge unabhängig von der Länge des Zuges}
}
\newglossaryentry{ep-assist-Bremsen}{
    name=ep-Assist,
    description={ep-Assist ist die Funktion des ep-Bremsen. Die Funktion beschränkt sich auf das unterstützte Bremsen der \gls{ep-Bremsen} und beinhaltet kein ep-Lösen}
}
\newglossaryentry{EOW}{
    name=EOW,
    description={Eine elektrisch ortsgestellte Weiche (EOW) ist eine elektrisch angetriebene Weiche, die nicht von einem Stellwerk, sondern vom Weichenort aus direkt bedient wird. EOW sind das moderne Äquivalent zu Handweichen und werden hauptsächlich in Gleisanlagen eingesetzt, in denen nur frei rangiert wird}
}
\newglossaryentry{Bergbremse}{
    name=Bergbremse,
    description={Bergbremsen sorgen bei großen Rangierbahnhöfen nach erfolgter Geschwindigkeitsmessung für eine Vorabbremsung, damit die Talbremsen ihre Funktion erfüllen können}
}
\newglossaryentry{Talbremse}{
    name=Talbremse,
    description={Talbremsen verzögern Wagen vor einer Richtungsgleisgruppe, so dass der Abstand zum vorherlaufenden Wagen groß genug bleibt, damit die Weichen in der Lücke umgestellt werden können}
}
\newglossaryentry{Gleisanschluss}{
    name=Gleisanschluss,
    description={Ein Gleisanschluss ist ein Schienenweg zur Erschließung eines Geländes oder Gebäudes, das selbst nicht zur öffentlichen Eisenbahninfrastruktur gehört}
}
\newglossaryentry{Hemmschuh}{
    name=Hemmschuh,
    description={Ein Hemmschuh ist eine keilförmige Konstruktion zum Festhalten von Schienenfahrzeugen. Er wird zwischen Rad und Schiene platziert, um durch die entstehende Reibung den Wagen zu bremsen beziehungswies an (selbstständiger) Bewegung zu hindern}
}
\newglossaryentry{Knotenbahnhof}{
    name=Knotenbahnhof,
    description={Ein Knotenbahnhof ist eine Bahnhof der für Betriebsabläufe zur Zugbereitstellung oder Verknüpfung mit anderen Verkehrsträgern erforderlich ist}
}
\newglossaryentry{Rangierbahnhof}{
    name=Rangierbahnhof,
    description={Rangierbahnhöfe sind die Zugbildungsbahnhöfe des Einzelwagenverkehrs im Güterverkehr der Eisenbahn}
}
\newglossaryentry{Rangierfahrt}{
    name=Rangierfahrt,
    description={Rangierfahrten bezeichnen das Bewegen einzelner Schienenfahrzeuge oder Fahrzeuggruppen, soweit es sich nicht um eine Zugfahrt (einschließlich Sperrfahrt) handelt}
}
\newglossaryentry{Satellitenbahnhof}{
    name=Satellitenbahnhof,
    description={Satellitenbahnhöfe bestehen in der Regel aus kleinen Gleisanlagen ohne Rangiereinrichtungen und -personal. Sie dienen der Sammlung und Verteilung Güterwagen auf die Lokalen Gleisanschlüsse\cite{Verkehrslogistik}}
}
\newglossaryentry{Schiebewandwagen}{
    name=Schiebewandwagen,
    description={Der gedeckte Güterwagen der Sonderbauart H, oder auch Schiebewandwagen, ist ein gebräuchlicher Wagen für nässeempfindliche, pallettierte Ware. Die verschieblichen Seitenwände ermöglichen es, die ganze Ladefläche von der Seite her zu be- und entladen}
}
\newglossaryentry{Sperrfahrt}{
    name=Sperrfahrt,
    description={Sperrfahrten sind Fahrten, die in ein Gleis der freien Strecke eingelassen werden, das gesperrt ist. Dies dient der Bedienung einer Anschlussstelle auf der freien Strecke\cite{RIL408}}
}
\newglossaryentry{Zugfahrt}{
    name=Zugfahrt,
    description={Eine Zugfahrt bezeichnet eine Fahrt im Bahnhof und auf der Strecke, die durch Hauptsignale gesichert und geregelt ist, sowie Züge im Bereich mit Führerstandsignali- sierung. Bei dieser Fahrt ist der Fahrweg bis zum Ende frei. Es wird Flankenschutz gewährt und Weichen gegen Umstellen gesichert. Die Strecke für die Zugfahrt hat eine maximale Geschwindigkeit vorgegeben, der Zug hat eine feste Zusammensetzung und eine eindeutige Zugnummer für diese Fahrt}
}
\newglossaryentry{Zugschluss}{
    name=Zugschluss,
    description={	Der Zugschluss bezeichnet den letzten Wagen eines Zuges. Dieser ist vom Zugschlusssignal gekennzeichent. Mit seiner Hilfe kann die Vollständigkeit von Zügen, visuell durch das Personal des Bahnbetriebs, überprüft werden}
}
\newglossaryentry{Vorbahnhof}{
    name=Vorbahnhof,
    description={Als Vorbahnhof bezeichnet man den äußeren Teil eines großen Bahnhofes oder auch einen Bahnhof in der Nähe eines großen Bahnhofs, der diesem Aufgaben abnimmt}
}
\newglossaryentry{Lokrangierfuehrer}{
    name=Lokrangierf\"uhrer,
    description={Der Lokrangierführer ist eine Bezeichnung für einen Triebfahrzeug- führer oder einen Eisenbahnfahrzeugführer von Lokomotiven mit Funkfernsteuerung im Rangierdienst}
}
\newglossaryentry{Eisenbahnbetriebsleiter}{
    name=Eisenbahnbetriebsleiter,
    description={Eisenbahnbetriebsleiter leiten und überwachen die sicherheitsrelevanten Abläufe in einem Eisenbahnunternehmen, das keinen grenzüberschreitenden Verkehr aufweist}
}
\newglossaryentry{Bremsprobeberechtigte}{
    name=Bremsprobeberechtigte,
    description={Der Bremsprobeberechtigte ist ein für die Bremsprobe ausgebildeter Mitarbeiter, der die Bremsprobe durchführen darf. Häufig haben Triebfahrtzeugführer diese Zusatzausbildung selbst, um die Bremsprobe selbstständig durchführen zu können}
}
\newglossaryentry{Bremsprobeanlage}{
    name=Bremsprobeanlage,
    description={Eine Bremsprobeanlage wird für die Herstellung der Betriebsbereitschaft eines Eisenbahnzuges benötigt. Sie dient dazu eine Hauptbremsprobe am Zugverband durchzuführen. Diese Hauptbremsprobe kann zwar auch ohne eine entsprechende (ortsfeste) Anlage durchgeführt werden, dann muss aber ein Triebfahrzeug und ein weiterer Bremsprobeberechntigter vorhanden sein. Aus Kostengründen werden deshalb in größeren Formationsbahnhöfen ortsfeste Bremsprobeanlagen eingesetzt, da damit Triebfahrzeuge und Personal eingespart werden können}
}
\newglossaryentry{Gueterwagen}{
    name=G\"uterwagen,
    description={Als konventionelle Güterwagen werden die Güterwagen bezeichnet, die aus konventionellen (herkömmlichen) Komponenten, wie Drehgestellen, Kupplungen, Bremsen und Wagenkästen, bestehen und der \acrshort{TSI} WAG entsprechen.}
}
\newglossaryentry{Gueterwagen 40}{
    name=G\"uterwagen 4.0,
    description={Als Güterwagen 4.0 wird der vollständig ausgebaute Güterwagen bezeichnet}
}
\newglossaryentry{Demonstrator}{
    name=Demonstrator,
    description={Als Demonstrator wird der in diesem Projekt modifizierte Güterwagen bezeichnet. Er stellt eine Teilmenge des Güterwagen 4.0 dar}
}
\newglossaryentry{40-Komponenten}{
    name=4.0-Komponenten,
    description={4.0-Komponenten sind die Subsysteme, die den konventionellen Güter- wagen zu einem Güterwagen 4.0 machen. Zu diesen Komponenten gehören die Energieversorgung, Aktoren und Sensoren sowie die Kommunikationssysteme des Güterwagen}
}
\newglossaryentry{Wagenzug}{
    name=Wagenzug,
    description={Ein Wagenzug ist ein gekuppelter Verband nicht angetriebenen Güterwagen}
}
\newglossaryentry{Zugverband}{
    name=Zugverband,
    description={Ein Verbund aus Güterwagen und Lok wird als Zugverband bezeichnet}
}
\newglossaryentry{Lastwechsel}{
    name=Lastwechsel,
    description={Die Ausstattung mit Lastwechsel ermöglicht das Anpassen des Bremsklotzdrucks an das effektive Wagengewicht. Sie verhindert ein Überbremsen bei geringer Zuladung und wirkt zu schwacher Bremskraft bei beladenen Fahrzeugen entgegen. Wenn bei gleichbleibender Bremskraft das Fahrzeuggewicht durch die Beladung zunimmt, verringert sich die Bremswirkung. Darum ist die Umstelleinrichtung in die passende Stellung („leer“, „beladen“ oder ggf. „teilbeladen“) manuell oder automatisch zu bringen}
}
\newglossaryentry{WagonOS}{
    name=WagonOS,
    description={WagonOS ist das offene Betriebssystem des Güterwagen 4.0. Das Betriebssystem ist auf Güterwagen 4.0 vorinstalliert und bietet den Nutzern eine Plattform zum anwendungsopimierten Güterwagen}
}
\newglossaryentry{Zweiwegefahrzeug}{
    name=Zweiwegefahrzeug,
    description={Ein Zweiwegefahrzeug ist ein Fahrzeug, das sowohl auf der Straße, als auch auf der Schiene fahren kann}
}
\newglossaryentry{RailPort}{
    name=RailPort,
    description={RailPorts sind multifunktionale Logistikzentren, die Gütern unabhängig von ihrer Beschaffenheit einen Umschlag von Schiene auf Straße, LAgerung oder LKW-Vor- und Nachlauf bieten. RailPorts verknüpfen europäische Schienennetze mit regionaler Straßeninfrastruktur}
}
\newglossaryentry{Bremsart}{
    name=Bremsart,
    description={Die Bremsart unterschiedet Bremsen von Schienenfahrzeugen nach Bremswirkung, Ansprechzeit und Energiequelle. Bei Güterwagen üblich sind die Bremsarten G und P. Beide bremsen bis zum Stillstand und sind pneumatisch. Die Bremsart G ist langsam wirkend, die Bremsart P schnellwirkend}
}
\newglossaryentry{technische Wagenbehandlung}{
    name=Technische Wagenbehandlung,
    description={Zur Gewährleistung eines sicheren Eisenbahnbetriebs werden Wagen, Ladung und intermodale Ladeeinheiten in Güterzügen einer technischen Behandlung unterzogen. Die technischen Wagenbetriebsarten stellen sicher, dass die im Einsatz befindlichen Wagen betriebssicher sind. Es gibt je nach Verkehr unterschiedliche Stufen der Ausführung\cite{RIL936}}
}
\newglossaryentry{Bremse 4.0}{
    name=Bremse 4.0,
    description={Die Bremse wird als eine der \gls{40-Komponenten} des \gls{Gueterwagen 40} mit Besonderheiten zur Verkürzung der Vorbereitungszeit der Güterwagen \cite{Stephenson, ETR_2}}
}
\newglossaryentry{Systembatterie}{
    name=Systembatterie,
    description={Die Systembatterie ist, anders als die Pufferbatterie, ein direkter Bestandteil des Bordnetztes und für die Funktion des Bordnetzes notwendig.}
}

%	Bedienfahrt	
%	Bezetteln   
%\newglossaryentry{Rangiermittel}{
 %   name=Rangiermittel,
  %  description={Rangiermittel sind Ein- oder Zweiwegefahrzeuge die (mit Elektroantrieb und oder im Handbetrieb) Die Aufgaben von Rangierlokomotiven übernehmen.}
%}
%\newglossaryentry{Sägefahrten}{
 %   name=Sägefahrten,
  %  description={Sägefahrten sind das koordinierte Vor- und Zurücksetzen von Wagen zum rangieren von Wagen.}
%}