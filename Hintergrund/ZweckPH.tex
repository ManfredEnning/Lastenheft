\section*{Zweck des Dokuments}
Das Pflichtenheft ist das, nach DIN 69901-5: Projektmanagement – Projektmanagementsysteme – Teil 5: Begriffe, ”vom Auftragnehmer erarbeitete Realisierungsvorhaben auf Basis des vom Auftraggeber vorgegeben Lastenheftes”\cite{DIN69901-5}.\par

Anhand des Pflichtenheftes findet die Bearbeitung des Projektes statt. Es bietet eine ausführ- liche
Antwort auf die Fragen ”wie” und ”womit”. Die Anforderungen des zuvor ausgearbeiteten Lastenhefts werden nun mit technischen Festlegungen verknüpft.\par

Das Pflichtenheft beinhaltet die vom Auftragnehmer erarbeiteten Anforderungen aus dem Lastenheft. Es werden im Allgemeinen konkrete Fälle ein- oder ausgeschlossen. Zusätzlich finden sich hier Dokumentationen von Analysen als Grundlage. Im Pflichtenheft werden die Zielbestimmungen exakt eingegrenzt. Diese werden in drei Kategorien aufgeteilt. 
\begin{itemize}
    \item Muss-Kriterien -- Diese sind unerlässlich, damit die Anwendung funktioniert
    \item Wunsch-Kriterien -- Diese bleiben entbehrlich. Auch ohne sie ist die Nutzung des Systems und dessen Inbetriebnahme möglich, sie sollen trotzdem umgesetzt werden.
    \item Abgrenzungs-Kriterien -- Diese machen klar, was das System auf keinen Fall beinhalten sollte.
\end{itemize}

Pflichtenhefte sind wichtig als Grundlage für den Projekterfolg. Sie enthalten eindeutige Projektspezifikationen, die als Leitlinie gelten. Alle Projektbeteiligten sollen sich bei der Erstellung des Pflichtenheftes darüber klar werden, was sie voneinander und vom Projekt erwarten und wie sie sich einen erfolgreichen Projektabschluss vorstellen.\par

In diesem Fall wird das Pflichtenheft im Rahmen des Projektes ''Neue Elektronik- und Kommunikationssysteme für den intelligenten, vernetzten Güterwagen'' von der FH Aachen erstellt.\par
Das Pflichtenheft ist sowohl als Teil des Arbeitspaketes 1 ''Grundlagenuntersuchungen'', als auch als Teil der Arbeitspakete 3 ''Entwicklung Aktorik'', 4 ''Entwicklung Sensorik'' und 5 ''Datenkommunikation'' zu sehen. Es bearbeitet die die Unterpunkte 1a bis 1f, 3a, 4a und 5a.
\begin{itemize}
    \item 1a: Grundlagenuntersuchung und Anforderungsanalyse Gesamtsystem, Sensoren, Aktoren, Kommunikation
    \item 1b: Grundlagenuntersuchung zur digitalen Vernetzung von Güterwagen untereinander und mit intelligenter Ladung (Digitales Typenschild, robuste IoT-Plattform, Güter- wagenbetriebssystem WagonOS, usw.)
    \item 1c: Grundlagenuntersuchung zur 'Digitalen Identität' und Zugbildung mit automatischer Bremseinstellung und Bremsprobe
    \item 1d: Grundlagenuntersuchung zu Anforderungen an sicheres funkgestütztes Bedienen und Beobachten bei der Zugbildung
    \item 1e: Konzeption und Spezifikationsentwicklung Power-Management, Grundlagenuntersuchung und Anforderungen an den Generator, die Batterie und das Lademanagement
    \item 1f: Grundlagenuntersuchung und Anforderungsanalyse für Data Analytics, System und Sicherheit
    \item 3a: Konzeptentwicklung Aktorik und Aktorsteuerung
    \item 4a: Konzeptentwicklung Sensorik
    \item 5a: Entwicklung eines Hardware- und Software-Konzeptes für die Kommunikation (z.B. Zugbus, Richtfunk, WLAN, Mobilfunk, etc.)
\end{itemize}