\section{Anforderungen}
\textbf{Einfügen: Wie lang sind Lösezeiten, Verwendung Handbremse, Umlegen Endabsperrhähne}\par
\textbf{Sicherheitsrelevanz, Bedienen, Beobachten, Anbringungspunkte von Bauteilen -- gefedert? -- Bahntauglichkeit}\par
\subsection{Allgemeine und nicht funktionale Anforderungen}
\begin{feat}[Anf. X]
Umgebungsbedingungen nach \acrshort{DIN} \acrshort{EN} 50155
\end{feat}
\begin{feat}[Anf. X]
mindestens gleichbleibende Sicherheit des Systems
\end{feat}
\begin{feat}[Anf. X]
Rüttel- und Schüttelfestigkeit nach \acrshort{DIN} \acrshort{EN} 61373
\end{feat}
\begin{feat}
EMV-Verträglichkeit nach ?
\end{feat}
\begin{feat}
Brandschutznachweise nach \acrshort{DIN} \acrshort{EN} 45545 (gegen TSI WAG prüfen)
\end{feat}
Verstellkomponenten möglichst nah beieinander um möglichst wenig Kabel zu ziehen\\
Sensoren immer überwachen! Mustererkennung - Rohdaten auswerten - was ist wichtig?
\subsection{Bremse}
\begin{feat}[REQ. 9]
elektrisches Lösen der Handbremse möglich
\end{feat}
\begin{feat}[REQ. 10]
elektrisches Lösen der Bremse möglich
\end{feat}
\begin{feat}[REQ. 11]
Automatische Einstellung der Bremsstellung
\end{feat}
\begin{feat}[REQ. 12]
Automatische oder ferngesteuerte Betätigung der Endabsperrhähne
\end{feat}
\begin{feat}[REQ. 12]
Bremsstellung G und P werden unterstützt
\end{feat}
\begin{feat}[REQ. 13]
automatische Bremsprobe für Bedienfahrt
\end{feat}
\begin{rem}
XXX
\end{rem}
\begin{feat}
automatische Bremsprobe für Rangierfahrt
\end{feat}
\begin{feat}
automatische Bremsprobe für Sperrfahrt
\end{feat}
\begin{feat}
automatische Bremsprobe für Zugfahrt
\end{feat}
\begin{feat}
automatische Bremsberechnung anhand des Wagenzuges und der Lok möglich
\end{feat}

\subsection{Kupplung}
\begin{feat}[REQ. 14]
elektrische Vorbereitung von Kuppelstellen möglich
\end{feat}
\begin{feat}
elektrische Vorbereitung von Trennstellen möglich
\end{feat}
\begin{rem}
XXX
\end{rem}

\subsection{Rechnerbasierte Anforderungen}
\begin{feat}
digitales Bilden einer Wagengruppe möglich
\end{feat}
\begin{feat}
Betriebsparameter für Bedienfahrt
\end{feat}
\begin{feat}
Betriebsparameter für Rangierfahrt
\end{feat}
\begin{feat}
Betriebsparameter für Sperrfahrt
\end{feat}
\begin{feat}
Betriebsparameter für Zugfahrt	
\end{feat}
\begin{feat}
automatische Wagengruppenbildung möglich
\end{feat}
\begin{feat}
Überprüfung der Wagenreihung möglich
\end{feat}
\begin{feat}
Überprüfung des technischen Zustands aus vorheriger Fahrt möglich
\end{feat}
\begin{feat}
automatische Übermittlung von Transportunterlagen möglich
\end{feat}

\subsection{Ideenspeicher}
\begin{feat}
automatische Zugschlussanzeige
\end{feat}
\begin{feat}
Rangierantrieb
\end{feat}