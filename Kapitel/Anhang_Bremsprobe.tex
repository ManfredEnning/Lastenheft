\section{RIL 915 - Bremsen im Betrieb bedienen und prüfen - Auszug}
Die folgenden Abschnitte sind Auszüge aus der RIL 915 - Bremsen im Betrieb bedienen und prüfen.\cite{RIL915}\par
Die Richtlinie (Ril) enthält die Bestimmungen für das Bedienen und Prüfen der Bremsen im Betrieb und für die damit zusammenhängenden Aufgaben.
\subsection{Bremsprobeberechtigte}
Für das Bedienen und Prüfen der Bremsen im Betrieb ist eine Befähigung zum Bremsprobeberechtigten erforderlich. Einzelheiten sind durch das Unternehmen zu regeln.
\subsection{Bremsausrüstungen der Fahrzeuge, Kurzbezeichnungen, Bremsanschriften}
\textbf{Grundsätzliche Ausrüstung}\par
Fahrzeuge besitzen in ihrer Grundausrüstung Reibungsbremsen, die als Klotz-, Scheiben- oder Trommelbremsen ausgebildet sind. Sie werden allgemein als Druckluftbremse bezeichnet.\par
\textbf{Grundsätzliche Steuerung}\par
Die Steuerung dieser Bremsen erfolgt über die Bremsleitung, durch:
\begin{itemize}
    \item Druckänderung in der Hauptluftleitung (HL) – selbsttätige Druckluftbremse –, ggf. unterstützt durch die elektropneumatische (ep) Bremssteuerung,
    \item direkte Ansteuerung des Bremszylinderdruckes - nichtselbsttätige Druckluftbremse -,
    \item direkte elektropneumatische (el) Ansteuerung des Bremszylinderdruckes
\end{itemize}
\textbf{Regelbetriebsdruck}\par
Der Regelbetriebsdruck der Hauptluftleitung beträgt 5,0 bar.\par
\textbf{Selbsttätigkeit}\par
Die Bremsen der Fahrzeuge sind selbsttätig, im Falle einer unbeabsichtigten Trennung der Bremsleitung wirkt die Bremse automatisch. Die Selbsttätigkeit wird entweder durch das Prinzip der indirekt wirkenden Druckluftbremse mit durchgehender Hauptluftleitung oder durch eine Schnellbremsschleife erreicht.\par
Abweichend hiervon können
\begin{itemize}
    \item Kleinlokomotiven eine nichtselbsttätige (direkt wirkende) Druckluftbremse mit einer Steuerung für angeschlossene selbsttätige Druckluftbremsen,
    \item Güterwagen nur eine durchgehende Hauptluftleitung,
    \item Nebenfahrzeuge eine nichtselbsttätige Bremse der Kraftfahrzeugbauart 
\end{itemize}
haben.\par
\textbf{Zusätzliche Bremsausrüstungen}\par
Als zusätzliche Bremsausrüstungen können vorhanden sein:
\begin{itemize}
    \item eine Feststellbremse
    \begin{itemize}
        \item als Handbremse (bedienbar vom Boden aus, auf der Bühne, im Wagen oder im Führerraum) oder
        \item als Federspeicherbremse (bedienbar im Führerraum oder unterhalb des Langträgers bzw. bei abgerüsteten Triebfahrzeugen durch Steuerung des Druckes in der Hauptluftleitung) oder
        \item als Fußbremse bei einigen Kleinlokomotiven,
    \end{itemize}
    \item eine Zusatzbremse an Triebfahrzeugen, Wagen, bzw. Nebenfahrzeugen mit Bremsen der Regelfahrzeugbauart, die am betreffenden Fahrzeug als nichtselbsttätige Bremse wirkt,
    \item eine dynamische Bremse, und zwar an elektrischen Triebfahrzeugen und Brennkrafttriebfahrzeugen mit elektrischem Antrieb eine generatorische Bremse (E-Bremse) sowie an anderen Brennkrafttriebfahrzeugen und Nebenfahrzeugen eine hydrodynamische Bremse (H-Bremse),
    \item eine Magnetschienenbremse (Mg) oder
    \item eine Wirbelstrombremse (WB).
\end{itemize}
Jedes Fahrzeug mit eigenständiger Bremsausrüstung hat mindestens eine Einrichtung zum Ein- und Ausschalten der Reibungsbremse und der eventuell zusätzlich vorhandenen Bremseinrichtungen. Bei einigen Bremsbauarten wird mit dem Ausschalten auch die Bremse gelöst. Abbildungen der hierzu üblichen Einrichtungen sind im Anhang 915.0107A03 dargestellt.\par
\textbf{Umstelleinrichtungen}
Fahrzeuge mit eigenständiger Bremseinrichtung können folgende Umstelleinrichtungen haben:
\begin{itemize}
    \item Mit dem Bremsstellungswechsel, ausgebildet als mechanischer Umstellhahn oder Schalter, können je nach Bauart der Bremse folgende Bremsstellungen mit unterschiedlicher Bremswirkung gewählt werden:
    \begin{itemize}
        \item G,
        \item P (bei Lokomotiven auch P2),
        \item R,
        \item P + Mg,
        \item R + Mg (ggf. am Bremsstellungswechsel nur mit MG bezeichnet) oder
        \item R + WB
    \end{itemize}
    \item  Automatische Lastabbremsung oder eine pneumatische oder von Hand einzustellende Lastwechselumstelleinrichtung zur Anpassung der Bremswirkung an die wechselnde Last. Abbildungen der hierzu üblichen Einrichtungen sind im Anhang 915.0107A04 dargestellt.
    \item Löseartwechsel (ein-/mehrlösig) und Geländewechsel an einigen Fahrzeugen fremder Bahnen, siehe auch Modul 915.0101 Abschnitt 1 Absatz (2).
\end{itemize}
\textbf{Löseeinrichtugnen}\par
Im Allgemeinen haben indirekt wirkende Druckluftbremsen zum Lösen der Bremse von Hand eine Löseeinrichtung (z.B. Lösezug/Lösetaster). Solange die Löseeinrichtung betätigt wird, wird die Bremse des Fahrzeuges gelöst. Schnelllöseventile ermöglichen bei entlüfteter Hauptluftleitung durch nur kurzzeitiges Bedienen der Löseeinrichtung ein vollständiges Lösen der Druckluftbremse. Löseeinrichtungen mit Schnelllöseventil sind durch den am Lösezug angebrachten Steg mit der Aufschrift "autom." gekennzeichnet. Nach dem Ausschalten der Druckluftbremse ist die Löseeinrichtung solange zu betätigen, bis die Druckluftbremse vollständig entlüftet ist.\par 
Wurden Löseeinrichtungen betätigt, ist im Rahmen der Bremsprobe das Anlegen und Lösen der Bremsen festzustellen.\par
\textbf{Bremsanzeigeeinrichtungen}\par
Reibungs- und Feststellbremsen, deren Brems- und Lösezustände von außen nicht erkennbar sind, können Bremsanzeigeeinrichtungen besitzen (z.B. an den Fahrzeuglängsseiten, im Führerraum).\par
Diese können den Zustand „angelegt/gelöst/Anzeige ungültig oder gestört“ anzeigen. \par
Wagen mit Klotzbremse und Bremsstellung R haben an jeder Wagenlängsseite eine Bremskontrollanzeige zur Prüfung der niedrigen/hohen Abbremsung im Stand.\par
An Fahrzeugen mit Magnetschienenbremse befindet sich mindestens eine Bremskontrollanzeige mit Prüfknopf und Leuchtmelder zur Funktionsprüfung dieser Bremseinrichtung. An einigen Triebzügen sind diese Anzeigeeinrichtungen nicht vorhanden, wenn der Zustand durch die Führerraumanzeigen dargestellt wird.\par 
An Fahrzeugen mit Wirbelstrombremse befinden sich Kontrollanzeigen. Diese können den Wirk- bzw. Ausschaltzustand anzeigen. \par
Fahrzeuge mit ep-Bremse können einen Prüfknopf und Leuchtmelder zur Funktionsprüfung besitzen. \par
Bei Fahrzeugen mit zentralen Bremsprobeanzeigeeinrichtungen bzw. Führerraumanzeigen kann der Zustand aller angeschlossenen Bremsen vom Führerraum aus überwacht werden.\par
Die Bremsanzeigeeinrichtungen sind im Anhang 915.0107A03 auszugsweise abgebildet.\par
\textbf{Bremsgewicht}\par
Das Leistungsvermögen der Bremse der Fahrzeuge wird durch das Bremsgewicht (in Tonnen) ausgedrückt. Beispiele hierzu siehe Anhang 915.0107A04.
\begin{itemize}
    \item Bei Triebfahrzeugen, bei denen das Bremsgewicht der dynamischen Bremse angerechnet werden darf, ist das Bremsgewicht der dynamischen Bremse zusammen mit den Bremsgewichten für die Bremsstellungen der Reibungsbremse angeschrieben (z. B. R + E 160, R + E, P + E, R + H). Diese Bremsgewichte sind rot angeschrieben.
    \item Bei Triebfahrzeugen, die im abgerüsteten Zustand eine niedrigere Bremskraft erzeugen und dies somit zu einem entsprechend niedrigeren Bremsgewicht führt, werden diese Bremsgewichte zusätzlich in Klammern angeschrieben. 
    \item Fahrzeuge mit Schnellbremsbeschleuniger haben für die Bremsstellung R zwei Bremsgewichtsanschriften. Dabei ist das Bremsgewicht mit wirkendem Schnellbremsbeschleuniger rot angeschrieben. 
    \item Bei Güterwagen mit automatischer Lastabbremsung ist das Bremsgewicht als Höchstwert oder in Tabellenform angeschrieben.
\end{itemize}
\textbf{Bremsgewicht Feststellbremse}\par
Das Leistungsvermögen der Feststellbremse wird durch das Bremsgewicht in t bzw. als Festhaltebremskraft in kN angegeben. Beispiele hierzu siehe Anhang 915.0107A04.\par
\textbf{Bremsanschriften}\par
An den Fahrzeugen sind als Bremsanschriften die Kurzbezeichnungen gemäß Anhang 915.0107A04 für Bremsbauarten, Bremsstellungen, zusätzliche Bremsausrüstungen und Ergänzungen angeschrieben.
\subsection{Arten von Bremsproben}
Es gibt folgende Arten von Bremsproben:
\begin{itemize}
    \item Volle Bremsprobe (mit oder ohne Zustandsgang)
    \item Vereinfachte Bremsprobe\newline
    Besondere Formen der vereinfachten Bremsprobe je nach Fahrzeugbauart
    \begin{itemize}
        \item Führerraumbremsprobe (ggf. mit Führerraumanzeige)
        \item Vereinfachte Bremsprobe mit zentraler Bremsanzeigeeinrichtung
    \end{itemize}
\end{itemize}
\textbf{Zustandsgang und Ausführungsformen}\par
Die Ausführung der Bremsproben erfolgt manuell, benutzergeführt oder automatisch.
\begin{itemize}
    \item Bei der manuellen Bremsprobe werden die erforderlichen Arbeitsschritte von Hand eingeleitet und augenscheinlich beim Zustandsgang kontrolliert. 
    \item Bei der benutzergeführten Bremsprobe werden die in der Führerraumanzeige aufgeführten Arbeitsschritte von Hand eingeleitet und deren Ergebnisse zur augenscheinlichen Kontrolle angezeigt. In bestimmten Fahrzeugen wird die Kontrolle auch automatisch durchgeführt.
    \item Bei der automatischen Bremsprobe werden die Arbeitsschritte und die Kontrolle der Ergebnisse automatisch durchgeführt.
\end{itemize}
\subsection{Zweck und Umfang der Bremsproben} 
Bremsproben sind in der Regel in der Bremsstellung auszuführen, die für die nachfolgende Zugfahrt eingestellt ist. Muss nach der Bremsprobe die Bremsstellung geändert werden, ist – außer beim Umstellen in die Bremsstellung R + Mg – keine erneute Bremsprobe erforderlich.\par
\textbf{Volle Bremsprobe - 915.0103 -} \par
Bei der vollen Bremsprobe sind der Zustand und die Funktion der Bremsen aller Fahrzeuge festzustellen.\par
\textbf{Vereinfachte Bremsprobe - 915.0104 -}\par
Bei der vereinfachten Bremsprobe ist festzustellen, ob die Durchgängigkeit der Steuer- und Versorgungsleitungen (z. B. Hauptluftleitung, Hauptluftbehälterleitung bzw. elektrische Bremssteuerleitung) bis zum letzten Fahrzeug des Zuges
gegeben ist und die Bremsen vom führenden Fahrzeug aus gelöst werden können. Werden Fahrzeuge neu an die Hauptluftleitung angeschlossen, so ist ggf. der Zustand, das Anlegen und Lösen der Bremsen dieser Fahrzeuge und in der Regel das Anlegen und Lösen der Bremsen an den angrenzenden Fahrzeugen (vor und hinter der Kuppelstelle) festzustellen.\par
\textbf{Führerraumbremsprobe - 915.0104A31}\par
Bei der Führerraumbremsprobe ist die Funktion des Führerbremsventils/der Fahrbremsschalter im führenden Fahrzeug zu prüfen. Der abgesperrte Zustand der nicht benutzten Führerbremsventile und ggf. anderer Bremssysteme ist festzustellen.\par
\textbf{Vereinfachte Bremsprobe mit zentraler Bremseinrichtung - 915.104A41 -}\par
Bei der vereinfachten Bremsprobe mit zentraler Bremsanzeigeeinrichtung ist die Funktion des Führerbremsventils/des Fahrbremsschalters im führenden Fahrzeug zu prüfen. Der abgesperrte Zustand der nicht benutzten Führerbremsventile/Fahrbremsschalter ist festzustellen.\par
\textbf{Funktionsprüfung}\par
Bei der Funktionsprüfung prüft der bedienende Bremsprobeberechtigte die Funktion des bei der anschließenden Fahrt zu bedienenden Führerbremsventils/Fahrbremsschalters unter Beobachtung der Anzeigeeinrichtungen.
