\section{Reale Ist-Prozesse}\label{sec:realeIst}
In diesem Kapitel sollen weitere Wagenarten und auch unterschiedliche Ladungen angeschnitten werden.\par
Im ersten Teil im Anhang soll es um Kesselwagen auf dem Betriebsgelände der IDR (Industrieterrains Düsseldorf-Reisholz AG) gehen. Diese werden sowohl mit Gefahrgut als auch mit ''normalen'' Chemikalien gefüllt. Die hier abgebildeten Prozesse entsprechen nur der Abwicklung auf dem Betriebsgelände und sind aus einer Interview mit dem EBL extrahiert worden. Auf dem Betriebsgelände werden die Wagen, auch wenn sie später als Ganzzug das Werk verlassen wie Einzelwagen oder kleine Wagengruppen behandelt. Die hier beschriebenen Prozesse zeigen vor allem die Disposition. Die Kupplungsvorgänge, technischen Wagenbehandlungen und Bremsproben entsprechen grob denen bereits beschriebenen.
\subsection{Kessenwagen auf dem Betriebsgelände}
\textbf{Hier Prozess beschreiben}
%Der Wagen wird telefonisch bei der Disposition angemeldet, die ein Formular zur Wagenan- und Wagenabmeldung für die Abfüllung ausfüllt. Danach wird ein Rangierauftrag zum Abfüllgleis erstellt.