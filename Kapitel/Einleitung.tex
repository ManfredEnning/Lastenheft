\section{Einleitung}
In diesem Lastenheft wird anhand eines Schiebewandwagens der grobe Ablauf eines Umlaufs im bisherigen System erläutert und daraufhin Anforderungen an den neuen Güterwagen erstellt. Als Motivation zeigen sich %neben der oben bereits erwähnten Verringerung von Verkehrsstaus und Schadstoffemissionen auch 
eine Erhöhung der Prozesssicherheit, Verringerung der aufzuwendenden Zeit am Wagen, eine Gestaltung von attraktiveren Arbeitsplätzen durch eine ergonomischere Arbeitsplatzgestaltung und mögliche Kosteneinsparungen an der Infrastruktur.\par
Der Güterverkehr in Deutschland wird bislang zu über 70\% von LKWs gestemmt, was Verkehrsstaus und hohe Schadstoffemissionen verursacht. Das Zukunftsprojekt Industrie 4.0 bietet dem Schienengüterverkehr die einzigartige Chance, durch intelligente Steuerung und Vernetzung Transportprozesse extrem flexibel, effizient und schadstoffarm zu gestalten. Realisiert werden kann dies durch die Integration vielfältiger moderner Sensorik und Elektronik in den Güterwagen 4.0.\cite{AZAP}\par
Herausforderungen wie die Migration und Annahme des Systems sollen beachtet werden. Die Kosten des Systems, sowie dessen Einbau müssen konkretisiert werden. Prozesse für produktivere Arbeitsabläufe mit dem neuen Güterwagen müssen gestaltet und Arbeitsabläufe bei Defekten konkretisiert werden.\par
Es geht bei dieser Art der Automatisierung nicht darum Arbeitsplätze abzuschaffen, sondern vor allem darum sie attraktiver zu gestalten, um das Problem des Personalmangels in den Griff zu bekommen und neues junges Personal zu finden. Gleichzeitig darf natürlich nicht die Sicherheit des Systems leiden. Das Sicherheitslevel bzw. die Sicherheitsanforderungsstufe muss mindestens genauso hoch bleiben, wie sie beim bisherigen System ist. Durch diese Erweiterung des konventionellen Güterwagens soll der Güterwagen ein besseres Arbeitsmittel auf dem Werksgelände und in Zugbildungsanlagen werden.\par



\begin{comment}
\textit{Anders oder raus! AM Ende nochmal drüber lesen!}
\textbf{Es ist geplant eine Möglichkeit für produktivere Verkehre zu schaffen.}\par
\textbf{Der im nächsten Kapitel beschriebene Ist-Zustand wird zeigen, dass viele manuelle Tätigkeiten für die Beladung und Abfertigung, sowie die Rangiervorgänge von Güterwagen im Einzelwagenverkehr notwendig sind. Dies verursacht hohe (Personal-)Kosten durch die benötigte Zeit und das benötigte Personal, sowie auch Kosten an der Verladestelle, wenn diese aufgrund von Kupplungs- und Rangiervorgängen still steht.}\par
\textbf{Diese Punkte sollen mit diesem Projekt angegangen werden. Dafür sollen verschiedene technische Stufen zu Ausstattungspaketen, die aufeinander aufbauen, zusammengestellt werden. Jedes Ausstattungspaket soll entweder bei gleicher Sicherheit manuelle Tätigkeiten abbauen oder vereinfachen, oder die Sicherheit erhöhen.}\par
%Hier fehlt eine Erklärung wo die Reise hin gehen soll, ein Fokus, was wir uns vom GW40 erhoffen und eine sensibilisierung für den Prozess und das Vorgehen zur Änderung dessen.\\
%Automatisierungslösungenen  Automatisches Abdrücken mit AK o. automatischer Schraubenkupplung – Trennstellen, Kommunikation mit Bergrechner – Vorbereitung Ablauf\\
\end{comment}



