\section{Einleitung}
In diesem Lastenheft wird anhand eines Schiebewandwagens der grobe Ablauf eines Umlaufs im bisherigen System erläutert und daraufhin Anforderungen an den neuen Güterwagen erstellt. Als Motivation zeigen sich %neben der oben bereits erwähnten Verringerung von Verkehrsstaus und Schadstoffemissionen auch 
eine Erhöhung der Prozesssicherheit, Verringerung der aufzuwendenden Zeit am Wagen, eine Gestaltung von attraktiveren Arbeitsplätzen durch eine ergonomischere Arbeitsgestaltung und mögliche Kosteneinsparungen an der Infrastruktur.\par
Der Güterverkehr in Deutschland wird bislang zu über 70\% von LKWs gestemmt, was Verkehrsstaus und hohe Schadstoffemissionen verursacht. Das Zukunftsprojekt Industrie 4.0 bietet dem Schienengüterverkehr die einzigartige Chance, durch intelligente Steuerung und Vernetzung Transportprozesse extrem flexibel, effizient und schadstoffarm zu gestalten. Realisiert werden kann dies durch die Integration vielfältiger moderner Sensorik und Elektronik in den Güterwagen4.0.\cite{AZAP}\par
%Im Projekt werden Sensoren und Elektroniksysteme zur Realisierung eines „intelligenten“, mittels Industrie 4.0-Technologien vernetzten Güterwagens entwickelt. Die Sensorik dient dabei der Online-Erfassung relevanter Güterwagen- und Zugverbund-Daten zur Zustands- und Verschleißanalyse. Durch diese Informationen wird erstmals eine durchgängige Logistik und die von Kunden geforderte Transparenz der Lieferkette sowie eine vorausschauende Planung von Wartungszyklen und Rentabilität für den Güterwagenbetreiber ermöglicht. Als weiterer Lösungsansatz werden Aktoren entwickelt und in den Güterwagen4.0 integriert, mit denen die beim Zusammenstellen bzw. Trennen von Güterwagen erforderlichen aufwendigen und oft sicherheitsrelevanten manuellen Tätigkeiten künftig vollautomatisiert durchgeführt und sensorisch online überwacht werden können.\footnote{Antrag auf Gewährung einer Bundeszuwendung auf Ausgabenbasis (AZAP), 26.04.2018}\\
%Der Güterwagen4.0 stellt einen wichtigen Baustein zur Lösung der gravierenden Verkehrsprobleme dar. Die Ergebnisse des Vorhabens tragen wesentlich dazu bei, durch einen zukunftsfähigen Schienenverkehr Verkehrsstaus und Schadstoffemissionen zu vermindern. Die angestrebten sensorischen Fähigkeiten des Güterwagen4.0 erlauben perspektivisch auch eine Übertragung in den Schienenpersonenverkehr.\footnote{Antrag auf Gewährung einer Bundeszuwendung auf Ausgabenbasis (AZAP), 26.04.2018}\\\\
Herausforderungen wie die Migration und Annahme des Systems sollen beachtet werden. Die Kosten des Systems, sowie dessen Einbau müssen konkretisiert werden. Prozesse für produktivere Arbeitsabläufe mit dem neuen Güterwagen müssen gestaltet werden. Arbeitsabläufe bei Defekten konkretisiert werden.\par
Es geht bei dieser Art der Automatisierung nicht darum Arbeitsplätze abzuschaffen, sondern vorallem darum sie attraktiver zu gestalten um das Problem des Personalmangels in den Griff zu bekommen und neues junges Personal zu finden. Gleichzeitig darf natürlich nicht die Sicherheit des Systems leiden. Die Sicherheit muss mindestens genauso bleiben, wie sie beim bisherigen System ist.\par
Es ist geplant eine Möglichkeit für produktivere Verkehre zu schaffen.\par
Der Ist-Zustand im nächsten Kapitel wird zeigen, dass viele manuelle Tätigkeiten für die Beladung und Abfertigung sowie die Rangiervorgänge von Güterwagen im Einzelwagenverkehr notwendig sind. Dies verursacht hohe (Personal-)Kosten durch die benötigte Zeit und das benötigte Personal, sowie auch Kosten an der Verladestelle, wenn diese aufgrund von Kupplungs- und Rangiervorgängen still steht.\par
Diese Punkte sollen mit diesem Projekt angegangen werden. Dafür sollen verschiedene technische Stufen zu Ausstattungspaketen, die aufeinander aufbauen, zusammengestellt werden. Jedes Ausstattungspaket soll entweder bei gleicher Sicherheit manuelle Tätigkeiten abbauen oder vereinfachen, oder die Sicherheit erhöhen.\par


%Hier fehlt eine Erklärung wo die Reise hin gehen soll, ein Fokus, was wir uns vom GW40 erhoffen und eine sensibilisierung für den Prozess und das Vorgehen zur Änderung dessen.\\
%Automatisierungslösungenen  Automatisches Abdrücken mit AK o. automatischer Schraubenkupplung – Trennstellen, Kommunikation mit Bergrechner – Vorbereitung Ablauf\\
• Vorteile des Systems\\
Dezentralität\\
Zukunftssicher\\
muss nicht auf AK warten, funktioniert aber auch damit\\
Wertschöpfungsprozess\\
Automatisierung\\
Automatisierung Bremse – Bremse lösen, lüften, Handbremse, Berechnung Bremsgewichte, Durchgängigkeit HL, automatische Bremsprobe, automatischer Bremszettel\\
Vorbereitung Trennstellen – HL-Absperrhähne Schließen, lüften, 2x Signal = trennen – Signal: kann, soll trennen\\
