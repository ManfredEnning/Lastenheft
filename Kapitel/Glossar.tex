\newglossaryentry{Ablaufberg}{
    name=Ablaufberg,
    description={Der Ablaufberg ist ein in der Regel künstlich angelegter Hügel, über den ein Gleis verläuft}
}
%	Bedienfahrt	
%	Bezetteln   
\newglossaryentry{Bremsprobe}{
    name=Bremsprobe,
    description={Eine Bremsprobe  ist ein zur Vorbereitung von Zugfahrten gehörender Vorgang, bei dem die Funktionsfähigkeit des Bremssystems der Fahrzeuge im Zugverband überprüft wird. Dabei wird im Stillstand das Anlegen und Lösen der zu prüfenden Bremsen kontrolliert. Siehe dazu auch Anhang \ref{sec:ABremsen}}
}
\newglossaryentry{EOW}{
    name=EOW,
    description={Eine elektrisch ortsgestellte Weiche ist eine elektrisch angetriebene Weiche, die nicht von einem Stellwerk, sondern vom Weichenort aus direkt bedient wird. EOW sind das moderne Äquivalent zu Handweichen und werden hauptsächlich in Gleisanlagen eingesetzt, in denen nur frei rangiert wird}
}
\newglossaryentry{Gleisanschluss}{
    name=Gleisanschluss,
    description={Ein Gleisanschluss ist ein Schienenweg zur Erschließung eines Geländes oder Gebäudes, das selbst nicht zur Eisenbahninfrastruktur gehört}
}
\newglossaryentry{Hemmschuh}{
    name=Hemmschuh,
    description={Ein Hemmschuh ist eine keilförmige Konstruktion zum Festhalten von Schienenfahrzeugen. Er wird zwischen Rad und Schiene platziert, um durch die entstehende Reibung den Wagen zu bremsen}
}
\newglossaryentry{Knotenbahnhof}{
    name=Knotenbahnhof,
    description={Ein Knotenbahnhof ist eine Bahnhof für Kreuzungen und Verknüpfungen im Verkehrssystem Eisenbahn, die für die Infrastruktur als Streckenknoten und Streckenkreuzungen oder für Betriebsabläufe zur Zugbereitstellung oder Verknüpfung mit anderen Verkehrsträgern erforderlich sind}
}
\newglossaryentry{Rangierbahnhof}{
    name=Rangierbahnhof,
    description={Rangierbahnhöfe sind die Zugbildungsbahnhöfe des Einzelwagenverkehrs im Güterverkehr der Eisenbahn}
}
\newglossaryentry{Rangierfahrt}{
    name=Rangierfahrt,
    description={Rangierfahrten bezeichnen das Bewegen einzelner Schienenfahrzeuge oder Fahrzeuggruppen, soweit es sich nicht um eine Zugfahrt (einschließlich Sperrfahrt) handelt}
}
%\newglossaryentry{Rangiermittel}{
 %   name=Rangiermittel,
  %  description={Rangiermittel sind Ein- oder Zweiwegefahrzeuge die (mit Elektroantrieb und oder im Handbetrieb) Die Aufgaben von Rangierlokomotiven übernehmen.}
%}
\newglossaryentry{Satellitenbahnhof}{
    name=Satellitenbahnhof,
    description={XXX}
}
%\newglossaryentry{Sägefahrten}{
 %   name=Sägefahrten,
  %  description={Sägefahrten sind das koordinierte Vor- und Zurücksetzen von Wagen zum rangieren von Wagen.}
%}
\newglossaryentry{Schiebewandwagen}{
    name=Schiebewandwagen,
    description={Der gedeckte Güterwagen der Sonderbauart H, oder auch Schiebewandwagen, ist eingebräuchlicher Wagen für nässeempfindliche, pallettierte Ware. Die verschieblichen Seitenwände ermöglichen es, die ganze Ladefläche von der Seite her zu be- und entladen}
}
\newglossaryentry{Sperrfahrt}{
    name=Sperrfahrt,
    description={Sperrfahrten sind Fahrten, die in ein Gleis der freien Strecke eingelassen werden, das gesperrt ist. Dies dient der Bedienung einer Anschlussstelle auf der freien Strecke\cite{RIL408}}
}
\newglossaryentry{Zugfahrt}{
    name=Zugfahrt,
    description={Eine Zugfahrt bezeichnet eine Fahrt im Bahnhof und auf der Strecke, die durch Hauptsignale gesichert und geregelt ist, sowie Züge im Bereich mit Führerstand -signalisierung}
}
\newglossaryentry{Zugschluss}{
    name=Zugschluss,
    description={	Der Zugschluss bezeichnet den letzten Wagen eines Zuges. Dieser ist vom Zugschlusssignal gekennzeichent. Mit seiner Hilfe kann die Vollständigkeit von Zügen visuell durch das Personal des Bahnbetriebs überprüft werden}
}
\newglossaryentry{SSS}{
    name=SSS,
    description={XXX}
}