\section{Soll-Zustand}
Der Ist-Zustand im vorigen Kapitel hat gezeigt, dass viele manuelle Tätigkeiten für die Beladung und Abfertigung sowie die Rangiervorgänge von Güterwagen im Einzelwagenverkehr notwendig sind.\par
Dies verursacht hohe (Personal-)Kosten durch die benötigte Zeit und das benötigte Personal, sowie auch Kosten an der Verladestelle, wenn diese aufgrund von Kupplungs- und Rangiervorgängen still steht.\par
Bei einem Umbau für die Demonstratoren muss darauf geachtet werden, dass ein vollständiger Rückbau der neuen einrichtungen möglich ist, aber alls so bahntauglich ist, dass die Demonstratoren für ein Folgeprojekt oder Feldversuche genutzt werden können. Diese Projekt muss nicht sofort zulassungsfähig sein, sollte aber eine Basis dazu bilden. \par
An der Bewegung der Wagen im Wagenverband an sich mit Hilfe einer Lok oder eines anderen Rangierhilfsmittel kann nichts groß automatisiert werden. Siehe dazu auch Kapitel \ref{sec:Zustellfahrt}, \ref{sec:Zugfahrt} und \ref{sec:Rangierfahrt}. Automatisierungen können aber selbst verständlich im Bereich der Kupplung, Bremse oder auch der informationstechnischen Prozesse stattfinden.\par
Zur besseren Einordnung sollen nun einige Komponenten definiert und erläutert werden aus denen sich verschiedene Stufen ergeben und aus denen sich wiederum später verschiedene Anforderungen ergeben.

\subsection{Bremse}
\subsubsection{Parkbremse}
mit Lust = ungebremst, ohne Luft = gebremst
\subsubsection{Bremshundertstel/Bremsberechnung}
Die Bremshundertstel werden bisher, genau wie das Bremsgewicht, händisch mittels Bremszettel berechnet. \textbf{BILD}. Auf diesem Vordruck trägt der Triebfahrzeugführer Achszahl, Zugmasse und Bremsgewichte des Zuges ein. Daraus berechnen sich die Bremshundertstel. Nach Vergleich dieser mit den Mindest-Bremshundertstel der Strecke, ergibt sich Höchstgeschwindigkeit und Bremsstellung. die Bremsstellung wird nach jeder neuen Zusammenstellung und vor Fahrtantritt, siehe Kapitel \ref{sec:UEdWagen}, neu berechnet.\par
Diese manuelle Berechnung ist für Triebfahrzeugführer tägliche Arbeit, dennnoch ist sie Fehleranfällig und kann durch Rechnereinsatz einfach automatisiert werden.

\subsubsection{automatische Bremse}
\textbf{TSI-Konformität des Steuerventils}\\ Bremsarten G und P mittels Umsteller und automatische Lastabbremsung (sichere Rückmeldung!)\\ 1.	Das Relaisventil ist für automatische Lastabbremsung mit Wiegeventil vorgesehen. \par
\textbf{fernbetättigte Absperrhähne}\\1.	Die Endabsperrhähne verfügen über einen freien Querschnitt von 1,25".
2.	Die Endabsperrhähne sind bistabil, d.h. verbleiben ohne Betätigung in ihrem letzten Zustand.
3.	Die Betätigungszeit für den Übergang Öffnen-Schließen beträgt maximal 60 s.
4.	Das Funktionsprinzip der Hähne ist geeignet, die Anforderungen der DIN EN 14601 erfüllen zu können. Für die Demonstratoren und Labormuster muss diese Norm nicht erfüllt werden.
5.	Eine fahrzeugseitige Verschraubung nach G1 1$/$4i (DIN EN ISO 228-1) ist zu bevorzugen. Für den Demonstrator kann die Kompatibilität durch einen Adapter hergestellt werden.
6.	Kupplungsseitig ist eine Verschraubung mit Whitworth-Gewinde mit stumpfen Gewinden für G1 1⁄4i—Leitungen zu bevorzugen. Für den Demonstrator kann die Kompatibilität durch einen Adapter hergestellt werden.
 \par
\textbf{Zustandsanzeige Bremse}\\ BILD, kann man kaufen. Norgren nachzertifizierbar?\\ 1.	Mindestens eine Anzeige je Fahrzeugende zum Einbau am Pufferträger ist vorzusehen.2.	Die Anzeige stellt den Zustand der Bremskupplung (drucklos/druckbeaufschlagt) dar.
3.	Die Anzeige muss Überdrücke > 0,5 bar in den Bremskupplungen anzeigen, bspw. durch die Farbe "Rot" im Schauglas.
4.	Bei Unterschreiten des Drucks wird bspw. die Farbe "Grün" angezeigt.
5.	Die Anzeige muss im stromlosen Zustand verfügbar sein.\par
\textbf{Fernbetätigung BRemse}\\ Schnelllösen, Bremse aus, Bremsstellung, Feststellbremse\\ 1.	Es müssen folgende Funktionen fernbetätigt werden können:
–	Schnelllösen 1.	Ein Löseimpuls löst die Schnelllösefunktion des Steuerventils aus.
2.	Der Löseimpuls kann elektromechanisch oder pneumatisch auf das Steuerventil übertragen werden.
–	Bremse aus: 1.	Die Funktion ist bistabil umzusetzen.
2.	Es werden folgende Verbindungen geöffnet bzw. geschlossen:
.	HLL - SV (Entlüftung zum SV)
.	SV - R (Entlüftung zum SV)
.	SV - Cv (Entlüftung zum Relaisventil)
3.	Die Betätigung muss in weniger als 10 Sekunden durchgeführt werden.
–	Bremsstellung 1.	Die Bremsstellung wird elektrisch bistabil durch Verstellen des SV umgestellt.
2.	Eine Rückmeldung ist vorzusehen.\\
1.	Die Feststellbremse kann unabhängig von der pneumatischen Energie im Wagen angelegt und gelöst werden.
2.	Das Anlegen und Lösen erfolgt bistabil durch eletrischen Impuls.
3.	Eine Rückmeldefunktion für den gelösten Zustand ist vorzusehen.
4.	Die Bremskraft am Bremszylinder beträgt 7500 kN (für 2%-Gefälle, 90 t, Klotzbremse).
5.	Alternative Lösungen, wie Federspeicher oder FT Park Lock können vorgeschlagen werden.\par
\textbf{ep-bremsen}\par
1.	Ein ep-Brems-Ventil wird mit der HLL verbunden.
2.	Das Ventil wird zum Bremsen bestromt.
3.	Das Ventil entlüftet die HLL im Wagen in (3,5...5) s von Regelbetriebsdruck auf 3,5 bar.
\textbf{Messung C-Druck}\\
1.	Der Bremszylinderdruck (D-Cruck) wird gemessen.
2.	Der Sensor arbeitet als Stromsensor (4-20 mA).
3.	Der Messbereich ist (0...5) bar.
4.	Die Messunsicherheit und Auflösing ist so gewählt, dass die erste und letzte Bremsstufe (0,45 bar Cv) sicher erkannt werden. Eine Messunsicherheit von 0,05 bar erfüllt diese Anforderung.
\par
\textbf{Steuerventil} \\ Aus Zulassungsgründen wird ein UIC/TSI-kompatibels Steuerventil eingesetzt\\1.	Es wird ein TSI-konformes Steuerventil für die pneumatische Bremse eingesetzt.
2.	Das Steuerventil verfügt über die Bremsstellungen G und P.
3.	Das Steuerventil verfügt über automatisches Schnelllösen.
4.	Das Relaisventil ist vorzugsweise nicht integriert.



\subsection{Kupplung}
Wie in Kapitel \ref{sec:Personal} angedeutet und in Kapitel \ref{sec:LuftumechKup} beschrieben, ist das mechanische Kuppeln und das Kuppeln von Luft aufwändig, körperlich anstrengend und fehlerbehaftet. \par
Zur Automatisierung von mechanischen Kupplungen gibt es Kupplungsrobotoren\footnote{Zum Beispiel die Bahn-Kupplungs-Robotoren BaKuRo und EntKuRo}, dort muss aber weiter händisch Luft gekuppelt werden \textbf{Ist das so?}. Alternativ ist die Automatische Kupplung eine Lösung, aber auch nach deren Einführung im vorigen Jahrhundert ist eine flächendeckende Nutzung im Güterverkehr noch nicht realisiert.\par
\textbf{Irgendwas zu Luft und elektrischen Ventilen.}\\
\textit{siehe auch \ref{sec:Vorpruefung} und \ref{sec:Zugvorbereitung}}
\par
Aufgrund dieser Punkte ist es erst einmal sinnvoll mich der mechanischen Kupplung weiter zu machen und Lösungen für die Automatische Kupplung kompatibel zu halten. Vor allem muss aber die Luftkupplung vereinfacht wreden wo es geht \textbf{ SIEHE AUCH: Luftventile.}

\subsection{Bewegung des Wagens}
In Kapitel \ref{sec:BewdWagen} werden die verschiedenen Möglichkeiten der Wagenbewegung betrachtet. Im Allgemeinen werden dafür zusätzliche Fahrzeuge, Personal und eine Gleisanlage benötigt. Je nach Beschaffung der Gleisanlage und die in Kapitel \ref{sec:Fahrweg} angesprochene Einschränkung, werden zusätzliche Sägefahrten zur korrekten Einsortierung der Wagen auf verschiedene Gleise benötigt. \par
Ein eigener Antrieb auf jedem Wagen, der eine selbstständige Bewegung in geringer Geschwindigkeit zulässt wäre hier eine Lösung.\par
\textit{\textbf{Probleme:} Stromversorgung, Akkuaufladung, Antrieb, Sicherheit}
\subsubsection{Adrücken}
DAs Abdrücken am Ablaufberg, siehe Kapitel \ref{sec:Abdruecken} ist bereits auf großen Rangierbahnhöfen automatisiert, siehe Kapitel \ref{sec:automAbdruecken}, dennoch ist auch dies Handarbeit zur Vorbereitung. Die Wagen müssen vorentkuppelt werden, sowohl mit der Luftkupplung, als auch mechanisch, dann aber wieder, mit einem Hemmschuh, festgelegt werden und kurz vorm Ablaufberg vollständig entkuppelt. 
Hemmschuh, vereinzeln, vorentkuppelt, Geschwindigkeit
\subsection{Sicherheit/Zugintegrität}
\subsubsection{Personalschulung}
Neue Systeme benötigen immer Zeit zur Einführung und Schulung des Personlas. Das gilt auch für dieses Projekt, aber auch jetzt schon sind die Handgriffe, wie in Kapitel \ref{sec:Personal} angesprochen, kompliziert und fehleranfällig. Durch informationstechnische Prozesse \textbf{MEHR DAZU UNTER}, die zusätzlich im Hintergrund laufen und durch weniger Handgriffe im üblichen Betrieb, sollen diese Prozesse einfacher und sicherer werden. Zusätzlich sollen die neu benötigten Handgriffe intuitiv zu den alten Handgriffen passen oder leicht erlenrnbar sein. Zusätzliches Schulungsmaterial soll leicht verständlich und intuitiv gestaltet werden.\par
\textit{\textbf{Probleme:} Annahme der Systeme, Kosten für die Schulung}

\subsection{Informationstechnische Prozesse}
1.	Alle elektrischen Komponenten arbeiten mit 24 V.
\subsubsection{Bremsprobe}
\textbf{SIL-Bewertung - liegt nicht an, liegt an, sicheres lösen, sicheres stellen}.\par
Die \gls{Bremsprobe} findet zur Vorbereitung einer Sperr-, Rangier- oder Zugfahrt statt und überprüft die Funktionsfähigkeit des Bremssystems im Zug- oder Wagenverbund.\par
\begin{figure}[htbp] 
    \begin{center}
            \includegraphics[width=9cm]{Bilder/bremsprobe.png}
            \caption{Schematische Übersicht der Bremsprobe}
            \label{fig:Bremsprobe}
    \end{center}
\end{figure} 
Sie wird von Bremsprobeberechtigten nach genau vorgeschriebenem Ablauf durchgeführt. Ja nach Zustand des Zuges und Fälligkeit der \gls{Bremsprobe} wird eine volle, vereinfachte, stationäre oder Führerraumbremsprobe durchgeführt. Der genaue Ablauf ist in der \acrshort{RIL} 915 oder der VDV-Schrift 757 geregelt und als grobe Übersicht in Abbildung \ref{fig:Bremsprobe} zu sehen.\par
\textit{Durch bekannte Vorprüfungen kann die vollständige \gls{Bremsprobe} vereinfacht werden. Zum Beispiel durch sogenannte vorgeprüfte Gruppen.\\
\textbf{Siehe auch DI, Luftventile\\
\ref{sec:vBremsprobe}, \ref{sec:UEdWagen}, \ref{sec:RangKnoten}}}
\textbf{Siehe auch Bremsprobe im Anhang}
\subsubsection{technische Wagenbehandlung}
Es gibt vier Stufen der technischen Wagenbehandlung. Diese sind der RIL 936 definiert.
\begin{itemize}
    \item TWb Stufe 1: Behandlung vor einer Rangierfahrt
    \item TWb Stufe 2: Prüfung nach Abstellung (PnA)
    \item TWb Stufe 3: Prüfung vor Zugfahrt
    \item TWb Stufe 4: Untersuchung und Qualitätscheck der Wagen
\end{itemize}
Siehe dazu auch Anhang \ref{sec:ATWb}.\par
\textbf{Durch Rechnereinsatz kann Stufe X vereinfacht werden zu ... trotzdem langlaufen?}
\subsubsection{Transportdokumente}
Wie bereits In Kapitel \ref{sec:Transdoc} beschrieben, gibt es die papierlose Transportabwicklung inklusive Gefahrgutdokumenten bisher im LKW-Bereich, allerdings nciht im Bahnsektor. Ein Grund dafür ist die fehlende Dateninfrastruktur.\par
Durch eine durchgängige Stromversorgung auf dem Wagen und einen bahntauglichen Rechner sowie entsprechende Datenverbindungen kann dieses Problem angegangen werden.\par
\textit{\textbf{Siehe dazu auch:} Stromversorgung, Rechner, Datenverbindung, DI (Digitale Identität)}
\subsubsection{Vergleicher}
\ref{sec:Vorpruefung}
\subsubsection{Zugvorbereitung}
\ref{sec:Vorpruefung}