\section{Soll-Zustand}
In diesem Kapitel sollen einzelne Themen aus dem Ist-Zustand herausgegriffen und für das Projekt Güterwagen 4.0 beschrieben werden. Hier werden vorstellbare Punkte ausgeführt, die aber nicht alle zur Standardausrüstung gehören müssen oder sollten. Eine einfache Nachrüstung soll möglich sein.\par
Durch eine Energieversorung als Basis des Güterwagens 4.0 ist es möglich verschiedene Sensorik des Wagens zu speisen, sowie die Kommunikation zwischen einzelnen Güterwagen sowie einem Server und verschiedenen mobilen Endgeräten herstellen, aber ebenfalls ist es möglich eine sporadische Aktorik für die Bremse zu ermöglichen. Die Punkte werden einzeln anhand von Aufgaben aus der Beschreibung des Ist-Zustandes beschrieben. \par
Durch diese Erweiterung des konventionellen Güterwagens soll der Güterwagen ein besseres Arbeitsmittel auf dem Werksgelände und in Zugbildungsanlagen werden.
\subsection{Allgemeines}
Es wird der Sollzustand des Demonstrators beschrieben und nicht des späteren vollständig ausgebauten Güterwagen 4.0.\par
Die Aufbauten auf dem Demonstrator sollen vor Zugfahrten vollständig abschaltbar sein.
\subsection{Energieversorgung}
Aufgrund der Prototypenentwicklung in diesem Projekt erscheint eine 24 V-Spannungsversorgung für dieses Projekt sinnvoll. Sowohl werden alle Grenzen der Berührspannungen im Kleinspannungsbereich eingehalten, als auch gibt es bereits sehr viele Produkte auf dieser Spannungsebene. Namentlich genannt die SPS, aber auch diverse Sensoren und Aktoren.\par
Gepuffert werden soll das System durch eine Batterie. Diese Batterie soll genügend Leistung liefern um sämtliche Daten-, Kommunikations- und Aktorsteuerungesprozesse speisen zu können. Außerdem soll sie genug Energie für den Erprobungsbetrieb bieten. Sie soll mittels Ladegerät nachladbar sein, aber auch eine Schnittstelle für einen Achsdeckelgenerator haben. Zum Aufladen wird dieser wahrscheinlich nicht geeignet sein, da keine entsprechenden Umlaufzyklen während der Testphase möglich sind. Diese sollen aber während des Projektes simuliert werden.\par
Eine Ausstattung als Nachrüstung von konventionellen Güterwagen muss in der Konstruktion bedacht werden.\par
Eine Beladung des Güterwagens und im restlichen Umlauf soll im Projektverlauf simuliert werden.\par
Für die Versorgung von Sensoren und Aktoren ist für dieses Projekt ein Klemmkasten vorzusehen, der in zukünftigen Projekten professionalisiert werden kann. Die Batterie ist in einem Einschubkasten so anzubringen, dass sie von außen zugänglich ist, aber auch vor Steinschlag und ähnlichem während einer Fahrt geschützt ist. Die Verrohrung für Daten- und Stromleitungen ist vorzusehen, sodass diese geschützt sind und klare Linien zur Verlegung von Leitungen zu sehen sind. Ein Anschluss im Anschlusskasten anzusetzen. Für die Batterie ist eine Ladeschnittstelle von 230 V anzusetzen. Sämtliche Anbauten sollen unter ATEX-gerechten Bedingungen angebracht werden, das bedeutet unter anderem, dass keine zur Explosion neigenden Materialien verwendet werden dürfen, kein Gasaustausch stattfindet und alles bahnrobust ausgelegt ist.\par
\subsection{Beladung und Ladungssicherung}
Hier ist für den Demonstrator keine Änderung geplant.
\subsection{Wagenbewegungen}
Wagenbewegungen finden in den folgenden Kapiteln statt:
\begin{itemize}
    \item Kapitel \ref{sec:Wagenwechsel} Wagenwechsel
    \item Kapitel \ref{sec:Zustellfahrt} Zustellfahrt
    \item Kapitel \ref{sec:Zugfahrt} Zugfahrt zum Satellitenbahnhof
    \item Kapitel \ref{sec:Rangierfahrt} Rangiervirgänge im Umsetzbetrieb
    \item Kapitel \ref{sec:RangKnoten} Rangiervorgänge im Knotenbahnhof
    \item Kapitel \ref{sec:Abdruecken} Abdrücken am Ablaufberg
    \item Kapitel \ref{sec:Abdruecken} Automatisches abdrücken
    \item Kapitel \ref{sec:Zugfahrt2} Zugfahrt
    \item Kapitel \ref{sec:Umsetzbetrieb} Umsetzbetrieb
    \item Kapitel \ref{sec:FahrtGA} Fahrt zum Gleisanschluss / zur Ladestelle
    \item Kapitel \ref{sec:RausrangWagen} Rausgangieren einzelner Wagen
\end{itemize}
Hier ist von verschiedenen Bewegungen die Rede. Diese Bewegungen können sein:
\begin{itemize}
    \item Rangieren mittels Lok und mit Luftkupplung
    \item Rangieren mittels Lok und ohne Luftkupplung
    \item Verschieben mittels Rangierhilfsmittel ohne Luftkupplung
\end{itemize}
Hier ist folgende Regelung für die Aktorsteuerung der Ventile in der HL vozusehen:
\begin{itemize}
    \item Ist der Güterwagen Luftgekuppelt, so ist das Ventil in der HL im Normalzustand an der Seite offen
    \item Ist der Güterwagen nicht Luftgekuppelt, so ist das Ventil in der HL im Normalzustand auf der Seite geschlossen
\end{itemize}
Weitere manuelle Steuerungen, beispielweise für Ablaufberge oder manuelle Bremsproben sind vorzusehen.\par
Folgende Regelung gilt für die Feststellbremse:
\begin{itemize}
    \item Die Feststellbremse ist lose, wenn der Wagen Luftgekuppelt ist
    \item Die Feststellbremse ist angezogen, wenn der wagen nicht Luftgekuppelt ist
\end{itemize}
Weitere manuelle Steuerungen sind mittels Endgerät oder Schalter am Wagen schnell und sicher zu ermöglichen. \par
Es ist für eine Stillstandsüberwachung zu sorgen.
\subsection{Wagenzusammenstellungen und -trennungen}
Wagenzusammenstellungen und -trennungen finden in den folgenden Kapiteln statt:
\begin{itemize}
    \item Kapitel \ref{sec:LuftumechKup} Luft- und mechanische Kupplung
    \item Kapitel \ref{sec:vBremsprobe} (Vereinfachte) Bremsprobe
    \item Kapitel \ref{sec:tWb} Technische Wagenbehandlung
    \item Kapitel \ref{sec:Vorpruefung} Vorprüfung und Vorbereitung
    \item Kapitel \ref{sec:Zugvorbereitung} Zugvorbereitung
    \item Kapitel \ref{sec:Nachordnung} Nachordnung
    \item Kapitel \ref{sec:Umsetzbetrieb} Umsetzbetrieb
    \item Kapitel \ref{sec:RausrangWagen} Rausrangieren einzelner Wagen
\end{itemize}
Dazu gehören sowohl die mechansichen Zusammenstellungen als auch die, danach und vor der Wagenbewegung, zugehörigen Bremsproben und technischen Wagenbehandlungen.\par
beim Demonstrator ist keine Automatisierung der mechanischen oder der Luftkupplung geplant.\par
Folgende Regelungen gelten für die Bremsprobe:
\begin{itemize}
    \item Der Wagen soll selbstständig seine Bremsfähigkeit angeben
    \item Der Wagen soll selbstständig seine Bremsfähigkeit erproben
    \item Der Wagen soll selbstständig eine Rückmeldung an ein passendes Gerät geben
    \item Der Wagen soll den Druck im Reservebehälter angeben können
    \item Der Wagen soll den Druck in der HL angeben können
    \item Der Wagen soll den C-Druck angeben können
\end{itemize}
Folgende Regelungen gelten für die technische Wagenbehandlung:
\begin{itemize}
    \item Hier muss was hin
\end{itemize}
Die Vorprüfung der Wagen im automatisierten Rangierbahnhof sollen bei einem Wagenzug, der nur aus ausgerüsteten Güterwagen 4.0 besteht nicht mehr notwendig sein, da die Reihenfolge der Wagen bekannt ist. Solange nicht alle Wagen Güterwagen 4.0 sind, muss dieser Prozess weiter ausgeführt werden.\par
Folgende Regelungen gelten für die Trennstellenanzeige:
\begin{itemize}
    \item Ist auf dem Endgerät eine Trennstelle angewählt, so müssen die HL-Absperrventile sich an diesen Punkte schließen um eine einfachere Luftkupplungs-trennung zu ermöglichen
    \item Die Trennstellenanzeige muss deutlich von allen Seiten des Güterwagens sichtbar sein
    \item Die Trennstellenanzeige muss farblich erkennbar sein
\end{itemize}
Folgende Regelungen gelten für die Bremsstellung:
\begin{itemize}
    \item Der Wagen soll nach Aufforderung seine Bremsstellung in den gewünschten Modus umstellen können
    \item Der Wagen soll abgeschaltete Bremsen selbstständig melden
\end{itemize}
Folgende Regelungen gelten für die Lastwechseleinstellung:
\begin{itemize}
    \item Der Wagen soll nach Aufforderung den Lastwechsel selbst vornehmen können
\end{itemize}
Die Regelungen zur Bremsberechung sind unter XXX zu finden.

\subsection{Zugfahrt}
Volle Bremsprobe
LL Einstellbar

\subsection{Datenhaltung und -übertragung}
Daten im Wagen über separates Kabel, zum nächsten Wagen nur kurze Funkstrecken
Es soll eine Kommunikation relaisiert werden. Sowohl zw. Sensoren und Rechner, Wagen, Wagen und Wagen - Device
Bedienung erfordert Datentechnische Verbindung. Auch zu Device in der ok
Reichweite? ausreichend, nicht von außen störbar, nicht abhörbar
Bandbreite? ausreichend
Zugriffsmöglichkeiten auf Daten nach Authorisierung. Verschiedene Authorisierungsstufen und Geofencing.
\subsection{Diagnosefunktionen}
vorausschauende Wartung
alles über sich wissen, was er wissen muss und speichern und abrufbar machen

\subsection{Wagen als Bestandteil im IT-Güterverkehr}
Wagen als Bestandteil im IT-Güterverkehr
elektronischer Ladezettel
Betriebsinfos lokal auf Wagen
Wartungsinfos in der Cloud

\subsection{sonstige}
Eine Migrationsstrategie ist zu planen\\
Alle Sensoren prüfen - Bereitschaft und Wert iO - und rückmeldung geben
Beispiele: Dreh-(Kugel)Pfanne am Drehgestell (VTG), Bremsbelag überwachung, Lagerzustände (während der Fahrt), erste Meter: Lagerschaden, Flachstelle, Entgleisungssicherheit (vRang = 25 km/h)