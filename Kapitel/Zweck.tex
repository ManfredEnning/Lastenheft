\section*{Zweck des Dokuments}
Das Lastenheft ist, nach \acrshort{DIN} 69901-5: Projektmanagement – Projektmanagementsysteme – Teil 5: Begriffe, ''die vom Auftraggeber festgelegte Gesamtheit der Forderungen an die Lieferungen und Leistungen eines Auftragnehmers innerhalb eines (Projekt-)Auftrags''\cite{DIN69901-5}.\par
Ein Lastenheft ist wichtig in der Analysephase und zur Kommunikation innerhalb des Auftrages/Projekts. Es bietet eine ausführliche Beschreibung der Arbeitsleistung und dient als Kommunikationsbasis.\par
Auf Basis des Lastenheftes wird das Pflichtenheft vom Auftragnehmer erarbeitet.\par
Untersuchungen zeigen, dass Fehler in der Produktentwicklung bei einer späten Aufdeckung und Behebung teurer werden als bei einer früheren Erkennung. Deshalb sollte man schon beim Lastenheft eine hohe Qualität anstreben.\cite{pmblog}\par
In diesem Fall wird das Lastenheft im Rahmen des Projektes "Neue Elektronik- und Kommunikationssysteme für den intelligenten, vernetzten Güterwagen" von der FH Aachen als Vorarbeit für das Pflichtenheft erstellt. \par
Klare gemeinsame Ziele innerhalb des Projektes sollen zu einer besseren und strukturierten Zusammenarbeit führen.
%Das Lastenheft wird aufgrund von Veröffentlichungen der Projektsteller und dem Gesamtverbundantrag des Projektes erstellt