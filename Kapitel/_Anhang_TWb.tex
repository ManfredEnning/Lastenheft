\section{RIL 936 - Technische Wagenbehandlung im Betrieb (Güterwagen) - Auszug}\label{sec:ATWb}
Die folgenden Abschnitte sind Auszüge aus der RIL 936 - Technische Wagenbehandlung im Betrieb (Güterwagen).\cite{RIL936}\par
Die Technischen Wagenbehandlungsarten stellen sicher, dass die im Einsatz befindlichen Wagen betriebssicher
sind. Die Prüfung der Verkehrstauglichkeit ist besonders geregelt.\par 
Die technische Wagenbehandlung darf nur durchgeführt werden, wenn die Bedingungen des Arbeitsschutzes hergestellt
sind.\par
Die technischen Wagenbehandlungsarten werden in folgenden Stufen ihrer Ausführung beschrieben.
\subsection{TWb Stufe 1: Behandlung vor einer Rangierfahrt}
\textbf{Ziel}\par
Durch die Behandlung der Stufe 1 soll der betriebssichere
Zustand der Wagen sowie deren Ladungen und intermodale
Ladeeinheiten (ILE) für die anschließende Rangierfahrt
festgestellt werden.\par
\textbf{Arbeitsumfang}\par
Die Behandlung der Stufe 1 beinhaltet eine Sichtprüfung
der Wagen, Ladungen und ILE auf Schäden und Mängel,
welche die Sicherheit der Rangierfahrt beeinträchtigen –
soweit sie vom Boden aus, neben dem Fahrzeug stehend,
erkennbar sind.\par
Dabei werden Wagen, Ladungen oder ILE nicht betreten
oder geöffnet.\par
Wagen, Ladungen und ILE werden in der Behandlungsstufe
1 augenscheinlich daraufhin geprüft, ob z.B.
\begin{itemize}
    \item die ordnungsgemäße Stellung von Türen, Schiebewände, Hauben, Dächer, Klappen, Sicherungsmittel usw. geschlossen und verriegelt sind, offensichtliche Schäden vorliegen, z. B. durch die Be- oder Entladung bzw.
    \item Eingriffe oder Manipulationen vorliegen,
    \item Tritte, Griffe, Handläufe und Aufstiegsleitern in bestimmungsgemäßem Zustand sind,
    \item kein Ladegut austritt,
    \item lose Wagenbestandteile ordnungsgemäß festgelegt oder befestigt sind,
    \item keine losen Gegenstände auf dem Wagen liegen, die die Betriebssicherheit gefährden können und
    \item Ladungssicherungen nicht beschädigt sind.
\end{itemize}
\textbf{Einzuleitende Maßnahmen und Dokumentation}\par
Bei erkannten Schäden und Mängeln sind Maßnahmen wie
\begin{itemize}
    \item Schaden oder Mangel selbst beheben (z.B. Tür schließen).
    \item Wagen von der Rangierfahrt ausschließen
\end{itemize}
einzuleiten.\par
Können vorgefundene Schäden/ Mängel nicht behoben
werden, sind erforderliche Maßnahmen über die zuständige
Dispostelle einzuleiten.
\subsection{TWb Stufe 2: Prüfung nach Abstellung (PnA}
\textbf{Ziel}\par
Durch die Behandlung der Stufe 2 sollen Einwirkungen Dritter während der Abstellzeit des Zuges (Wagen, Ladungen und ILE) festgestellt bzw. behoben werden, um für die anschließende Zugfahrt (inkl. Feststellen der Fahrbereitschaft) den sicheren Betrieb zu gewährleisten.\par
\textbf{Arbeitsumfang}\par
Die Behandlung Stufe 2 beinhaltet zur Feststellung der
Abfahrbereitschaft eine beidseitige Sichtprüfung der Wagen,
Ladungen und ILE. \par
Wagen, Ladungen und ILE werden augenscheinlich auf offensichtliche
Eingriffe oder Manipulationen geprüft.\par
Bei dieser augenscheinlichen Behandlung ist besonders
darauf zu achten, dass z.B.
\begin{itemize}
    \item Türen, Seitenwände, Dächer und Hauben usw. am Fahrzeug geschlossen und verriegelt sind,
    \item lose/bewegliche Fahrzeugteile festgelegt sind,
    \item Fahrzeuge ordnungsgemäß gekuppelt sind,
    \item Ladungssicherungen nicht beschädigt oder offensichtlich entfernt sind und
    \item dass kein Ladegut austritt.
\end{itemize}
Bei der Behandlung der Stufe 2 muss der Abgleich der ersten und letzten Wagennummer (Wagenliste oder Bremszettel) durchgeführt werden.\par
\textbf{Einzuleitende Maßnahmen und Dokumentation}\par
Bei erkannten Schäden und Mängel sind Abhilfemaßnahmen wie
\begin{itemize}
    \item Schaden oder Mangel selbst beheben (z.B. Tür schließen)
    \item Wagen von der Zugfahrt ausschließen einzuleiten.
\end{itemize}
Festgestellte Schäden und Mängeln sind mit Schadzettel (z.B. Störmeldezettel Tf) zu dokumentieren und dem zuständigen Disponenten zu melden. Können Sie deren Auswirkungen nicht sicher abschätzen oder nicht beheben, ist die weitere Vorgehensweise mit dem zuständigen Disponenten festzulegen.\par
Vor dem Einleiten von Abhilfemaßnahmen nach offensichtlichen Eingriffen oder von Manipulationen an Wagen wie z.B. geöffnete Türen/Verschlüsse oder das Anbringen
von nicht identifizierbaren Gegenständen, ist die weitere Vorgehensweise unverzüglich mit der zuständigen Dispostelle abzustimmen. Weitere Maßnahmen könnten von dem Ergebnis der Spurensicherung durch die Polizeibehörden abhängig sein. \par
Hinweise auf Schäden und Mängel sowie weiterführende Regelungen und Maßnahmen, sind im „Kriterienkatalog für Schäden und Mängel“ der Ril 936ff geregelt.\par
\subsection{TWb Stufe 3: Prüfung vor der Zugfahrt}
\textbf{Ziel} \par
Feststellung des betriebssicheren Zustands der Wagen,
Ladungen und ILE vor der Zugfahrt.\par
\textbf{Varianten} \par
Für die Stufe 3 können verschiedene Varianten erforderlich sein:
\begin{itemize}
    \item Behandlung vor der Zugfahrt - DBCDE Verkehr
    \item Behandlung vor der Zugfahrt – AVV Verkehr
\end{itemize}
\textbf{Besonderheiten} \par
Sendungen, an deren Transport besondere Bedingungen gestellt sind (z.B. außergewöhnliche Sendungen, Militärverkehr (MV) usw.) bedürfen einer vorherigen Wagensonderuntersuchung
(WSU) im Rahmen der Stufe 4 mit Abnahme. Die Dokumentation ist nach Ril 936.0301 vorzunehmen.\par
\textbf{Arbeitsumfang} \par
Die Durchführung der Stufe 3 erfolgt i.d.R. am fertig gebildeten Zug/ Zugteil. Dabei werden Systemdaten grundsätzlich mittels vorhandener mobiler DV überprüft und dokumentiert. Grundsätzlich ist die Durchführung der Stufe 3 mit der Reihung zu verbinden.\par
Die Stufe 3 beinhaltet die Feststellung
\begin{itemize}
    \item des betriebssicheren Zustandes der Fahrzeuge und Ladungen,
    \item das Ladungen und deren Sicherung soweit einsehbar, nicht beschädigt sind,
    \item auf Überladung,
    \item der Einhaltung bestimmter Zugbildungskriterien wie z.B.
    \begin{itemize}
        \item Kuppelzustand allgemein (gekuppelt und geschlaucht),
        \item Prüfung auf das Einstellen nicht zugelassener Wagen (z.B. schwerbeschädigte Wagen),
        \item Prüfung auf außergewöhnliche Sendungen (Stellung im Zug, Schutzabstände),
        \item Prüfung der Schutzabstände bei Gefahrgutsendungen GGVSEB,
        \item Abgleich der Ladungsgewichte sowie Wagenreihungskontrolle.
    \end{itemize}
\end{itemize}
\textbf{Einzuleitende Maßnahmen und Dokumentation} \par
Bei erkannten Schäden und Mängeln sind Abhilfemaßnahmen wie
\begin{itemize}
    \item Schaden oder Mangel selbst beheben (z.B. Tür schließen),
    \item Wagen von der Zugfahrt ausschließen
\end{itemize}
einzuleiten.\par
Festgestellte Schäden und Mängel sind mit dem erforderlichen Schadzettel zu bezetteln, zu dokumentieren und soweit erforderlich dem zuständigen Disponenten zu melden. Können Sie deren Auswirkungen nicht sicher abschätzen oder nicht beheben, ist die weitere Vorgehensweise mit dem zuständigen Disponenten festzulegen.\par
Hinweise auf Schäden und Mängel sowie weiterführende Regelungen und Maßnahmen, sind im „Kriterienkatalog für Schäden und Mängel“ der Ril 936ff geregelt.
\subsection{TWb Stufe 4: Untersuchung und Qualitätscheck Wagen}
\textbf{Ziel} \par
Die Feststellung des betriebssicheren Zustands der Wagen,
Ladungen und ILE.
\begin{itemize}
    \item Beurteilung von Schäden und Mängel und ausführliche Dokumentation,
    \item Prüfung auf uneingeschränkte Nutzbarkeit bzw. Festlegung der weiteren Einsatzkriterien und ausführliche Dokumentation
    \item Abhilfe durch Kleinstschadenbeseitigung oder Behandlung zum Verbleib im Betrieb.
    \item Erfassung und Beschreibung von Schäden zur Arbeitsvorbereitung für die Instandhaltung und Entscheidungsfindung für den Halter.
\end{itemize}
\textbf{Varianten} \par
Für die Stufe 4 können verschiedene Untersuchungen erforderlich sein:
\begin{itemize}
    \item Untersuchung von Wagen am Zug oder Zugteil
    \item Untersuchung von leeren Wagen vor der Beladung am Zug oder Zugteil im kombinierten Verkehr (siehe Ril 936.0103 KV)
    \item Untersuchung von beladenen Wagen am Zug oder Zugteil im Militärverkehr (siehe Ril 936.0104 MV)
\end{itemize}
Für die Durchführung der WSU ist, soweit diese Tätigkeiten nicht im Zeitfenster der Behandlungsart durchgeführt werden können, eine besondere Beauftragung nach Ril 936.0301 (Vordruck 936.0301V32) erforderlich.\par
\textbf{Besonderheiten} \par
Wird eine Untersuchung der Stufe 4 durchgeführt, ersetzt diese die Stufe 1, 2, 3, 3 KV und 3 AVV in jedem Fall. \par
\textbf{Arbeitsumfang} \par
Die Durchführung der Stufe 4 erfolgt i.d.R.am fertig gebildeten Zug/ Zugteil. Dabei werden Systemdaten grundsätzlich mittels vorhandener mobiler DV überprüft und dokumentiert. Grundsätzlich ist die Durchführung der Stufe 4 mit der Reihung zu verbinden. Weiter sind erforderliche Beschädigungs- und Mängelberichte zu erstellen.\par
Die Stufe 4 beinhaltet die Feststellung
\begin{itemize}
    \item des betriebssicheren Zustandes der Fahrzeuge und Ladungen,
    \item das Ladungen und deren Sicherung soweit einsehbar, nicht beschädigt sind,
    \item auf Überladung
    \item bestimmter Zugbildungskriterien wie z.B.
    \begin{itemize}
        \item Kuppelzustand allgemein (gekuppelt und geschlaucht),
        \item Prüfung auf das Einstellen nicht zugelassener Wagen (z.B. schwerbeschädigte Wagen),
        \item Prüfung auf außergewöhnliche Sendungen (Stellung im Zug, Schutzabstände),
        \item Prüfung der Schutzabstände bei Gefahrgutsendungen GGVSEB,
        \item Abgleich der Ladungsgewichte sowie Wagenreihungskontrolle.
    \end{itemize}
\end{itemize}
Die Suche nach verdeckten oder schwer erkennbaren Schäden und Mängeln muss erfolgen, wenn Merkmale an den Bauteilen, die Lage der Bauteile zueinander, Funktionsstörungen oder andere Gründe auf das Vorliegen von Unregelmäßigkeiten schließen lassen. Das dabei erforderliche Messen und Berechnen einzelner Maße ist unter Verwendung von Hilfs- und Messmitteln durchzuführen.\par
\textbf{Einzuleitende Maßnahmen und Dokumentation} \par
Bei erkannten Schäden und Mängeln sind Abhilfemaßnahmen wie
\begin{itemize}
    \item Schaden oder Mangel selbst beheben (siehe Ril 936.13 bzw. 936.95),
    \item Wagen von der Zugfahrt ausschließen
\end{itemize}
einzuleiten.



