\section{Einleitung}
Das Projekt ”Neue Elektronik- und Kommunikationssysteme für den intelligenten, vernetzten Güterwagen" wurde mittels folgender Aussage im Projektantrag motiviert: "Der Güterverkehr in Deutschland wird bislang zu über 70\% von LKWs gestemmt, was Verkehrsstaus und hohe Schadstoffemissionen verursacht. Das Zukunftsprojekt Industrie 4.0 bietet dem Schienengüterverkehr die einzigartige Chance, durch intelligente Steuerung und Vernetzung Transportprozesse extrem flexibel, effizient und schadstoffarm zu gestalten. Realisiert werden kann dies durch die Integration vielfältiger moderner Sensorik und Elektronik in den Güterwagen 4.0." \cite{AZAP} \par
In diesem Lastenheft wird anhand eines Schiebewandwagens der grobe Ablauf eines Umlaufs im bisherigen System erläutert und daraufhin Anforderungen an den neuen Güterwagen erstellt. Als Motivation zeigen sich eine Erhöhung der Prozesssicherheit, Verringerung der aufzuwendenden Zeit am Wagen, eine Gestaltung von attraktiveren Arbeitsplätzen durch eine ergonomischere Arbeitsplatzgestaltung und mögliche Kosteneinsparungen an der Infrastruktur.\par
Herausforderungen wie die Migration und Annahme des Systems sollen beachtet werden. Die Kosten des Systems, sowie dessen Einbau müssen konkretisiert werden. Prozesse für produktivere Arbeitsabläufe mit dem neuen Güterwagen müssen gestaltet und Arbeitsabläufe bei Defekten konkretisiert werden.\par
Es geht bei dieser Art der Automatisierung nicht darum Arbeitsplätze abzuschaffen, sondern vor allem darum sie attraktiver zu gestalten, um das Problem des Personalmangels in den Griff zu bekommen und neues, junges Personal zu finden. Gleichzeitig darf natürlich nicht die Sicherheit des Systems leiden. Das Sicherheitslevel bzw. die Sicherheitsanforderungsstufe muss mindestens genauso hoch bleiben, wie sie beim bisherigen System ist. Durch diese Erweiterung des konventionellen Güterwagens soll der Güterwagen ein besseres Arbeitsmittel auf dem Werksgelände und in Zugbildungsanlagen werden.\par

Das Lastenheft beginnt mit der Beschreibung des aktuellen Zustandes in Kapitel \ref{sec:Istzustand} Ist-Zustand. In diesem wird ein beispielhafter Umlauf eines Güterwagens aufgezeigt. Er beginnt mit den Vorgängen auf dem Werksgelände und der Ladestelle und führt über Knotenbahnhöfe und Zugbildungsanlagen sowie Rangierbahnhöfe wieder zur Ladestelle und wird dort entladen.\par
Im Anhang \ref{sec:realeIst} lässt sich dazu noch ein weiterer beispielhafter Prozess auf dem Werksgelände der IDR finden. Dieser besitzt den Fokus auf der Kommunikation der einzelnen Partein sowie der Beschreibung von Besonderheiten.\par
Nach dem Ist-Zustand folgt in Kapitel \ref{sec:Abgrenzung} die Abgrenzung des Projektes. In diesem wird der geplante Weg der Automatisierung des Bahnverkehrs im Allgemeinen gezeigt, gefolgt von der Beschreibung der Ausbaustufen des aktiven, kommunikativen Güterwagens. Danach erfolgt die Beschreibung der Abgrenzung des Güterwagen 4.0 vom konventionellen Güterwagen sowie eine kurze Beschreibung weiterer Möglichkeiten des Güterwagen 4.0.\par
In Kapitel \ref{sec:Soll} Soll-Zustand und Anforderungen wird der Ausbau des Demonstrators für das Projekt als Soll-Zustand beschrieben und mit den Anforderungen konkretisiert.
