\section{Ist-Zustand}\label{sec:Istzustand} %Beschreibung wo Produktivität vergeudet wird um daraus ableiten zu können, was benötigt wird. % Kochsiek, Knoll nach Meinung zu diesem Kapitel fragen!
Im Folgenden wir der Umlauf eines \gls{Schiebewandwagen} (Habinns) beschrieben. Der Wagen wird in einem \gls{Gleisanschluss}/\gls{RailPort} mit palettiertem Gut beladen, läuft dann durch das deutsche/europäische Einzelwagensystem und wird in einer Ladestelle an einem \gls{Gleisanschluss} oder \gls{RailPort} entladen. Fokus dieser Beschreibung sind die durchzuführenden manuellen Tätigkeiten.\par
Davon leitet sich im Folgenden ein Anforderungskatalog an einen aktiven, kommunikativen \gls{Gueterwagen 40} ab, der viele oder alle der manuellen Tätigkeiten durch Technikfunktionen ersetzt.\par
Die hier beschriebenen Prozesse sind leicht idealisiert. So werden Störfälle zunächt ausgenommen. Auch kann davon ausgegangen werden, dass jeder Prozess leicht anders aussieht. Trotzdem zeigt er viele wichtige Punkte, die verbessert werden können und sollten. Bei anderen Wagen- und Ladungsarten unterscheidet sich der Umlauf natürlich noch weiter.\par
Weitere Informationen zu anderen Prozessen und vor allem auch zu Prozessen in der Disposition sind im Anhang \ref{sec:realeIst} zu finden. \par
\subsection{Vorgänge an der Ladestelle/im Gleisanschluss}\label{sec:Ladestelle}
%In diesem Unterkapitel wird eine kurze Beschreibung der Fahrwege, Bewegung von Fahrzeugen, der Personaltätigkeit sowie der Beladung, dem Wagenwechsel, der Ladungssicherung und der benötigten Transportdokumente gegeben.
\subsubsection{Fahrweg} \label{sec:Fahrweg}
Die Bewegung von Wagen erfolgt prinzipiell auf Gleisen. Eine Verzweigung von Fahrwegen erfordert Weichen. In kleinen und mittleren Gleisanschlüssen sind dies überwiegend handbetätigte Weichen. 
\subsubsection{Bewegung der Wagen} \label{sec:BewdWagen}
Je nach Wagenaufkommen kommen folgende Methoden der Wagenbewegung in Betracht:
\begin{itemize}
	\item Bedienung durch das \gls{Eisenbahnverkehrsunternehmen} \acrshort{EVU}, welches auch die Zustellfahrten durchführt
	\item Eigene (zum \gls{Gleisanschluss} zugehörige) Rangierlokomotive
	\item Eigenes Rangierhilfsmittel
	\begin{itemize}
	    \item Gleisfahrbar (Rangierroboter)
	    \item \gls{Zweiwegefahrzeug}
	\end{itemize}
	\item Stationäre Verschubeinrichtung (Seilzuganlage)
	\item Verschub mittels Muskelkraft von Tieren oder Menschen (z.B. Knippstange, Wagenrücker)
\end{itemize}
\subsubsection{Personalfähigkeiten}\label{sec:Personal}
Allen Methoden ist gemeinsam, dass Sie spezielle Personalfähigkeiten bzw. eine entsprechende Ausbildung benötigen. Dies ist nicht zuletzt auf die Tatsache zurückzuführen, dass ein einmal in Bewegung gesetzter Güterwagen auch bei langsamer Fahrt eine hohe kinetische Energie aufweist. Zusätzlich ist die Bedienung der Wagenbremsen (Luft- und/oder Handbremse) für ungeschultes Personal kompliziert und fehleranfällig.
\subsubsection{Bereitstellung}
Eine Bereitstellung der Wagen findet aus einem \gls{Vorbahnhof} als Einfahrgruppe oder Ordnungsgruppe statt. Diese stehen dort im Allgemeinen mit entlüfteter Bremse und vorgelegtem Hemmschuh oder angelegter Handbremse.
\subsubsection{Beladung}
Das Beladen von Güterwagen mit Paletten erfolgt von einer Rampe oder einer Ladekante mit Überfahrbrücke aus; in der Regel per Gabelstapler. Im Gegensatz zur Heckbeladung von LKW existieren kaum Lösungen zur Automatisierung der Beladung. Dies ist aber ein Logistikprozess des Verladers und wird hier nicht weiter behandelt.
\subsubsection{Ladungssicherung}
Während oder nach der Beladung muss die Ladung gegen Verrutschen gesichert werden. Bei Schiebewandwagen erfolgt dies unter anderem durch von Hand verschiebbare und durch Verriegeln zu sichernde Zwischenwände.\par
Auch für diese Aufgabe ist der Verlader verantwortlich. Er muss den Wagen als bahntechnisch sicher an das \acrshort{EVU} übergeben. \par
Das \acrshort{EVU} darf nur mit einem augenscheinlich (also für den Beobachter sicheren) Wagen fahren. Dies betrifft in einem geschlossenen Wagen nicht die Ladung, aber die Türen und Verschlüsse; auf einem offenen (Holz-)Transporter aber auch die sichtbare Ladung. Bei Gefahrgut müssen noch weitere Kontrollen von geschultem Personal durchgeführt und der Wagen entsprechend gekennzeichnet werden. Die beschriebenen Vorgänge sind Teil der "`Technischen Wagenbehandlung"' (siehe \ref{sec:tWb}).
\subsubsection{Wagenwechsel}\label{sec:Wagenwechsel}
Eine Ladekante ist meist für Einzelwagen oder kleine Gruppen (bis zu 4 Wagen) gestaltet, so dass häufig Wagen an der Ladestelle getauscht werden müssen, bevor die Zustellfahrt erfolgt. Dies liegt unter anderem an der Seitenbeladung, für die viel Platz benötigt wird. Diese hat aber, gegenüber der Heckbeladung bei LKW, den Vorteil der Parallelisierbarkeit.% LKW gehen nach der Beladung direkt in den Umlauf, sodass bei diesen das häufige Umsetzen nicht als Nachteil gesehen wird. 
\par
Beim Beladen von Wagen mit Gefahrgut ist es üblich, die Weichen vor und hinter der Beladestelle, in der dem Beladegleis abgewandten Position, abzuschließen. Dies soll ein ungewolltes Bewegen der Wagen verhindern. Diese Weichen werden vom Verlader erst nach Beenden der Verladearbeiten wieder geöffnet.\par
Wagen können grundsätzlich rangiert oder verschoben werden. Findet die Wagenbewegung mit \gls{Lokrangierfuehrer} (\acrshort{LRF}) und Lok statt, wird rangiert. Verschoben werden die Wagen wenn die Wagenbewegung ohne Lok und Lokrangierführer stattfindet.\par
Rangiert wird wenn möglich ohne Luftkupplung und entsprechend auch ohne \gls{Bremsprobe}. Ob Rangieren ohne Luftkupplung möglich ist, hängt von der Lokomotive, der Last, der Achsanzahl der zu rangierenden Wagen und der Neigung des Ladegleises ab. Muss aufgrund dieser Eigenschaften mit Luft gefahren werden, ist auch eine vereinfachte \gls{Bremsprobe} notwendig. Häufig findet diese Rangierfahrt in der Bremsstellung G statt.\par
Im Allgemeinen werden einzelne Wagen oder kleinere Wagengruppen für nur wenige Meter verschoben. Dies kann zum Beispiel mit einem \gls{Zweiwegefahrzeug} auf LKW-Basis stattfinden. Hier verschiebt ein angelernter und unterwiesener Verlademitarbeiter mit einem zusätzlichen Sicherheitsposten die Wagen\footnote{Der Verlademitarbeiter wird vom \gls{Eisenbahnbetriebsleiter} (\acrshort{EBL}) angelernt und unterwiesen.}.
\subsubsection{Transportdokumente}\label{sec:Transdoc}
Im LKW-Bereich entstehen bereits Lösungen zur papierlosen Transportabwicklung einschließlich der Behandlung von Gefahrgutdokumenten. Bei Bahntransporten herrscht die klassische Methode der Übergabe von Frachtdokumenten an das \acrshort{EVU} vor. Oft werden diese aber auch direkt elektronisch übergeben. Dann sind sie später digital verfügbar. Ausgenommen sind Gefahrgutscheine, diese werden immer in Papierform mitgeführt.\par
Auch die Wagen werden -- vor allem im Einzelwagenverkehr -- noch "`bezettelt"'\footnote{Siehe dazu auch VDV-Schrift 758 - Prüfen von Güterwagen im Eisenbahnbetrieb}. Diese Zettel sind Vereinfachungen des Frachtbriefes, die direkt auf dem Wagen mitgeführt werden. Auf diesen steht im Allgemeinen, aus was die Ladung besteht, woher diese kommt und wohin sie geht. Weitere Besonderheiten werden hier ebenfalls vermerkt.
%Der Lkw wird meist in Längsrichtung durch das Heck an einem Tor / Vorsatzschleuse beladen. Für die eigentliche Beladung ist es  ungünstiger als die Seitenbeladung, es führt aber zu einer effizienten Flächennutzung in der Halle (Zeilenstruktur). Daher ist die Heckbeladung gerade bei Logistik/Verteilzentren und Speditionen heute die vorherrschende Methode.  
%Für spezielle Güter / Wagen ist Beladung mit Hallenkran von oben üblich (Papier, Stahlcoils)
\subsection{Abholen im Gleisanschluss}
Die Abholung von Wagen erfolgt entweder durch das \acrshort{EVU} unmittelbar an der Ladestelle oder, wenn lokale Rangierhilfsmittel zur Verfügung stehen,von einem Übergabegleis im Werksgelände. Die Bedienung des Anschlusses ist in der Regel Teil eines Umlaufs, in dem mehrere Anschlüsse nacheinander bedient werden.
\subsubsection{Luft- und mechanische Kupplung}\label{sec:LuftumechKup}
Der oder die Wagen stehen im festgelegten Zustand zur Abholung bereit. Die Luftbremse ist gewöhnlich außer Funktion, der Reserveluftbehälter ist leer. Nach dem Ansetzen der Lok bzw. des aktuell letzten Wagens der Rangierabteilung ist manuell zu kuppeln. Der Lokführer oder Rangierbegleiter betritt den Berner Raum und verbindet zunächst manuell die Schraubenkupplung. Danach wird die Hauptluftleitung gekuppelt und die Absperrhähne der \acrshort{HL} geöffnet. %prüfen
\subsubsection{(Vereinfachte) Bremsprobe}\label{sec:vBremsprobe}
Im Anschluss erfolgt eine (vereinfachte) \gls{Bremsprobe}. Dazu wird zunächst am letzten Wagen der gelöst-Zustand überprüft, dann der \acrshort{HL}-Druck abgesenkt und das Führerbremsventil abgesperrt. Die Bremsen müssen anlegen.\par
Im Anschluss wird zur Durchgängigkeitsprüfung wieder die Fahrstellung eingenommen und das \acrshort{HL}-Absperrventil des letzten Wagen für mindestens 15 Sekunden geöffnet. Die Bremsen müssen anlegen und wieder lösen\footnote{Siehe dazu auch RIL 915 - Bremsen im Betrieb bedienen und prüfen}.
\subsubsection{Technische Wagenbehandlung}\label{sec:tWb}
Vor Beginn der Rangier-/ \gls{Sperrfahrt} ist eine \gls{technische Wagenbehandlung} (\acrshort{twb}) der Stufe 1 %(siehe dazu auch Anhang \ref{sec:ATWb}) 
durchzuführen\footnote{Siehe dazu auch RIL 936 - Technische Wagenbehandlung im Betrieb (Güterwagen)}. 
\subsubsection{Zustellfahrt}\label{sec:Zustellfahrt}
%Erfolgt die Zustellfahrt über die freie Strecke, ist das Streckengleis durch das Stellwerk für \gls{Zugfahrt}en zu sperren.\par
Wenn der Gleisanschluss selbst nicht als Bahnhof ausgebildet ist, also keine eigenen Ausfahrsignale besitzt, ist die Strecke für Zugfahrten zu sperren und die Fahrt  vom Satellitenbahnhof zum Anschluss ist eine  "`\gls{Sperrfahrt}"'.\par
Wenn der \gls{Gleisanschluss} an ein Bahnhofsgleis angeschlossen ist, handelt es sich um eine \gls{Rangierfahrt}.\par
In jedem Fall ist es eine Fahrt auf Sicht mit geringer Geschwindigkeit. Worst Case für Streckenkapazität wäre eine geschobene \gls{Sperrfahrt}; diese ist nicht nur mit geringer Geschwindigkeit sondern auch mit höherem Personalaufwand durch Besetzung der Spitze verbunden.

\subsection{Fahrt zum Knotenbahnhof und Zugbildung}
\subsubsection{Zugfahrt zum Satellitenbahnhof}\label{sec:Zugfahrt}
Satellitenbahnhöfe sind Bahnhöfe, in denen im Einzelwagenverkehr \gls{Zugfahrt}en enden oder beginnen. Sie sind in der Regel für die Übergabegruppe nur Durchgangsstation. Die Kombination der Übergabe mit weiteren Übergaben von anderen Anschlüssen/Satelliten erfolgt im \gls{Knotenbahnhof}. Weil die Fahrt vom Satelliten- zum \gls{Knotenbahnhof} eine \gls{Zugfahrt} (mit in der Regel Geschwindigkeiten $\ge 80 km/h$) darstellt, ist vor Beginn der Fahrt eine Berechnung des Bremsgewichtes und eine volle \gls{Bremsprobe} durchzuführen.%erfragen
\subsubsection{Rangiervorgänge im Umsetzbetrieb}\label{sec:Rangierfahrt}
Knotenbahnhöfe können über \gls{Ablaufberg}e verfügen. In der Regel ist dies aber nicht der Fall und die Rangiervorgänge finden im so genannten Umsetzbetrieb statt. Dabei werden jeweils Wagengruppen angekuppelt (je nach Lok ist die Kupplung der Luft meist nicht notwendig) und durch eine Sägefahrt in ein anderes Gleis versetzt und dort gekuppelt. Das Stellen der Weichen für diese Vorgänge erfolgt meist vom Stellwerk des Bahnhofs aus oder durch eine \gls{EOW}\footnote{Elektrisch ortsgestellte Weichen}-Anlage.
\subsubsection{Übergabe der Wagen}\label{sec:UEdWagen}
Wie bei der Abholung aus dem Gleis muss auch vor der Abfahrt des aus Übergaben zusammengestellten Zuges eine \gls{Bremsprobe} und eine \gls{technische Wagenbehandlung} durchgeführt werden. Außerdem sind erneut Bremsgewicht und Bremshunderstel zu berechnen. Bei jeder Zugzusammenstellung sind Papiere zu prüfen und zu übergeben.

\subsection{Fahrt zum Knotenbahnhof des Zielbereichs über einen oder mehrere Rangierbahnhöfe}
\subsubsection{Rangiervorgänge im Knotenbahnhof}\label{sec:RangKnoten}
Ab dem \gls{Knotenbahnhof} sind die weiteren Fahrten/Umstellvorgänge wie folgt charakterisiert:
\begin{itemize}
    \item Züge haben idealerweise die volle Länge von $740~m$
    \item Zugumstellungen erfolgen in den großen Rangierbahnhöfen
\end{itemize}
Wie bei einem Postverteilsystem haben die Rangierbahnhöfe die Aufgabe, hereinkommende Züge aufzulösen und die Wagen auf neue Züge umzustellen, die sie ihrem Ziel näher bringen. Dieser Abschnitt des Einzelwagenverkehrs ist hochautomatisiert und \mbox{-produktiv}. Dennoch verbleiben auch hier große Automatisierungslücken. 
\subsubsection{Vorprüfung und Vorbehandlung}\label{sec:Vorpruefung}
Am Beispiel der Behandlung der Wagen in einem automatisierten \gls{Rangierbahnhof} werden diese Automatisierungslücken deutlich:\par
Nach der Einfahrt des Zuges in die Einfahrgruppe wird durch den so genannten "`Vergleicher"' die Übereinstimmung der übermittelten Zugliste mit der tatsächlichen Reihung geprüft. Grund für diesen scheinbar überflüssigen Schritt ist die Tatsache, dass das Herausrangieren von fehlerhaften Wagen auf der Strecke (meist in der Folge einer Alarmmeldung einer \gls{Heisslaeuferortungsanlage}) nicht in jedem Fall technisch übermittelt wird.\par
Die weiteren Schritte bei der Vorbereitung sind wie folgt:
\begin{enumerate}
    \item Zugbremse anlegen, \acrshort{HL} entlüften, Lok abkuppeln
    \item Zug festlegen (in der Regel durch \gls{Hemmschuh}e)
    \item Wagen und Wagengruppen vereinzeln. Dafür an den vorgesehenen Trennstellen nach Zerlegeliste Luft- und mechanische Schraubenkupplungen trennen und Luftbremse durch Ziehen am Lösezug lösen (Entlüften der A-Kammer/des Reserveluftbehälters). Je nach Rangierverfahren wird die Schraubenkupplung komplett ausgehängt (vorentkuppelt abdrücken) oder sie wird lose eingehängt gelassen und erst am \gls{Ablaufberg} durch den "`Stangler"' ausgeworfen.
\end{enumerate}
\subsubsection{Abdrücken am Ablaufberg}\label{sec:Abdruecken}
Wenn die in Kapitel \ref{sec:Vorpruefung} beschriebenen Vorbereitungen abgeschlossen sind, wird der Zug zum Abdrücken freigegeben. Die Abdrücklokomotive schiebt dann, nach dem Entfernen der Hemmschuhe, den Zug mit Geschwindigkeiten von $1,2\text{ bis }3~m/s$ ($4,3\text{ bis }10,8~km/h$; Schrittgeschwindigkeit) %ist das so?
über den \gls{Ablaufberg}. Hinter dem Gipfel vereinzeln sich die Wagen. Durch den Schwerkrafteinfluss werden diese durch ein gestaffeltes System von mechanischen Gleisbremsen auf die Eintrittsgeschwindigkeit im Richtungsgleis heruntergebremst und laufen langsam in die Richtungsgleise ein.\par
Klassisch ist dies das Arbeitsfeld der Hemmschuhleger, die durch gezieltes Platzieren von Hemmschuhen die Wagen punktgenau vor dem letzten Wagen im Richtungsgleis abbremsen.
\subsubsection{Automatisiertes Abdrücken}\label{sec:automAbdruecken}
In den großen Rangierbahnhöfen sind die Fortschritte der Automatisierungstechnik sichtbar. In allen Anlagen gibt es automatische Steuerungen der Verteilweichen. In einigen Anlagen sind die Ablaufsteuerungen mit einer automatischen Steuerung der Abdrücklokomotiven gekoppelt, so dass während des Abdrückvorgangs kein manueller Steuereingriff notwendig ist; lediglich die Rückfahrt der Lok zum nächsten Einsatz bleibt Handarbeit.\par 
In allen Anlagen werden die Bremsen (\gls{Bergbremse} und \gls{Talbremse}) geregelt betrieben, so dass unterschiedliche Laufeigenschaften der Wagen und Windeinflüsse automatisch kompensiert werden und in den meisten Anlagen ist das Legen von Hemmschuhen zur Zielbremsung ersetzt worden durch automatische Fördereinrichtungen, die so genannten Räum- und Beidrückförderer.\par
Wenn dieser hoch automatisierte Prozess für den betrachteten Wagen im Richtungsgleis abgeschlossen ist, geht es wieder in Handarbeit weiter.
\subsubsection{Zugvorbereitung}\label{sec:Zugvorbereitung}
Nach dem Schließen des Richtungsgleises für weitere Abläufe werden die Wagen provisorisch miteinander gekuppelt und die Wagengruppe wird in ein Gleis für die Nachbehandlung gezogen. In einigen Anlagen erfolgt die Nachbehandlung im Richtungsgleis. Dort werden die folgenden Schritte durchgeführt:
\begin{enumerate}
    \item Mechanisch kuppeln
    \item Luftleitungen kuppeln, \acrshort{HL}-Absperrhähne öffnen
    \item \acrshort{HL}-Absperrhahn am (neuen) \gls{Zugschluss} schließen, Zugschlusstafel stecken
    \item Bremsberechnung (Zuggewicht, Bremsgewicht, Bremseigenschaften der Lok)
    \item Bremsstellungswechsel je nach \gls{Bremsart} der folgenden \gls{Zugfahrt}
    \item Volle \gls{Bremsprobe}, in der Regel mittels stationärer \gls{Bremsprobeanlage}
    \item \gls{technische Wagenbehandlung}, Stufe 3
\end{enumerate}
\subsubsection{Nachordnung}\label{sec:Nachordnung}
Falls die örtlichen Verhältnisse in den Zielgleisanschlüssen und die Reihenfolge von deren Bedienung bekannt sind, erfolgt noch eine Nachordnung der Wagen nach Empfängern durch einen erneuten Berglauf oder in einer Nachordnungsgruppe. Dieses ist effizienter als manuelle Umstellvorgänge mittels Sägefahrten in Knoten oder Satellitenbahnhöfen.

\subsection{Fahrt zum Satellitenbahnhof}
\subsubsection{Zugfahrt}\label{sec:Zugfahrt2}
Der im Knotenbahnhof vorbereitete Zug wird dann für den nächsten Abschnitt von der vorgesehenen Streckenlok übernommen. Nach einer vereinfachten \gls{Bremsprobe} kann die \gls{Zugfahrt} beginnen.\par
\subsubsection{Umsetzbetrieb}\label{sec:Umsetzbetrieb}
Am Satellitenbahnhof angekommen wird der Zug wieder, wie oben beschrieben, im Umsetzbetrieb umgestellt. Dabei werden
jeweils Wagengruppen angekuppelt (je nach Lok ist die Kupplung der Luft meist nicht notwendig) und durch eine Sägefahrt in ein anderes Gleis versetzt und dort gekuppelt.
\subsubsection{Übergabe der Wagen/Zugvorbereitung}
Auch  vor  dieser  Abfahrt  wird der  zusammengestellte  Zug wieder mittels \gls{Bremsprobe} und technischer Wagenbehandlung geprüft, ebenso die dazugehörigen Papiere.

\subsection{Fahrt zum Gleisanschluss/zur Ladestelle}\label{sec:FahrtGA}
Wenn die Zustellfahrt über die freie Strecke erfolgt, ist das Streckengleis für Zugfahrten zu sperren. Diese Fahrt kann als Sperrfahrt oder Rangierfahrt stattfinden.
\subsubsection{Rausrangieren einzelner Wagen}\label{sec:RausrangWagen}
Angekommen am \gls{Gleisanschluss} müssen einzelne Wagen wieder aus dem \gls{Zugverband} rausgangiert oder zumindest abgekuppelt werden.
\subsubsection{Entladen}
Danach wird der Wagen zur Entladestelle gebracht und entladen.