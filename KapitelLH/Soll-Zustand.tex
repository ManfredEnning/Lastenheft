\section{Soll-Zustand und Anforderungen}
%In diesem Kapitel wird der Soll-Zustand des Labormusters und des Demonstrators beschrieben. Dafür werden einzelne Themen aus dem Ist-Zustand herausgegriffen und für das Projekt Güterwagen 4.0 beschrieben.\par
In diesem Kapitel findet die Beschreibung des Soll-Zustandes gleichzeitig %Gleichzeitig findet die 
mit der Beschreibung der Anforderungen statt. Diese bestehen immer aus einer Anforderungsnummer und einem Anforderungstext sowie eventuell dazugehörigen Notizen.\par
%Die Anforderungen ergeben den Sollzustand.

\subsection{Allgemeines}
Für alle Aspekte ist eine spätere Zulassung des System zu berücksichtigen. Dem entsprechend ist auf eine Beachtung der folgenden Normen zu achten. Labormuster und \gls{Demonstrator} sind aber als Prototyp zu sehen, sodass bei einzelnen Prototypenentwicklungen auch davon abgewichen werden kann. \par
Alle Komponenten sind schwingungsarm unterzubringen. Dies soll, für den Demonstrator, die Möglichkeit geben auch Komponenten, die nicht aus dem Bahnbereich stammen, verwenden zu können.
\begin{feat}
Der Güterwagen muss alle Anforderungen an einen konventionellen Gü- terwagen ebenfalls erfüllen.
\end{feat}
\begin{feat}
Die \gls{40-Komponenten} auf dem \gls{Demonstrator} müssen vor Zugfahrten voll- ständig abschaltbar sein.
\end{feat}
\begin{feat}
Die \gls{40-Komponenten} auf dem \gls{Demonstrator} müssen vollständig nachrüst- bar sein.
\end{feat}
\begin{feat}
Für Zulassungen ist die CSM-VO - Verordnung (EG) Nr. 402/2013 anzuwenden.
\end{feat}
\begin{feat}
Die \acrshort{DIN} \acrshort{EN} 50155:2018 Bahnanwendungen - Elektronische Einrichtungen auf Schienenfahrzeugen ist anzuwenden
\end{feat}
\begin{feat}
Die \acrshort{DIN} \acrshort{EN} 61373:2011-04 Bahnanwendungen - Betriebsmittel von Bahnfahrzeugen - Prüfungen für Schwingen und Schocken ist anzuwenden.
\end{feat}
\begin{feat}
Die Technische Richtlinie zur EMV Verträglichkeit TR-EMV ist anzuwenden.
\end{feat}
\begin{feat}
Die \acrshort{DIN} \acrshort{EN} 45545 Brandschutz in Schienenfahrzeugen ist anzuwenden
\end{feat}
\begin{rem}[zu Anf. 6-8]
Möglich wäre beispielsweise eine schwingungsarme Metallbox, die sowohl die EMV- als auch die Brandschutzrichtlinien erfüllt.
\end{rem}

\subsection{Energieversorgung}\label{sec:EV}
Die Energieversorgung des Bordnetzes erfolgt mittels Systembatterie. Diese wird durch externe Versorgung gespeist. Weil im Rahmen des F\&E-Projekts keine Umläufe mit kalkulierbarem Streckenanteil geplant sind, wird die Batterie in der Regel im Stand mit Hilfe eines Netzgerätes geladen. Der Demonstrator soll darüber hinaus die Möglichkeit der Aufladung durch einen Achsdeckelgenerator beinhalten (Anpassung für unterschiedliche Einspeisespannungen). Siehe dazu auch Abbildung \ref{fig:Soll-EV}. 
\begin{figure}[htp]
    \centering
    %\tikzset{wagon/.style={draw = gray, ultra thick, opacity = 0.7}}

\begin{tikzpicture}[font = \sffamily, scale = 0.8]
\tikzstyle{every node}=[font=\small]
%Bordnetz
\path[wagon] (-6,0) -- (6,0) {};
\node (BN) at (0,0.3) {Bordnetz};
\path[wagon] (-6,1) -- (-5,1) -- (-5,-1) -- (-6,-1){};
\fill(-5,0)circle(2pt);
\path[wagon] (6,1) -- (5,1) -- (5,-1) -- (6,-1){};
\fill(5,0)circle(2pt);

%Netzspannung
\path[wagon] (-3.75,0) -- (-3.75,-2.5) {};
\fill(-3.75,0)circle(2pt);
\path[wagon] (-5,-2.5) -- (-2.5,-2.5) -- (-2.5,-4.5) -- (-5,-4.5) -- (-5,-2.5){};
\path[wagon] (-5,-4.5) -- (-2.5,-2.5) {};
\node (=) at (-4.4, -3.2) {=};
\node (-) at (-3.1, -3.9) {\huge{\textasciitilde}};
\path[wagon] (-3.75,-4.5) -- (-3.75,-5.5) {};
\node (NS) at (-3.75,-5.7) {Netzspannung};

%Achsdeckelgenerator
\path[wagon] (0,0) -- (0,-2.5) {};
\fill(0,0)circle(2pt);
\path[wagon] (-1.25,-2.5) -- (1.25,-2.5) -- (1.25,-4.5) -- (-1.25,-4.5) -- (-1.25,-2.5){};
\node (AE) at (0, -3.2) {Anpassungs-};
\node (AE2) at (0, -3.9) {elektronik};
\path[wagon] (0,-4.5) -- (0,-5.5) {};
\node (NS) at (0,-5.7) {Achsdeckelgenerator};

%Batterie
\path[wagon] (3.75,0) -- (3.75,-2.5) {};
\fill(3.75,0)circle(2pt);
\path[wagon] (5,-2.5) -- (2.5,-2.5) -- (2.5,-4.5) -- (5,-4.5) -- (5,-2.5){};
\node (Bat) at (3.75, -3.2) {System-};
\node (Bat2) at (3.75, -3.9) {batterie};

\end{tikzpicture}
    \caption{Beispielhafte Energieversorgung des Güterwagen 4.0}
    \label{fig:Soll-EV}
\end{figure}

\subsubsection{Spannungsversorgung}
\begin{feat}
Die Nennspannung des Systems ist mit 24 V vorzusehen
\end{feat}
\begin{rem}[zu Anf. 9]
Aufgrund der Prototypenentwicklung in diesem Projekt erscheint eine 24 V-Spannungsversorgung für dieses Projekt sinnvoll. Sowohl werden alle Grenzen der Berührspannungen im Kleinspannungsbereich eingehalten, als auch gibt es bereits sehr viele Produkte auf dieser Spannungsebene. Namentlich genannt die SPS, aber auch diverse Sensoren und Aktoren.
\end{rem}
\subsubsection{Batterie}
%Gepuffert werden soll das System durch eine Batterie. Diese Batterie soll genügend Leistung liefern um sämtliche Daten-, Kommunikations- und Aktorsteuerungsprozesse speisen zu können. Außerdem soll sie genügend Energie für den Erprobungsbetrieb bieten.\par
%Die Batterie soll mittels Ladegerät von 230 V nachladbar sein, aber auch eine Schnittstelle für einen Achsdeckelgenerator bieten. Zum Aufladen wird dieser wahrscheinlich nicht geeignet sein, da keine entsprechenden Umlaufzyklen während der Testphase möglich sind. Die Batterie ist in einem Einschubkasten so anzubringen, dass sie von außen zugänglich, aber auch vor Steinschlag und ähnlichem während einer Fahrt geschützt ist.\par
\begin{feat}
Es ist eine Systembatterie vorzusehen.
\end{feat}
\begin{rem}[zu Anf. 10]
Die Batterie soll genügend Leistung, Spannung und Energie für sämtliche Daten-, Kommunikations- und Aktorprozesse liefern.
\end{rem}
\begin{rem}[zu Anf. 10]
Die Batterie soll genügend Leistung, Spannung und Energie für den Erprobungsbetrieb haben.
\end{rem}
\begin{feat}
Für die Batterie ist ein Batteriekasten vorzusehen. 
\end{feat}
\begin{rem} [zu Anf. 11]
Das Fach ist so anzubringen, dass es von außen zugänglich, aber auch vor Steinschlag und ähnlichem während einer Fahrt geschützt ist und eine möglichst einfache Wartbarkeit ermöglicht.
\end{rem}
\begin{rem} [zu Anf. 11]
Zusätzlich muss die Befestigung des Kastens sicher und stoßgeschützt sein.
\end{rem}

\subsubsection{Externe Versorgung}
\begin{feat}
Es sind folgende externe Versorgungsmöglichkeiten vorzusehen:
\begin{itemize}
    \item 100 - 240 V - Netzspannung mit Anpassungselektronik / Gleichsetzsteller
    \item Achsdeckelgenerator mit Anpassungselektronik
    \item Weitere
\end{itemize}
\end{feat}

\subsubsection{Anschlusskasten und Leitungen}
\begin{feat}
Es ist mindestens ein Anschluss- oder Klemmenkasten für die Spannungsversorgung von Sensoren und Aktoren vorzusehen. 
\end{feat}
\begin{feat}
An beiden Wagenenden ist ein Zugang zu den Daten- und Stromleitungen ist vorzusehen.
\end{feat}
\begin{feat}
Die Daten- und Stromleitungen sind geschützt in Stahlrohren zu verlegen.
\end{feat}

\subsection{Sensoren und Aktoren}
\begin{feat}
Zustand und Verstellbefehl für Sensoren und Aktoren sind auf unab- hängigen Kanälen zu übertragen.
\end{feat}
\subsubsection{Pneumatische Bremse}
\paragraph{G/P-Umstellung}
\begin{feat}
Eine elektrisch gesteuerte Einstellung der Bremsstellung jedes Wagens ist möglich.
\end{feat}
\begin{rem} [zu Anf. 17]
Dafür werden die Bremsstellungen G und P unterstützt.
\end{rem}
\begin{rem} [zu Anf. 17]
Die Bremsstellung ist sowohl für jeden einzelnen Wagen, als auch für Wagengruppen zugleich, einzustellen.
\end{rem}

\paragraph{Lösen der Bremse}
\begin{feat}
Das elektrisch gesteuerte Lösen des Lösezuges ist möglich.
\end{feat}
\begin{rem}[zu Anf. 18]
Das bedeutet, die A-Kammer ist elektrisch gesteuert zu entlüften.
\end{rem}
\begin{rem}[zu Anf. 18]
Die Funktion darf nur außerhalb des \gls{Zugverband}es bei entlüfteter Hauptluftleitung möglich sein.
\end{rem}

\paragraph{Lastwechsel}
\begin{feat}
Es ist ein automatischer \gls{Lastwechsel} möglich.
\end{feat}
\begin{rem}[zu Anf. 19]
Dieser darf nur im Stand möglich sein.
\end{rem}

\paragraph{Abschalten der Bremse}
\begin{feat}
Ein Abschalten der kompletten Bremse bleibt möglich
\end{feat}

\paragraph{Aktoren in der Hauptluftleitung}
\begin{feat}
Es sind Aktoren in der \acrshort{HL} vorzusehen, die entsprechende bistabile Endabsperrhähne auf elektrische Betätigung hin verschließen oder öffnen können.
\end{feat}

\subsubsection{Feststellbremse}
\begin{feat}
Es ist eine automatische Feststellbremse vorzusehen.
\end{feat}
\begin{feat}
Wechselwirkungen zwischen Feststellbremse und pneumatischer Bremse sind beim Anlegen und Lösen zu berücksichtigen.
\end{feat}
\begin{rem}[zu Anf. 23]
Die Feststellbremse darf nur im Stillstand und wenn die HL leer ist angesteuert werden.
\end{rem}
\begin{feat}
Es ist für eine Stillstandsüberwachung zu sorgen
\end{feat}
%\begin{rem}[zu Anf. 20]
%Die Stillstandsüberwachung wird für die Nutzung der Feststellbremse benötigt.
%\end{rem}

\paragraph{Außenanzeige}
\begin{feat}
Es ist für eine geeignete Anzeige der Endabsperrhähne, Bremsstellung, \gls{Lastwechsel} und der Feststellbremse zu sorgen.
\end{feat}
\begin{rem}[zu Anf. 25]
Diese ist zweikanalig auszuführen.
\begin{itemize}
    \item passiver Kanal: pneumatisch - Anzeige farblich
    \item aktiver Kanal: digital - Anzeige bspw. e-Paper
\end{itemize}
\end{rem}
\begin{feat}
Die digitale Anzeige muss auch im stromlosen Zustand eine Zustandsanzeige bieten
\end{feat}

\subsubsection{Condition Monitoring}
\begin{feat}
Für das Condition Monitoring werden folgende Sensoren benötigt:
\begin{itemize}
    \item Lagertemperatur
    \item Beschleunigung in x-, y- und z-Richtung
    \begin{itemize}
        \item Zur Überwachung von: Stößen, Lagerzuständen und Flachstellen
    \end{itemize}
    \item Laufleistung / Radumdrehung
    \item .
\end{itemize}
\end{feat}

%%%%%%%%%%%%%%%%%%%%%%%%%%%%%%%%
\begin{comment}
\subsubsection{Sonstige Sensoren}
\begin{feat}
Weitere Benötigte Sensoren:
\begin{itemize}
    \item 
\end{itemize}
\end{feat}
\end{comment}
%%%%%%%%%%%%%%%%%%%%%%%%%%%%%%%%%

 \subsection{Fahrzeugsteuerung und -kommunikation}
Die Wagen kommunizieren untereinander. Eine gebildete Wagengruppe ist nach außen wie ein Wagen zu behandeln. Jeder Wagen in dieser Wagengruppe kann Informationen über jeden Wagen der Wagengruppe herausgeben.\par
Jede Funktion des Wagens wird durch ein Befehl der Steuerung veranlasst. Es gibt verschiedene Möglichkeiten der Initialisierung von Befehlen.\par
Grundsätzlich gibt es vier Möglichkeiten der Datenübertragung:
\begin{itemize}
    \item Kommunikation innerhalb des Wagens
    \item Kommunikation zwischen den Wagen
    \item Kommunikation im Nahbereich
    \item Kommunikation im Fernbereich
\end{itemize}
Damit eine sichere Datenhaltung und -übertragung möglich ist, wird die in Abbildung \ref{fig:Rechnerarchitektur} gezeigte Rechnerarchitektur vorgeschlagen.\par
\begin{figure}[ht]
    \centering
    \begin{tikzpicture}[font = \sffamily, scale = 0.75]
\tikzstyle{every node}=[font=\small]
%%%%%%%%%%%%%%%%%%%%%%%%%%%%%%%%%%%%%%%%%%%%%%%
\begin{comment}
    %%%%%%%%%%%%%%%%%%%%%%%%%% Funkkommunikation
    %%%%%%%%%%%% Seite A
    \node (FunkA1a) at (-7.5, 6.8) {Funkkommunikation};
    \node (FunkA1b) at (-7.5, 6.3) {zum nächsten Wagen};
    \path[class4] (-8, 6) -- (-7, 6) -- (-7, 5) -- (-8,5) -- (-8 ,6);
    \node (FunkA2a) at (-7.5, -4.7) {Funkkommunikation};
    \node (FunkA2b) at (-7.5, -4.2) {zum nächsten Wagen};
    \path[class4] (-8, -6) -- (-7, -6) -- (-7, -5) -- (-8,-5) -- (-8 ,-6);
    %%%%%%%%%%%% Seite B
    \node (FunkB1a) at (7.5, 6.8) {Funkkommunikation};
    \node (FunkB1b) at (7.5, 6.3) {zum nächsten Wagen};
    \path[class4] (8, 6) -- (7, 6) -- (7, 5) -- (8,5) -- (8 ,6);
    \node (FunkB2a) at (7.5, -4.7) {Funkkommunikation};
    \node (FunkB2b) at (7.5, -4.2) {zum nächsten Wagen};
    \path[class4] (8, -6) -- (7, -6) -- (7, -5) -- (8,-5) -- (8 ,-6);
    
    %%%%%%%%%%%%%%%%%%%%%%%%%% Sensoren
    \node (SensA) at (-7.5, 3.6) {Sensoren};
    \path[class3] (-8, 3) -- (-7, 3) -- (-7, 2) -- (-8,2) -- (-8 ,3);
    \path[class3] (-8, 1.5) -- (-7, 1.5) -- (-7, 0.5) -- (-8,0.5) -- (-8 ,1.5);
    \node (SensB) at (7.5, 3.6) {Sensoren};
    \path[class3] (8, 3) -- (7, 3) -- (7, 2) -- (8,2) -- (8 ,3);
    \path[class3] (8, 1.5) -- (7, 1.5) -- (7, 0.5) -- (8,0.5) -- (8 ,1.5);
    
    %%%%%%%%%%%%%%%%%%%%%%%% Aktoren
    \node (AktB) at (7.5, -0.1) {Aktoren};
    \path[class5] (8, -1.5) -- (7, -1.5) -- (7, -0.5) -- (8,-0.5) -- (8 ,-1.5);
    \path[class5] (8, -3) -- (7, -3) -- (7, -2) -- (8,-2) -- (8 ,-3);
    \node (AktA) at (-7.5, -0.1) {Aktoren};
    \path[class5] (-8, -3) -- (-7, -3) -- (-7, -2) -- (-8,-2) -- (-8 ,-3);
    \path[class5] (-8, -1.5) -- (-7, -1.5) -- (-7, -0.5) -- (-8,-0.5) -- (-8 ,-1.5);
    \end{comment}
%%%%%%%%%%%%%%%%%%%%%%%%%%%%%%%%%%%%%%%%%%%%%%%

%%%%%%%%%%%%%%%%%%%%%%%%% Geschwungene Linie
%\draw[wagon, dash dot] plot [smooth] coordinates {(-6,8) (-5.2,6) (-6.5, 0) (-5.2, -4) (-6,-7)};
%\draw[wagon, dash dot] plot [smooth] coordinates {(6,8) (5.2,6) (6.5, 0) (5.2, -4) (6,-7)};
\draw[wagon, dash dot] (-5, 7) -- ( -5, -3.7) -- (5, -3.7) -- (5, 7) -- (-5, 7);

%%%%%%%%%%%%%%%%%%%%%%%%%% sicherheitskritischer Bereich
%Beschriftung
\node[gray] (Beschriftung) at (6.2, 3.5) {Schicht 1};
%Rechner 
\path[class1, fill=class1!50] (-1,5) rectangle +(2.5,1.5) node[pos = 0.5] (R1) {Rechner 1};
\path[class1, fill=class1!50] (-1,1) rectangle +(2.5,1.5) node[pos = 0.5] (R2) {Rechner 2};
\draw[<->, dashed, wagon] (0.25, 5) to (0.25,2.5);
\draw[->, class1] (R1) -| (-6.5, 4.7); %R1-Wagen
\draw[->, class1] (-1, 2) -| (-6.5, 2.8); %R2 - Wagen
\draw[<-, class1] (-0.5, 5) |- (-5.2, 3.9); %Wagen -R2
\draw[<-, class1] (-0.5, 2.5) |- (-5.2, 3.6); %Wagen-R!

%Zwischenspeicher
\path[class3, fill=class3!50] (-3.2, 1.15) ellipse (1.5 and 0.6);
\node[text width = 1.8cm] (B) at (-3., 1.15) {\small{Zwischen- speicher}};
\path[class1, thin] (-3.25, 1.7) |- (-1, 5.2) {}; %Zwischenspeicher - R1
\path[class1, thin] (-2., 1.5) -- (-1, 1.5) {}; %Zwischenspeicher - R2

%Wagen
\node[gray] (SAK1) at (-7, 3.5) {Sensoren};
\node[gray] (SAK2) at (-7, 3.0) {Aktoren};
\node[gray] (SAK3) at (-7, 4.0) {Kommunikation};
\node[gray] (SAK4) at (-7, 4.5) {Nachbarwagen};

%Speicher
\path[class1, fill=class1!50] (3, 1.15) ellipse (1.1 and 0.6);
\node (s) at (3, 1.15) {Speicher};
\draw[->, class1] (1.5, 5.75) -| (3.9, 1.5); %R1 - Speicher
\draw[->, class1] (1.5, 2.25) -| (2.7, 1.7); %R2 - Speicher
\draw[->, dashed, wagon] (2.3, 0.55) to (2.3, -1.5); %Speicher GSM
\draw[->, dashed, wagon] (2.3, 0.55) to (-2.25, -1.5); %Speicher- Nahbereich

%%%%%%%%%%%%%%%%%%%%%%%%%%%%%%%%% Linie
\path[wagon] (-5,0) -- (5,0) {};
\path[wagon] (-5,0.05) -- (5,0.05) {};
%%%%%%%%%%%%%%%%%%%%%%%%% nicht sicherheitskrischer Bereich
%Beschriftung
\node[gray] (Beschriftung) at (6.2, -1.8) {Schicht 2};
%GSM-Modul
\path[class4, fill=class4!50] (0.8, -3) rectangle +(3,1.5) node[pos = 0.5] (GSM) {GSM-Modul};
\draw[->, class4, dotted] (2., -3) to (2., -5);
\draw[<-, class4, dotted] (2.5, -3) to (2.5, -5);
\node[gray] (C) at (2.3, -5.3) {Cloud};


%Nahfunkbereich
\path[class4, fill=class4!50] (-4.3, -3) rectangle +(4,1.5) node[pos = 0.5] (NF) {Nahbereichsfunk};
\draw[->, class1] (-2.35, -1.5) to (-3,0.5);
\draw[->, class4, dotted] (-2.0, -3) to (-2.0, -5);
\draw[<-, class4, dotted] (-2.5, -3) to (-2.5, -5);
\node[gray] (B) at (-2.3, -5.3) {Bediener};

\end{tikzpicture}
    \caption{Vorgeschlagene Rechnerarchitektur für sichere und nicht sichere Funktionen und Daten des Güterwagen 4.0}
    \label{fig:Rechnerarchitektur}
\end{figure}
Im oberen Teil der Rechnerarchitektur ist der sicherheitskritische Bereich zu sehen. %Dieser ist (auf Dauer, wahrscheinlich) zulassungspflichtig. 
Dieser besteht aus zwei Rechnern, die für eine Zweikanaligkeit sorgen, einem Speicher,  für alle notwendigen Daten und einem Zwischenspeicher für die Bedienung.\par
Im unteren Teil ist der nicht sicherheitskritische Bereich. Dieser ist einkanalig ausgeführt. Dieser besteht aus dem Nahbereichsfunk und dem GSM-Modul als Fernbereichsfunk.\par
Angedeutet sind der Bediener, dessen Schnittstelle der Nahbereichsfunk darstellt, das GSM-Modul, die Schnittstelle zur Cloud und die Sensoren, Aktoren und Kommunikationseinheiten im Wagen.\par
Die beiden Rechner tauschen sich gegenseitig rückkopplungsfrei aus (gestrichelte graue Linie) und geben ihre Befehle (rote Pfeile nach außen) zweikanalig an Sensoren, Aktoren und die Kommunikationseinheiten im Wagen weiter. Die zurückkommenden Informationen (rote Pfeile zu Rechner 1 und Rechner 2) werden von den Rechnern verarbeitet und im Speicher (rote Pfeile zum Speicher) gesichert.\par
Befehle aus der Nahbereichsschnittstelle werden vom Bediener gegeben (gepunktete Linie zum Nahbereichsfunk) und im Zwischenspeicher (nur im Stillstand) an die Rechner weiter gegeben (dünne rote Linien) und dort verarbeitet. Wird der Befehl zur Unzeit oder während der Bewegung gegeben, verfällt er. Er wird nicht weiter zwischen gespeichert und später verarbeitet.\par
Alle notwendigen verarbeiteten Daten und Informationen werden im Speicher gespeichert (rote Pfeile zum Speicher) und dort entweder an den Nahbereichsfunk weitergegeben oder mit etwas Zeitverzug (und nur im Stillstand) über das GSM-Modul in der Cloud aktualisiert. Von dort aus sind sie für Cloud und Bediener einzusehen.\par

\subsubsection{Datenhaltung und -übertragung}
\begin{feat}
Alle Steuerbefehle müssen sicher und zuverlässig übertragen werden
\end{feat}
\begin{feat}
Die Reichweite soll für jeden Zweck ausreichend sein, aber von außen nicht stör- oder abhörbar.
\end{feat}
\begin{feat}
Die Bandbreite soll für jede Kommunikationsart ausreichend sein.
\end{feat}

\paragraph{Kommunikation innerhalb des Wagens}
\begin{feat}
Innerhalb des Wagenkastens werden alle Daten vorzugsweise kabelgebunden transportiert. 
\end{feat}
\begin{feat}
Die Kommunikation findet über einen \textbf{Zugbus o.ä.} statt.
\end{feat}
\begin{feat}
Die Informationen laufen alle im Bordrechner zusammen
\end{feat}

\paragraph{Kommunikation zwischen den Wagen}
\begin{feat}
Die Kommunikation zwischen den Wagen findet vorzugsweise über in der Nähe der Puffer statt.
\end{feat}
\begin{feat}
Für eine Übertragung zum nächsten Wagen sind kurze Funkstrecken oder Kabel (nach UIC 568) vorzusehen.
\end{feat}
\begin{rem}[zu Anf. 35]
Kurze Funkstrecken können über
\begin{itemize}
    \item WLAN (60GHz),
    \item NFC,
    \item Bluetooth,
    \item ...
\end{itemize}
realisiert werden.
\end{rem}

\paragraph{Kommunikation im Nahbereich}
\begin{feat}
Eine Kommunikation des Wagens / der Wagengruppe zum Bediener ist innerhalb vom Nahbereichsfunk möglich.
\end{feat}
\begin{feat}
Eine Kommunikation des Wagens / der Wagengruppe zum Bediener ist nur im Stillstand möglich.
\end{feat}

\paragraph{Kommunikation im Fernbereich}
\begin{feat}
Eine Kommunikation mit der Cloud ist zum Beispiel über den Mobilfunk möglich
\end{feat}
\begin{feat}
Die Kommunikation muss Rückstandsfrei sein. Es darf nur eine Leseberechtigung seitens der Cloud gewährt werden.
\end{feat}

\subsubsection{Autorisierung}
\begin{feat}
Es ist Sicherzustellen, dass Bediener berechntigt sind.
\end{feat}
\begin{rem} [zu Anf. 40]
Es sind verschiedene Autorisierungsstufen vorzusehen. Jede Gruppe braucht eine ausreichende Zugriffsmöglichkeit auf benötigte Daten.
Autorisierungsstufen können sein:
\begin{itemize}
    \item Stufe 1: Rangierpersonal in den Einfahrgruppen der Rangierbahnhöfe
    \item Stufe 2: Wagenmeister in Ausfahrgruppen
    \item Stufe 3: Notfallmanager auf freier Stecke
\end{itemize}
\end{rem}
\begin{rem} [zu Anf. 40]
Verschiedene Daten sollen nur in gewissen Geofencing-Bereichen abrufbar und veränderbar sein.
\end{rem}

\subsubsection{Bordrechner}
\begin{feat}
Es ist ein Bordrechner vorzusehen
\end{feat}
\begin{feat}
Der Bordrechner verfügt über genug Rechenleistung für die Verarbeitung sämtlicher auf dem Wagen vorhandener Daten sowie die Kommunikation mit weiteren Wagen, Anwendern und der ggf. Cloud.
\end{feat}
\begin{feat}
Der Bordrechner verfügt über genug Speicher für die Speicherung aller notwendigen Daten.
\end{feat}
\begin{feat}
Sichere und nicht sichere Funktionen und ihre Daten sind physikalisch getrennt zu verarbeiten und zu speichern.
\end{feat}

\subsubsection{Pneumatische Bremse}
\paragraph{Bremsstellungen}
\begin{feat}
Folgende Regelungen gelten für die Bremsstellung:
\begin{itemize}
    \item Der Wagen soll nach Aufforderung seine Bremsstellung in den gewünschten Modus umstellen können.
    \item Der Wagen soll abgeschaltete Bremsen selbstständig melden.
\end{itemize}
\end{feat}

\paragraph{Lastwechsel}
\begin{feat}
Folgende Regel gilt für die \gls{Lastwechsel}einstellung:
\begin{itemize}
    \item Der Wagen soll nach Aufforderung den \gls{Lastwechsel} selbst vornehmen können, oder
    \item über eine automatische \gls{Lastwechsel}einstellung verfügen
\end{itemize}
\end{feat}

\paragraph{Aktorsteuerung in der Hauptluftleitung}
\begin{feat}
Hier ist folgende Regelung für die Aktorsteuerung der Endabsperrhähne in der \acrshort{HL} vorzusehen:
\begin{itemize}
    \item Ist der Güterwagen luftgekuppelt, so ist das Ventil in der \acrshort{HL} im Normalzustand an der Seite offen.
    \item Ist der Güterwagen nicht luftgekuppelt, so ist das Ventil in der \acrshort{HL} im Normalzustand auf der Seite geschlossen.
\end{itemize}
\end{feat}
\begin{rem} [zu Anf. 47]
Weitere manuelle Steuerungen, beispielsweise für Ablaufberge oder manuelle Bremsproben sind vorzusehen.
\end{rem}

\subsubsection{Feststellbremse}
\begin{feat}
Folgende Regelung gilt für die Feststellbremse:
\begin{itemize}
    \item Die Feststellbremse ist gelöst, wenn der Wagen luftgekuppelt ist.
    \item Die Feststellbremse ist angezogen, wenn der Wagen nicht luftgekuppelt ist.
\end{itemize}
\end{feat}
\begin{rem} [zu Anf. 48]
Weitere manuelle Steuerungen sind mittels Endgerät oder \newline Schalter am Wagen schnell und sicher zu ermöglichen.
\end{rem}

\subsubsection{Bremsprobe und Bremsberechnung}
\paragraph{Bremsprobe}
\begin{feat}
Folgende Regelungen gelten für die Bremsprobe:
\begin{itemize}
    \item Der Wagen soll selbstständig seine Bremsfähigkeit (in t und in \%) angeben
    \item Der Wagen soll selbstständig seine Bremsfähigkeit (in t und in \%) erproben
    %\item Der Wagen soll selbstständig eine Rückmeldung an ein passendes Gerät geben
    \item Der Wagen soll den Druckverlauf im Reservebehälter angeben können
    \item Der Wagen soll den Druckverlauf in der \acrshort{HL} angeben
    \item Der Wagen soll den C-Druckverlauf angeben
    \item DEr wagen soll den A-Druckverlauf angeben
\end{itemize}
\end{feat}
\begin{feat}
Eine vom Rechner gesteuerte automatische Bremsprobe ist möglich.
\end{feat}
\begin{rem} [zu Anf. 50] 
Der Rechner gibt die Bremshunderstel für die gegebene Konfiguration des \gls{Zugverband}es an.
%%%%%%%%%%%%%%%%%%%%%%
\begin{comment}
Diese hat sich für folgende Fahrten zu unterscheiden:
\begin{itemize}
    \item für Bedienfahrt
    \item für Rangierfahrt
    \item für Sperrfahrt
    \item für Zugfahrt
\end{itemize}
\end{comment}
%%%%%%%%%%%%%%%%%%%%
\end{rem}

\paragraph{Technische Wagenbehandlung}
%%%%%%%%%%%%%%%%%%%%%%%%
\begin{comment}
\begin{feat}
So weit eine technische Wagenbehandlung aktor- und sensorgeführt möglich ist, soll diese integriert werden.
\end{feat}
\begin{feat}
Folgende Regelungen gelten für die technische Wagenbehandlung:
\begin{itemize}
    \item Hier muss was hin
\end{itemize}
\end{feat}
\end{comment}
%%%%%%%%%%%%%%%%%%%%%%%%%%

\paragraph{Bremsberechnung}
%\begin{feat}Eine automatische Bremsberechnung anhand des Wagenzuges und der Lok ist möglich. \end{feat}
%\begin{rem} [zu Anf. 30]
%Die Regelungen zur Bremsberechung sind:
%\begin{itemize}
 %   \item ...
%\end{itemize}
%\end{rem}

%%%%%%%%%%%%%%%%%%%%%%%%%%%%
%Hier muss was hin.

\subsubsection{Wagentrennung und -zusammenstellung}
\begin{feat}
Eine elektrische Vorbereitung von Trennstellen ist möglich.
\end{feat}
\begin{rem} [zu Anf. 51]
Folgende Regelungen gelten für die Trennstellenanzeige:
\begin{itemize}
    \item Ist auf dem Endgerät eine Trennstelle angewählt, so müssen die \acrshort{HL}-Absperrventile sich an diesen Punkte schließen um eine Luftkupplungstrennung zu ermöglichen.
    \item Die Trennstellenanzeige muss deutlich von allen Seiten des Güterwagens sichtbar sein.
    \item Die Trennstellenanzeige muss farblich erkennbar sein.
\end{itemize}
\end{rem}
\begin{feat}
Eine automatische Wagengruppenbildung ist möglich.
\end{feat}
\begin{feat}
Die elektrische Vorbereitung von Kuppelstellen ist möglich.
\end{feat}
Die Vorprüfung der Wagen im automatisierten Rangierbahnhof sollen bei einem Wagenzug, der nur aus ausgerüsteten Güterwagen 4.0 besteht, nicht mehr notwendig sein, da dort die Reihenfolge der Wagen bekannt ist. Solange nicht alle Wagen Güterwagen 4.0 sind, muss dieser Prozess weiter ausgeführt werden.
\begin{feat}
Eine Überprüfung der Wagenreihung ist möglich.
\end{feat}
\begin{feat}
Eine Speicherung des technischen Zustands aus vorheriger Fahrt ist \newline wünschenswert.
\end{feat}
\begin{feat}
Eine automatische Übermittlung von Transportunterlagen ist\newline möglich.
\end{feat}
\begin{feat}
Auf dem Wagen soll ein elektronischer Frachtbrief digital vorhanden und abrufbar sein.
\end{feat}

\subsection{Diagnosefunktion}
Noch was zu Prognose und Bezug auf KI - wissensbasierte Systeme, neuronale Netze, Verknüpfung von Systemen
\begin{feat}
Alle für eine vorausschauende Wartung notwendigen Sensoren sollen eingebaut und überwacht werden.
\end{feat}
%%%%%%%%%%%%%%%%%%%%%%
\begin{comment}
\begin{rem} [zu Anf. 56]
Beispiele dafür sind:
\begin{itemize}
    \item Lagertemperaturüberwachung
    \item Stoßüberwachung
    \item Laufleistung
    \item Drehkugelpfanne am Drehgestell
    \item Bremsbelagüberwachung
    \item Lagerzustände
    \item Flachstellen
    \item Entgleisungssicherheit aufgrund von Beladung
\end{itemize}
\end{rem}
\end{comment}
%%%%%%%%%%%%%%%%%%%%%%%%%%%%%%%%
\begin{feat}
Diese Informationen soll der Wagen über sich speichern und abrufbar bereithalten.
\end{feat}
\begin{feat}
Alle bereits über den Wagen vorhandenen Informationen, wie Wartungszyklen, Instandhaltungen und ähnliches sollen als Dokument lokal auf dem Wagen  verfügbar sein.
\end{feat}

\subsection{Sonstiges}
\paragraph{Migrationsstrategie}
\begin{feat}
Eine Migrationsstrategie ist zu planen.
\end{feat}
\paragraph{Beladung und Ladungssicherung}
Hier sind für den \gls{Demonstrator} keine Änderungen geplant.
\paragraph{Mechanische und Luftkupplung}
Beim \gls{Demonstrator} ist keine Automatisierung der mechanischen oder der Luftkupplung geplant.
