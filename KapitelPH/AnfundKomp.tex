\section{Anforderungen und Komponentenauswahl}


\subsection{Anforderungen}
Funktional, nicht funktional, sonstige
\begin{feat}[Anf. X]
test
\end{feat}

\subsection{Anforderungen an elektrische Komponenten}

\subsubsection{Allgemeine Anforderungen}
\paragraph{Spannungsniveau}
\paragraph{Kabelquerschnitte}

\subsubsection{Anforderungen an die Komponenten}
\paragraph{Radsatzgenerator}
\paragraph{Externe Ladeelektronik}
\paragraph{Batterie}
\paragraph{Bordelektronik}
\paragraph{Zustandsanzeige pneumatische Kupplung}
\begin{feat}
Der Güterwagen 4.0 verfügt über eine Anzeige je Fahrzeugende an den Pufferträgern.
\end{feat}
\begin{feat}
Der Güterwagen 4.0  verfügt über eine Zustandsanzeige für die pneumatische Kupplung mit folgenden Funktionen:
\begin{itemize}
    \item Die Anzeige stellt den Zustand des Hahns (drucklos/druckbeaufschlagt) der Bremskupplung dar.
    \item Die Anzeige muss Überdrücke > 0.5 bar in den Bremskupplungen anzeigen. Beispielsweise durch die Farbe 'rot' im Schauglas.
    \item Bei Unterschreiten des Drucks wird ebenfalls angezeigt. Beispielsweise durch die Farbe 'grün' im Schauglas.
    \item Die Anzeige muss auch im Stromlosen Zustand verfügbar sein.
\end{itemize}
\end{feat}

\subsection{Anforderungen an pneumatische Komponenten}
\paragraph{Steuerventil}
\begin{feat}
Es wird ein UIC/TSI-konformes Steuerventil eingesetzt oder beibehalten.
\begin{itemize}
    \item Das Steuerventil verfügt über die Bremsstellungen G und P.
    \item Das Steuerventil verfügt über automatisches Schnelllösen. 
    \item Das Relaisventil ist vorzugsweise nicht integriert.
\end{itemize}
\end{feat}

\paragraph{Relaisventil}
\begin{feat}
Das Relaisventil ist für automatische Lastabbremsung mit Wiegeventil vorgesehen
\end{feat}

\paragraph{Endabsperrhahn}
\begin{feat}
Es werden zwei (auf jeder Seite eines) Endabsperrhähne mit folgenden Eigenschaften eingesetzt:
\begin{itemize}
    \item Die Endabsperrhähne verfügen über einen freien Querschnitt von 1,25''.
    \item Die Endabsperrhähne sind bistabil, d.h. verbleiben ohne Betätigung in ihrem letzten Zustand.
    \item Die Betätigungszeit für den Übergang ''Öffnen - Schließen'' beträgt maximal 60 s.
    \item Das Funktionsprinzip der Hähne ist geeignet, die Anforderungen der DIN EN 14601 erfüllen zu können.
    \begin{itemize}
        \item Für die Demonstratoren und Labormuster muss diese Norm nicht erfüllt werden.
    \end{itemize}
    \item Eine fahrzeugseitige Verschraubung nach G1 1/4i (DIN EN ISO 228-1) ist zu bevorzugen.
    \begin{itemize}
        \item Für die Demonstratoren kann die Kompatibilität durch einen Adapter hergestellt werden.
    \end{itemize}
    \item Kuppelseitig ist eine Verschraubung mit Whitworth-Gewinde mit stumpfen Gewinden für G1 1/4i-Leitungen zu bevorzugen.
    \begin{itemize}
        \item Für die Demonstratoren kann die Kompatibilität durch einen Adapter hergestellt werden.
    \end{itemize}
\end{itemize}
\end{feat}

\paragraph{Feststellbremse}
\begin{feat}
Der Wagen verfügt über eine Feststellbremse mit folgenden Funktionen:
\begin{itemize}
    \item Die Feststellbremse kann unabhängig von der pneumatischen Energie im Wagen angelegt und gelöst werden.
    \item Das Anlegen und Lösen erfolgt bistabil über einen elektrischen Impuls
    \item Eine Rückmeldefunktion ist für den gelösten Zustand vorzusehen.
    \item Die Bremskraft am Bremszylinder beträgt ?? (für 2\%-Gefälle, 90 t, Klotzbremse)
\end{itemize}
Alternative Lösungen, wie Federspeicher oder FT-Park Lock können vorgeschlagen werden.
\end{feat}

\paragraph{ep-Bremse}
\begin{feat}
Der Wagen verfügt über eine ep-Bremse mit folgenden Funktionen:
\begin{itemize}
    \item Das ep-Bremsventil wird zum Bremsen bestromt.
    \item Das Ventil entlüftet die Hauptluftleitung im Wagen in (3,5 ... 5) s von Regelbetriebsdruck auf 3,5 bar.
\end{itemize}
\end{feat}


\subsection{Anforderungen an sensorische Kompoenten}

\paragraph{Batteriestand}
Der Ladungsstand der Batterie kann eingesehen werden
\paragraph{Pneumatische Kupplung}
Der Zustand der Pneumatischen Kupplung (offen/geschlossen) kann eingesehen werden
\paragraph{Steuerventilstellung}
Die Stellung des Steuerventils wird detektiert
\paragraph{Relaisventilstellung}
Die Stellung des Relaisventils wird detektiert
\paragraph{Stellung der HL-Ventile}
Die Stellung der Ventile in der HL wird detektiert
\paragraph{Stellung der Feststellbremse}
Die Stellung der Feststellbremse wird detektiert
\paragraph{C-Druck}
\label{sec:CDruck}
\begin{feat}
Der C-Druck (Bremszylinderdruck) wird gemessen.
\end{feat}
\begin{feat}
Der Sensor hat folgende Eigenschaften:
\begin{itemize}
    \item Der Sensor arbeitet als Stromsensor (4...20 mA)
    \item Der Messbereich beträgt (0...5) bar
    \item Die Messunsicherheit und Auflösung ist so gewählt, dass die erste und letzte Bremsstufe (0,45 bar Cv) sicher erkannt werden. Eine Messunsicherheit von 0,05 bar erfüllt diese Anforderung.
\end{itemize}
\end{feat}





\subsection{Anforderungen an Daten- und Datenkommunikationskompoenten}

\subsection{Komponentenauswahl}
Welche Hard- und Softwarekomponenten sollen benutzt werden?