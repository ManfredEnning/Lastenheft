\subsection{Realisierung Datenkommunikation}\label{sec:dKomp}
In diesem Kapitel geht es um die Ausführung der Datenkommunikation. Es teilt sich auf in die Teile Hardware und Software und ist zugehörig zum Arbeitspaket 5 - Datenkommunikation; genauer AP 5a: Entwicklung eines Hardware- und Software-Konzepts für die Kommunikation und behandelt die Ausführung der notwendigen Daten und deren Kommunikationsmöglichkeiten in Hard- und Software.  Auch die Rechnerarchitektur wird hier beschrieben.\par
Die entsprechenden Hardwarekomponenten sind in Abbildung \ref{fig:dKomp} in violett farblich markiert.
\begin{figure}[hbt]
    \centering
    \pgfdeclarelayer{background}
\pgfdeclarelayer{foreground}
\pgfdeclarelayer{Beschriftung}
%Farbendefinition
\definecolor{elek}{RGB}{105,105,105}%grau{55,126,184}%blau
\definecolor{pneu}{RGB}{105,105,105}%grau{77,175,74}%grün
\definecolor{sens}{RGB}{105,105,105}%grau{228,26,28}%rot
\definecolor{dat}{RGB}{152,78,163}%violett
\definecolor{strom}{RGB}{105,105,105}%grau{0,0,0}%schwarz

\tikzset{wagon/.style={draw = gray, ultra thick, opacity = 0.7}}
\tikzset{seite/.style={opacity = 1}}
\tikzset{elek/.style= {draw = elek, ultra thick, opacity = 1}} %elektrische Komponenten
\tikzset{sens/.style={draw = sens, ultra thick, opacity = 1}} %sensorische Komponenten
\tikzset{pneu/.style={draw = pneu, ultra thick, opacity = 1}} %pneumatische Komponenten
\tikzset{dat/.style={draw = dat, ultra thick, opacity = 1}} %Datenkomponenten
\tikzset{strom/.style={draw = strom, ultra thick, opacity = 1}} %Strom- und Datenleitungen
\tikzset{annotation/.style={draw = black, thick, opacity = 0.7, font=\scriptsize}}

\pgfsetlayers{background,main,Beschriftung,foreground}
\begin{tikzpicture}[scale=0.7]
%%%%%%%%%%%%%%%%%%%%%%%%%%%%%%%%%%%%%%%%%%%%%%%% Hintergrund %%%%%%%%%%%%%%%%%%%%%%%%%%%%%%%%%%%%%%%%%%%%%
    \begin{pgfonlayer}{background}
    %Seiten
        %Seite A
        \path[seite] (-6, 3) rectangle +(.6,.2) node[pos = 0.5] (seiteA) {Seite A};
        %Seite B
        \path[seite] (5.3, 3) rectangle +(.6,.2) node[pos = 0.5] (seiteB) {Seite B};
    %wagon als Basis
    \path[wagon] (-5,-2) -- (-5,2) -- (5,2) -- (5,-2) -- cycle;
    % HL
    \path[wagon, color=pneu] (-5,-.5) -- (5,-.5) node[pos = 0.6, above] {\color=\gray \tiny{HLL}};
    % Buffer
    \begin{scope}[shift = {(-5,1.5)}]
    	\path[wagon] (-.8,.3) -- (0,.3) -- (0,-.3) -- (-.8,-.3);
    	\path[wagon] (-1,.25) -- (-.8,.25) -- (-.8,-.25) -- (-1,-.25);
    	\path[wagon] (-1,-.5) .. controls (-1.05,0) and (-1.05,0) .. (-1,.5);
    \end{scope}
    \begin{scope}[shift = {(-5,-1.5)}]
    	\path[wagon] (-.8,.3) -- (0,.3) -- (0,-.3) -- (-.8,-.3);
    	\path[wagon] (-1,.25) -- (-.8,.25) -- (-.8,-.25) -- (-1,-.25);
    	\path[wagon] (-1,-.5) .. controls (-1.05,0) and (-1.05,0) .. (-1,.5);
    \end{scope}
    \begin{scope}[shift = {(5,-1.5)}, rotate = 180]
    	\path[wagon] (-.8,.3) -- (0,.3) -- (0,-.3) -- (-.8,-.3);
    	\path[wagon] (-1,.25) -- (-.8,.25) -- (-.8,-.25) -- (-1,-.25);
    	\path[wagon] (-1,-.5) .. controls (-1.05,0) and (-1.05,0) .. (-1,.5);
    \end{scope}
    \begin{scope}[shift = {(5,1.5)}, rotate = 180]
    	\path[wagon] (-.8,.3) -- (0,.3) -- (0,-.3) -- (-.8,-.3);
    	\path[wagon] (-1,.25) -- (-.8,.25) -- (-.8,-.25) -- (-1,-.25);
    	\path[wagon] (-1,-.5) .. controls (-1.05,0) and (-1.05,0) .. (-1,.5);
    \end{scope}
    %Wheelset
    \begin{scope}[shift = {(-4,0)}]
    	\path[wagon] (-.1,1.7) -- (.1,1.7) -- (.1,-1.7) -- (-.1, -1.7) -- cycle; 
    	\path[wagon] (-.6,1.4) -- (.6,1.4) -- (.55,1.5) -- (-.55, 1.5) -- cycle; 
    	\path[wagon] (-.6,-1.4) -- (.6,-1.4) -- (.55,-1.5) -- (-.55, -1.5) -- cycle; 
    \end{scope}
        \begin{scope}[shift = {(-2.5,0)}]
    	\path[wagon] (-.1,1.7) -- (.1,1.7) -- (.1,-1.7) -- (-.1, -1.7) -- cycle; 
    	\path[wagon] (-.6,1.4) -- (.6,1.4) -- (.55,1.5) -- (-.55, 1.5) -- cycle; 
    	\path[wagon] (-.6,-1.4) -- (.6,-1.4) -- (.55,-1.5) -- (-.55, -1.5) -- cycle; 
    \end{scope}
    \begin{scope}[shift = {(4,0)}]
    	\path[wagon] (-.1,1.7) -- (.1,1.7) -- (.1,-1.7) -- (-.1, -1.7) -- cycle; 
    	\path[wagon] (-.6,1.4) -- (.6,1.4) -- (.55,1.5) -- (-.55, 1.5) -- cycle; 
    	\path[wagon] (-.6,-1.4) -- (.6,-1.4) -- (.55,-1.5) -- (-.55, -1.5) -- cycle; 
    \end{scope}
    \begin{scope}[shift = {(2.5,0)}]
    	\path[wagon] (-.1,1.7) -- (.1,1.7) -- (.1,-1.7) -- (-.1, -1.7) -- cycle; 
    	\path[wagon] (-.6,1.4) -- (.6,1.4) -- (.55,1.5) -- (-.55, 1.5) -- cycle; 
    	\path[wagon] (-.6,-1.4) -- (.6,-1.4) -- (.55,-1.5) -- (-.55, -1.5) -- cycle; 
    \end{scope}
    \end{pgfonlayer}
%%%%%%%%%%%%%%%%%%%%%%%%%%%%%%%%%%%%%%%%%%%%%%% Vordergrung %%%%%%%%%%%%%%%%%%%%%%%%%%%%%%%%%%%%%%%%%%%%%%%%%
    \begin{pgfonlayer}{foreground} %Komponete
    %elektronische Komponenten
        %Radsatzgenerator
        \path[elek, fill = elek, thin] (-4.3, -1.9) rectangle +(.6,.2) node[pos = 0.5] (wsg) {};
        %Rechner
        \path[elek, fill = elek, thin] (-.5, -1.6) rectangle +(1,.3) node[pos = 0.5] (bcu) {};
        %Batterie
        \path[elek, fill = elek, thin] (-1.8, -1.8) rectangle +(1,.5) node[pos = 0.5] (bat) {};
        %optionale Ladeelektronik
        \path[elek, fill = elek, thin] (-1.8, 1.2) rectangle +(0.8,.4) node[pos = 0.5] (ole) {};
    % pneumatische Komponenten
        %epBremse
        \path[pneu, fill = pneu] (.9,-1.3) rectangle (1.1,-1.5) node[pos = 0.5] (epb) {};
        %Endabsperrhähne
        \path[pneu, fill = pneu] (-5.2, -.4) rectangle (-5,-.6) node[pos = 0.5] (eca) {};
        \path[pneu, fill = pneu] (5.2, -.4) rectangle (5,-.6) node[pos = 0.5] (ecb) {};
        %Aktorik Bremse
        \path[pneu, fill = pneu, thin] (-.5, 0) rectangle +(1,.5) node[pos = 0.5] (bcu2) {};
    %sensorische Komponenten
        %drahtloserSensor
        \path[sens, fill = sens, thin] (3.9, -1.7) rectangle +(.2,-.2) node[pos = 0.5] (wss) {};
        %Flachstellendektektor
        \path[sens, fill=sens, thin] (3.15, 1.8) rectangle +(.2,-.2) node[pos = 0.5] (flachstelle) {};
        \path[sens, fill=sens, thin] (-3.35, 1.8) rectangle +(.2,-.2) node[pos = 0.5] (flachstelle2) {};
        %Laufleistung
        \path[sens, fill = sens, thin] (3.9, 0) rectangle +(.2,-.2) node[pos = 0.5] (ll) {};
    %Datenkomponeten
        %Kurzstreckenfunk
        \path[dat, fill = dat] (-5.2, -.9) rectangle (-5,-1.05)node[pos = 0.5] (sra) {};
        \path[dat, fill = dat] (5.2, -.9) rectangle (5,-1.05)node[pos = 0.5] (srb) {};
        \path[dat, fill = dat] (5.2, .9) rectangle (5,1.05)node[pos = 0.5] (src) {};
        \path[dat, fill = dat] (-5.2, .9) rectangle (-5,1.05) node[pos = 0.5] (srd) {};
    \end{pgfonlayer}
%%%%%%%%%%%%%%%%%%%%%%%%%%%%%%%%%%%%%%%%%%%%%%% Beschriftung %%%%%%%%%%%%%%%%%%%%%%%%%%%%%%%%%%%%%%%%%%%%%%%%%
    \begin{pgfonlayer}{Beschriftung}
    %elektronische Komponenten
        %Radsatzgenerator
        \path[annotation, thin] (wsg) -- +(-.5,-.5) node[left] {Achsdeckelgenerator};
        %Rechner
        \path[annotation, thin] (bcu) -- +(1,-1) node[right] {Bordelektronik};
        %Batterie
        \path[annotation, thin] (bat) -- +(-1,-1) node[left] {Batterie};
        %optionale Ladeelektronik
        \path[annotation, thin] (ole) -- +(-.9,.9) node[left] {externe Ladeschnittstelle};
    %pneumatische Komponenten
        %epBremse
        \path[annotation, thin] (epb) -- +(.4,-.4) node[right] {ep-Bremse};
        %Endabsperrhähne
        \path[annotation, thin] (eca) -- +(-.5,.5) node[left] {Aktor Endabsperrhahn};
        %Aktorik Bremse
        \path[annotation, thin] (bcu2) -- +(.5,.5) node[right] {Aktorik Bremse};
    %sensorische Komponenten
        %drahtloserSensor
        \path[annotation, thin] (wss) -- +(.5,-.5) node[right] {Laufleistung}; 
        %Flachstellen
        \path[annotation, thin] (flachstelle) -- +(.7,.7) node[right] {Lagertemperatur};
        \path[annotation, thin] (flachstelle2) -- +(7.2,.7) node[right] {};
        %Laufleistung
        \path[annotation, thin] (ll) -- +(1.2,.5) node[right] {Beschleunigungen};
    %Datenkomponeten
        %Kurzstreckenfunk
        \path[annotation, thin] (srd) -- +(-.5,-.5) node[left] {Kurzstreckenfunk};    
    \end{pgfonlayer}
%%%%%%%%%%%%%%%%%%%%%%%%%%%%%%%%%%%%%%%%%%%%%% Main %%%%%%%%%%%%%%%%%%%%%%%%%%%%%%%%%%%%%%%%%%%%%%%%%%%%
    \begin{pgfonlayer}{main} %Leitungen
    %elek
        %Radsatzgenerator
        \path[elek] (wsg) +(-.1,0) -| (-3.3, -1);%-- +(1,0) -- (-3.2,-1);
        \path[strom, dashed] (wsg) +(-.1,0) -| (-3.3, -1);%-- +(1,0) -- (-3.2,-1);
        %Rechner
        \path[elek] (bcu) +(0,0)  -- (0,-1);
        \path[strom, dashed] (bcu) +(0,0)  -- (0,-1);
        %Batterie
        \path[elek] (bat) +(0,0)  -- (-1.3,-1);
        \path[strom, dashed] (bat) +(0,0)  -- (-1.3,-1);
        %optionale Ladeelektronik
        \path[elek] (ole) +(0,0)  -- (-1.35,-1);
        \path[strom, dashed] (ole) +(0,0)  -- (-1.35,-1);
    %pneu
        %ep-Bremse
        \path[pneu] (1,-.5) -- (1,-1.3);
        %\path[strom, dashed] (1,-.5) -- (1,-1.3);
        \path[pneu] (epb) +(.1,0) -- +(-.5,0);
        \path[strom, dashed] (epb) +(.1,0) -- +(-.5,0);
        %Endabsperrhähne
        \path[pneu] (eca) +(-.1,0) -- +(.2,0) -- (-4.9,-1);
        \path[strom, dashed] (eca) +(-.1,0) -- +(.2,0) -- (-4.9,-1);
        \path[pneu] (ecb) +(.1,0) -- +(-.2,0) -- (4.9,-1);
        \path[strom, dashed] (ecb) +(.1,0) -- +(-.2,0) -- (4.9,-1);
        %Aktorik Bremse
        \path[pneu] (bcu2) +(0,0)  -- (0,-1);
        \path[strom, dashed] (bcu2) +(0,0)  -- (0,-1);
    %Sensorische Kompoenten
        %Flachstellen
        \path[sens] (flachstelle)+(0,0) -- (3.2, -1);
        \path[strom, dashed] (flachstelle)+(0,0) -- (3.2, -1);
        \path[sens] (flachstelle2) +(0,0)-- (-3.3, -1);
        \path[strom, dashed] (flachstelle2) +(0,0)-- (-3.3, -1);
        %Laufleistung
        \path[sens] (ll)+(0,0) -- (4, -1);
        \path[strom, dashed] (ll)+(0,0) -- (4, -1);
        %drahloser Sensor
        \path[strom, dotted] (wss)+(0,0) -- (4, -1);
    %Strom- und Datenleitung
        \path[dat] (-5,-1) -- (5,-1);
        \path[strom, dashed] (-5,-1) -- (5,-1);
        %Kurzstreckenfunk
        \path[dat] (srd) +(-.1,0) -- +(.3,0) -- (-4.8,-1);
        \path[strom, dashed] (srd) +(-.1,0) -- +(.3,0) -- (-4.8,-1);
        \path[dat] (src) +(.1,0) -- +(-.3,0) -- (4.8,-1);
        \path[strom, dashed] (src) +(.1,0) -- +(-.3,0) -- (4.8,-1);
    \end{pgfonlayer}

\end{tikzpicture}

    \caption{Hardwarekomponenten zur Datenkommunikation des Gesamtsystems}
    \label{fig:dKomp}
\end{figure}
Damit die Wagen untereinander sozial interagieren können, ist eine Kommunikation untereinander ebenso notwendig wie Informationen über sich selbst. Zur Verarbeitung und Sicherung eigener Daten werden Sensoren und Aktoren ausgelsen und im Digitalen Zwilling gespeichert.\par
Dieser Digitale Zwlling wird bei der Kommunikation mit anderen WAgen, der Lok oder mobilen Device je nach Autorisierung mit Lese- oder Lese- und Schreibzugriff ausgetauscht.\par
\begin{figure}[hbt]
    \centering
    \includesvg[width=\textwidth]{Bilder/wagen_draufsicht_2-achs_comm}
    \caption{Kommunikationsmöglichkeiten eines einzelnen Güterwagen 4.0\cite{autonBetrieb}}
    \label{fig:Wagenkomm}
\end{figure}
Die Kommunikation findet entweder über ein WLAN-Mesh-Grid (siehe Abbildung \ref{fig:Wagenkomm}, blau) in mittlerer Distanz oder direkt über eine Kurzdistanz-Verbindung (lila) statt. Bei beiden ist wichtig, dass die Kommunikation sicher und unempfindlich gegenüber Störungen und Manipulation ist. Diese soll, wie in Abbildung \ref{fig:dKomp} zu sehen, Redundand ausgeführt werden. Dies soll für eine höhere Sicherheit und Zuverlässigkeit führen. \par
Zusätzlich soll auch noch eine Verbindungzu einer Cloud im Fernbereichsfunk zur Verfügung stehen. Diese erhält allerdings nur einen Lesezugriff um eine Manipulation über die Cloud so schwierig wie Möglich zu gestalten.\par
\begin{figure}[hbt]
    \centering
    \includesvg[width=\textwidth]{Bilder/zug_draufsicht_bogen}
    \caption{Kommunikation im Zugverband\cite{autonBetrieb}}
    \label{fig:Zugkomm}
\end{figure}
Der Funk über die mittlere Distanz soll vorallem für eine Migration zur Verfügung stehen. Siehe dazu auch Abbildung \ref{fig:Zugkomm}.\par
Bei einer Vollausrüstung von Wagen mit Kommunikation von Wagen zu Wagen über Funk und innerhalb der Wagen über Ethernet ist diese Kommunikation kaum störbar. Hier bietet der Güterwagen auch volles Potential für Zugautomatisierungen inklusive Zugtaufe, Bremsprobe und ep-Bremsen. Sogar ETCS Level 3 kann möglich sein.\par
Aber auch bei nur einer Teilausrüstung der Wagen kann eine Nutzentfaltung durch Digitalisierung von Prozessen an der Ladestelle stattfinden. Eine Automatisierung ist dann in Verbindung mit stationärer Technik im Betrieb möglich.\par
Die Vernetzung zur Lok kann mittels eines Dongles an der UIC 556-Schnittstelle über das Wire-Train-Bus-Gateway stattfinden.\par

Damit eine sichere Datenhaltung und -übertragung möglich ist, wird die in Abbildung \ref{fig:Rechnerarchitektur} gezeigte Rechnerarchitektur vorgeschlagen.\par
\begin{figure}[ht]
    \centering
    \begin{tikzpicture}[font = \sffamily, scale = 0.75]
\tikzstyle{every node}=[font=\small]
%%%%%%%%%%%%%%%%%%%%%%%%%%%%%%%%%%%%%%%%%%%%%%%
\begin{comment}
    %%%%%%%%%%%%%%%%%%%%%%%%%% Funkkommunikation
    %%%%%%%%%%%% Seite A
    \node (FunkA1a) at (-7.5, 6.8) {Funkkommunikation};
    \node (FunkA1b) at (-7.5, 6.3) {zum nächsten Wagen};
    \path[class4] (-8, 6) -- (-7, 6) -- (-7, 5) -- (-8,5) -- (-8 ,6);
    \node (FunkA2a) at (-7.5, -4.7) {Funkkommunikation};
    \node (FunkA2b) at (-7.5, -4.2) {zum nächsten Wagen};
    \path[class4] (-8, -6) -- (-7, -6) -- (-7, -5) -- (-8,-5) -- (-8 ,-6);
    %%%%%%%%%%%% Seite B
    \node (FunkB1a) at (7.5, 6.8) {Funkkommunikation};
    \node (FunkB1b) at (7.5, 6.3) {zum nächsten Wagen};
    \path[class4] (8, 6) -- (7, 6) -- (7, 5) -- (8,5) -- (8 ,6);
    \node (FunkB2a) at (7.5, -4.7) {Funkkommunikation};
    \node (FunkB2b) at (7.5, -4.2) {zum nächsten Wagen};
    \path[class4] (8, -6) -- (7, -6) -- (7, -5) -- (8,-5) -- (8 ,-6);
    
    %%%%%%%%%%%%%%%%%%%%%%%%%% Sensoren
    \node (SensA) at (-7.5, 3.6) {Sensoren};
    \path[class3] (-8, 3) -- (-7, 3) -- (-7, 2) -- (-8,2) -- (-8 ,3);
    \path[class3] (-8, 1.5) -- (-7, 1.5) -- (-7, 0.5) -- (-8,0.5) -- (-8 ,1.5);
    \node (SensB) at (7.5, 3.6) {Sensoren};
    \path[class3] (8, 3) -- (7, 3) -- (7, 2) -- (8,2) -- (8 ,3);
    \path[class3] (8, 1.5) -- (7, 1.5) -- (7, 0.5) -- (8,0.5) -- (8 ,1.5);
    
    %%%%%%%%%%%%%%%%%%%%%%%% Aktoren
    \node (AktB) at (7.5, -0.1) {Aktoren};
    \path[class5] (8, -1.5) -- (7, -1.5) -- (7, -0.5) -- (8,-0.5) -- (8 ,-1.5);
    \path[class5] (8, -3) -- (7, -3) -- (7, -2) -- (8,-2) -- (8 ,-3);
    \node (AktA) at (-7.5, -0.1) {Aktoren};
    \path[class5] (-8, -3) -- (-7, -3) -- (-7, -2) -- (-8,-2) -- (-8 ,-3);
    \path[class5] (-8, -1.5) -- (-7, -1.5) -- (-7, -0.5) -- (-8,-0.5) -- (-8 ,-1.5);
    \end{comment}
%%%%%%%%%%%%%%%%%%%%%%%%%%%%%%%%%%%%%%%%%%%%%%%

%%%%%%%%%%%%%%%%%%%%%%%%% Geschwungene Linie
%\draw[wagon, dash dot] plot [smooth] coordinates {(-6,8) (-5.2,6) (-6.5, 0) (-5.2, -4) (-6,-7)};
%\draw[wagon, dash dot] plot [smooth] coordinates {(6,8) (5.2,6) (6.5, 0) (5.2, -4) (6,-7)};
\draw[wagon, dash dot] (-5, 7) -- ( -5, -3.7) -- (5, -3.7) -- (5, 7) -- (-5, 7);

%%%%%%%%%%%%%%%%%%%%%%%%%% sicherheitskritischer Bereich
%Beschriftung
\node[gray] (Beschriftung) at (6.2, 3.5) {Schicht 1};
%Rechner 
\path[class1, fill=class1!50] (-1,5) rectangle +(2.5,1.5) node[pos = 0.5] (R1) {Rechner 1};
\path[class1, fill=class1!50] (-1,1) rectangle +(2.5,1.5) node[pos = 0.5] (R2) {Rechner 2};
\draw[<->, dashed, wagon] (0.25, 5) to (0.25,2.5);
\draw[->, class1] (R1) -| (-6.5, 4.7); %R1-Wagen
\draw[->, class1] (-1, 2) -| (-6.5, 2.8); %R2 - Wagen
\draw[<-, class1] (-0.5, 5) |- (-5.2, 3.9); %Wagen -R2
\draw[<-, class1] (-0.5, 2.5) |- (-5.2, 3.6); %Wagen-R!

%Zwischenspeicher
\path[class3, fill=class3!50] (-3.2, 1.15) ellipse (1.5 and 0.6);
\node[text width = 1.8cm] (B) at (-3., 1.15) {\small{Zwischen- speicher}};
\path[class1, thin] (-3.25, 1.7) |- (-1, 5.2) {}; %Zwischenspeicher - R1
\path[class1, thin] (-2., 1.5) -- (-1, 1.5) {}; %Zwischenspeicher - R2

%Wagen
\node[gray] (SAK1) at (-7, 3.5) {Sensoren};
\node[gray] (SAK2) at (-7, 3.0) {Aktoren};
\node[gray] (SAK3) at (-7, 4.0) {Kommunikation};
\node[gray] (SAK4) at (-7, 4.5) {Nachbarwagen};

%Speicher
\path[class1, fill=class1!50] (3, 1.15) ellipse (1.1 and 0.6);
\node (s) at (3, 1.15) {Speicher};
\draw[->, class1] (1.5, 5.75) -| (3.9, 1.5); %R1 - Speicher
\draw[->, class1] (1.5, 2.25) -| (2.7, 1.7); %R2 - Speicher
\draw[->, dashed, wagon] (2.3, 0.55) to (2.3, -1.5); %Speicher GSM
\draw[->, dashed, wagon] (2.3, 0.55) to (-2.25, -1.5); %Speicher- Nahbereich

%%%%%%%%%%%%%%%%%%%%%%%%%%%%%%%%% Linie
\path[wagon] (-5,0) -- (5,0) {};
\path[wagon] (-5,0.05) -- (5,0.05) {};
%%%%%%%%%%%%%%%%%%%%%%%%% nicht sicherheitskrischer Bereich
%Beschriftung
\node[gray] (Beschriftung) at (6.2, -1.8) {Schicht 2};
%GSM-Modul
\path[class4, fill=class4!50] (0.8, -3) rectangle +(3,1.5) node[pos = 0.5] (GSM) {GSM-Modul};
\draw[->, class4, dotted] (2., -3) to (2., -5);
\draw[<-, class4, dotted] (2.5, -3) to (2.5, -5);
\node[gray] (C) at (2.3, -5.3) {Cloud};


%Nahfunkbereich
\path[class4, fill=class4!50] (-4.3, -3) rectangle +(4,1.5) node[pos = 0.5] (NF) {Nahbereichsfunk};
\draw[->, class1] (-2.35, -1.5) to (-3,0.5);
\draw[->, class4, dotted] (-2.0, -3) to (-2.0, -5);
\draw[<-, class4, dotted] (-2.5, -3) to (-2.5, -5);
\node[gray] (B) at (-2.3, -5.3) {Bediener};

\end{tikzpicture}
    \caption{Vorgeschlagene Rechnerarchitektur für sichere und nicht sichere Funktionen und Daten des Güterwagen 4.0}
    \label{fig:Rechnerarchitektur}
\end{figure}
Im oberen Teil der Rechnerarchitektur ist der sicherheitskritische Bereich zu sehen. %Dieser ist (auf Dauer, wahrscheinlich) zulassungspflichtig. 
Dieser besteht aus zwei Rechnern, die für eine Zweikanaligkeit sorgen, einem Speicher,  für alle notwendigen Daten und einem Zwischenspeicher für die Bedienung.\par
Im unteren Teil ist der nicht sicherheitskritische Bereich. Dieser ist einkanalig ausgeführt. Dieser besteht aus dem Nahbereichsfunk und dem GSM-Modul als Fernbereichsfunk.\par
Angedeutet sind der Bediener, dessen Schnittstelle der Nahbereichsfunk darstellt, das GSM-Modul, die Schnittstelle zur Cloud und die Sensoren, Aktoren und Kommunikationseinheiten im Wagen.\par
Die beiden Rechner tauschen sich gegenseitig rückkopplungsfrei aus (gestrichelte graue Linie) und geben ihre Befehle (rote Pfeile nach außen) zweikanalig an Sensoren, Aktoren und die Kommunikationseinheiten im Wagen weiter. Die zurückkommenden Informationen (rote Pfeile zu Rechner 1 und Rechner 2) werden von den Rechnern verarbeitet und im Speicher (rote Pfeile zum Speicher) gesichert.\par
Befehle aus der Nahbereichsschnittstelle werden vom Bediener gegeben (gepunktete Linie zum Nahbereichsfunk) und im Zwischenspeicher (nur im Stillstand) an die Rechner weiter gegeben (dünne rote Linien) und dort verarbeitet. Wird der Befehl zur Unzeit oder während der Bewegung gegeben, verfällt er. Er wird nicht weiter zwischen gespeichert und später verarbeitet.\par
Alle notwendigen verarbeiteten Daten und Informationen werden im Speicher gespeichert (rote Pfeile zum Speicher) und dort entweder an den Nahbereichsfunk weitergegeben oder mit etwas Zeitverzug (und nur im Stillstand) über das GSM-Modul in der Cloud aktualisiert. Von dort aus sind sie für Cloud und Bediener einzusehen.%\par

\subsubsection{Hardware}
An Hardwarekomponenten sind Funkstellen für die Wagenkommunikation geplant.
\begin{itemize}
    \item Komponenten für Kurzstreckenfunk -- WLAN 60GHz/2,4GHz -- Wagen-Wagen, Wagen-Lok -- 1b
    \item Komponenten für mittlere Distanzen -- Wagen...Wagen, Wagen-Device
    \item Komponenten für Fernbereichsfunk -- Wagen-Cloud
\end{itemize}
\subsubsection{Software}
\begin{itemize}
    \item Lademanagement \label{sec:Lademanagment} -- 3d
    \item Anforderungen an sicheres funkgestütztes Bedienen und beobachten bei der Zugbildung -- 1d
    \item DI -- 1c
    \item \gls{WagonOS}
    \item Kommunikationsprotokolle - Datenkommunikation -- 1c
    \item automatische Zugbildung -- 1c
    \item automatische Bremsprobe -- 1c
    \item automatische Bremseinstellung
\end{itemize}