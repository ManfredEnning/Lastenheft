\documentclass[11pt,a4paper,parskip]{scrartcl}
%\def\xcolorversion{2.00}
%\def\xkeyvalversion{1.8}
\usepackage[utf8x]{inputenc}
\usepackage{ucs}
\usepackage{amsmath}
\usepackage{amsfonts}
\usepackage{amssymb}
\usepackage{graphicx}
\usepackage{lmodern}
\usepackage{hyperref}
\usepackage[ngerman]{babel}
\usepackage{pdfpages}
\bibliographystyle{alphadin}
%\usepackage[version=0.96]{pgf}
\usepackage{tikz} %Tikz
\usepackage{verbatim}

%Anforderungsnummerierung nach Raphael Beginn
%\theoremstyle{remark}
\newtheorem{rem}{Notiz}
%\theoremstyle{definition}
\newtheorem{feat}{Merkmal}
%\newtheoremstyle{definition}% name of the style to be used
%  {5pt}% measure of space to leave above the theorem. E.g.: 3pt
%  {5pt}% measure of space to leave below the theorem. E.g.: 3pt
%  {}% name of font to use in the body of the theorem
%  {}% measure of space to indent
%  {\bfseries}% name of head font
%  {}% punctuation between head and body
%  {\newline}% space after theorem head; " " = normal interword space
%  {}% Manually specify head
%Anforderungsnummerierung nach Raphael Ende

%Deckblatt Beginn
\author{Daniela Wilbring}
\renewcommand{\familydefault}{\sfdefault}
\title{Lastenheft}
\subtitle{Neue Elektronik- und Kommunikationssysteme für den intelligenten, vernetzten Güterwagen\\FKZ 16ES0850K}
\date{\today}
%Deckblattt Ende

\begin{document}

\setlength{\parskip}{5mm}% plus5mm minus5mm} %Absatzlänge
\maketitle \newpage
\tableofcontents \newpage
\section*{Zweck des Dokuments}
Das Lastenheft ist, nach \acrshort{DIN} 69901-5\footnote{Projektmanagement – Projektmanagementsysteme – Teil 5: Begriffe}, ''die vom Auftraggeber festgelegte Gesamtheit der Forderungen an die Lieferungen und Leistungen eines Auftragnehmers innerhalb eines (Projekt-)Auftrags''.\par
Ein Lastenheft ist wichtig in der Analysephase und zur Kommunikation innerhalb des Auftrages/Projekts. Es bietet eine ausführliche Beschreibung der Arbeitsleistung und dient als Kommunikationsbasis.\par
Auf Basis des Lastenheftes wird das Pflichtenheft vom Auftragnehmer erarbeitet.\par
Untersuchungen zeigen, dass Fehler in der Produktentwicklung bei einer späten Aufdeckung und Behebung teurer werden als bei einer früheren Erkennung. Deshalb sollte man schon beim Lastenheft eine hohe Qualität anstreben.\footnote{Quelle: \url{https://www.pm-blog.eu/themen/methoden/lastenhefte-eine-schrittweise-anleitung-fur-den-perfekten-aufbau.html} }\par
In diesem Fall wird das Lastenheft im Rahmen des Projektes "Neue Elektronik- und Kommunikationssysteme für den intelligenten, vernetzten Güterwagen" von der FH Aachen als Vorarbeit für das Pflichtenheft erstellt. \par
Klare gemeinsame Ziele innerhalb des Projektes sollen zu einer besseren und strukturierten Zusammenarbeit führen.
%Das Lastenheft wird aufgrund von Veröffentlichungen der Projektsteller und dem Gesamtverbundantrag des Projektes erstellt% \newpage
\section{Einleitung}
In diesem Lastenheft wird anhand eines Schiebewandwagens der grobe Ablauf eines Umlaufs im bisherigen System erläutert und daraufhin Anforderungen an den neuen Güterwagen erstellt. Als Motivation zeigen sich %neben der oben bereits erwähnten Verringerung von Verkehrsstaus und Schadstoffemissionen auch 
eine Erhöhung der Prozesssicherheit, Verringerung der aufzuwendenden Zeit am Wagen, eine Gestaltung von attraktiveren Arbeitsplätzen durch eine ergonomischere Arbeitsgestaltung und mögliche Kosteneinsparungen an der Infrastruktur.\par
Der Güterverkehr in Deutschland wird bislang zu über 70\% von LKWs gestemmt, was Verkehrsstaus und hohe Schadstoffemissionen verursacht. Das Zukunftsprojekt Industrie 4.0 bietet dem Schienengüterverkehr die einzigartige Chance, durch intelligente Steuerung und Vernetzung Transportprozesse extrem flexibel, effizient und schadstoffarm zu gestalten. Realisiert werden kann dies durch die Integration vielfältiger moderner Sensorik und Elektronik in den Güterwagen4.0.\cite{AZAP}\par
%Im Projekt werden Sensoren und Elektroniksysteme zur Realisierung eines „intelligenten“, mittels Industrie 4.0-Technologien vernetzten Güterwagens entwickelt. Die Sensorik dient dabei der Online-Erfassung relevanter Güterwagen- und Zugverbund-Daten zur Zustands- und Verschleißanalyse. Durch diese Informationen wird erstmals eine durchgängige Logistik und die von Kunden geforderte Transparenz der Lieferkette sowie eine vorausschauende Planung von Wartungszyklen und Rentabilität für den Güterwagenbetreiber ermöglicht. Als weiterer Lösungsansatz werden Aktoren entwickelt und in den Güterwagen4.0 integriert, mit denen die beim Zusammenstellen bzw. Trennen von Güterwagen erforderlichen aufwendigen und oft sicherheitsrelevanten manuellen Tätigkeiten künftig vollautomatisiert durchgeführt und sensorisch online überwacht werden können.\footnote{Antrag auf Gewährung einer Bundeszuwendung auf Ausgabenbasis (AZAP), 26.04.2018}\\
%Der Güterwagen4.0 stellt einen wichtigen Baustein zur Lösung der gravierenden Verkehrsprobleme dar. Die Ergebnisse des Vorhabens tragen wesentlich dazu bei, durch einen zukunftsfähigen Schienenverkehr Verkehrsstaus und Schadstoffemissionen zu vermindern. Die angestrebten sensorischen Fähigkeiten des Güterwagen4.0 erlauben perspektivisch auch eine Übertragung in den Schienenpersonenverkehr.\footnote{Antrag auf Gewährung einer Bundeszuwendung auf Ausgabenbasis (AZAP), 26.04.2018}\\\\
Herausforderungen wie die Migration und Annahme des Systems sollen beachtet werden. Die Kosten des Systems, sowie dessen Einbau müssen konkretisiert werden. Prozesse für produktivere Arbeitsabläufe mit dem neuen Güterwagen müssen gestaltet werden. Arbeitsabläufe bei Defekten konkretisiert werden.\par
Es geht bei dieser Art der Automatisierung nicht darum Arbeitsplätze abzuschaffen, sondern vorallem darum sie attraktiver zu gestalten um das Problem des Personalmangels in den Griff zu bekommen und neues junges Personal zu finden. Gleichzeitig darf natürlich nicht die Sicherheit des Systems leiden. Die Sicherheit muss mindestens genauso bleiben, wie sie beim bisherigen System ist.\par
Es ist geplant eine Möglichkeit für produktivere Verkehre zu schaffen.\par
Der Ist-Zustand im nächsten Kapitel wird zeigen, dass viele manuelle Tätigkeiten für die Beladung und Abfertigung sowie die Rangiervorgänge von Güterwagen im Einzelwagenverkehr notwendig sind. Dies verursacht hohe (Personal-)Kosten durch die benötigte Zeit und das benötigte Personal, sowie auch Kosten an der Verladestelle, wenn diese aufgrund von Kupplungs- und Rangiervorgängen still steht.\par
Diese Punkte sollen mit diesem Projekt angegangen werden. Dafür sollen verschiedene technische Stufen zu Ausstattungspaketen, die aufeinander aufbauen, zusammengestellt werden. Jedes Ausstattungspaket soll entweder bei gleicher Sicherheit manuelle Tätigkeiten abbauen oder vereinfachen, oder die Sicherheit erhöhen.\par


%Hier fehlt eine Erklärung wo die Reise hin gehen soll, ein Fokus, was wir uns vom GW40 erhoffen und eine sensibilisierung für den Prozess und das Vorgehen zur Änderung dessen.\\
%Automatisierungslösungenen  Automatisches Abdrücken mit AK o. automatischer Schraubenkupplung – Trennstellen, Kommunikation mit Bergrechner – Vorbereitung Ablauf\\
• Vorteile des Systems\\
Dezentralität\\
Zukunftssicher\\
muss nicht auf AK warten, funktioniert aber auch damit\\
Wertschöpfungsprozess\\
Automatisierung\\
Automatisierung Bremse – Bremse lösen, lüften, Handbremse, Berechnung Bremsgewichte, Durchgängigkeit HL, automatische Bremsprobe, automatischer Bremszettel\\
Vorbereitung Trennstellen – HL-Absperrhähne Schließen, lüften, 2x Signal = trennen – Signal: kann, soll trennen\\
 %\newpage
\section{Ist-Zustand} %Beschreibung wo Produktivität vergudet wird um daraus ableiten zu können, was benötigt wird. % Kochsiek, Knoll nach Meinung zu diesem Kapitel fragen!
Im Folgenden wir der Umlauf eines \gls{Schiebewandwagen} (Habinns) beschrieben. Der Wagen wird in einem \gls{Gleisanschluss}/RailPort mit palletiertem Gut beladen, läuft dann durch das deutsche/europäische Einzelwagensystem und wird in einer Ladestelle an einem Gleisanschluss oder RailPort entladen. Fokus der Beschreibung sind die durchzuführenden manuellen Tätigkeiten. Davon leitet sich im Folgenden ein Anforderungskatalog an einen aktiven, kommunikativen Güterwagen 4.0 ab, der viele oder alle der manuellen Tätigkeiten durch Technikfunktionen ersetzt.

\subsection{Vorgänge an der Ladestelle/im Gleisanschluss}
%In diesem Unterkapitel wird eine kurze Beschreibung def Fahrwege, Bewegung von Fahrzeugen, der Personaltätigkeit sowie der Beladung, dem Wagenwechsel, der Ladungsicherung und der benötigten Transportdokumente gegeben.
\subsubsection{Fahrweg} \label{sec:Fahrweg}
Die Bewegung von Wagen erfolgt prinzipiell auf Gleisen, eine Verzweigung von Fahrwegen erfordert Weichen. In kleinen und mittleren Gleisanschlüssen sind dies überwiegend handbetätigte Weichen. 
\subsubsection{Bewegung der Wagen} \label{sec:BewdWagen}
Je nach Wagenaufkommen kommen folgenden Methoden der Wagenbewegung in Betracht:
\begin{itemize}
	\item Bedienung durch das \acrshort{EVU}, welches auch die Zustellfahrten durchführt
	\item Eigene (zum \gls{Gleisanschluss} zugehörige) Rangierlokomotive
	\item Eigenes Rangierhilfsmittel
	\begin{itemize}
	    \item Gleisfahrbar (Rangierroboter)
	    \item Zweiwegefahrzeug
	\end{itemize}
	\item Stationäre Verschubeinrichtung (Seilzuganlage)
	\item Verschub mittels Muskelkraft von Tieren oder Menschen (z.B. Knippstange, Wagenrücker)
\end{itemize}
\subsubsection{Personalfähigkeiten}\label{sec:Personal}
Allen Methoden ist gemeinsam, dass Sie spezielle Personalfähigkeiten bzw. eine entsprechende Ausbildung benötigen. Dies ist nicht zuletzt auf die Tatsache zurückzuführen, dass ein einmal in Bewegung gesetzter Güterwagen auch bei langsamer Fahrt eine hohe kinetische Energie aufweist. Zusätzlich ist die Bedienung der Wagenbremsen (Luft- und/oder Handbremse) für ungeschultes Personal kompliziert und fehleranfällig.
\subsubsection{Bereitstellung}
Eine Bereitstellung der Wagen findet aus einem \gls{Vorbahnhof} als Einfahrgruppe oder Ordnungsgruppe statt. Diese stehen dort im Allgemeinen mit entlüfteter Bremse und vorgelegtem Hemmschuh oder angelegter Handbremse.
\subsubsection{Beladung}
Das Beladen von Güterwagen mit Paletten erfolgt von einer Rampe oder einer Ladekante mit Überfahrbrücke aus in der Regel per Gabelstapler. Im Gegensatz zur Heckbeladung von LKW existieren kaum Lösungen zur Automatisierung der Beladung. Dies ist ein Logistikprozess des Verladers.
\subsubsection{Ladungssicherung}
Während oder nach der Beladung muss die Ladung gegen Verrutschen gesichert werden. Bei Schiebewandwagen erfolgt dies unter anderem durch von Hand verschiebbare und durch Verriegeln zu sichernde Zwischenwände. Auch für diese Aufgabe ist der Verlader verantwortlich. Er muss den Wagen als bahntechnisch sicher an das EVU übergeben. Das EVU darf nur mit einem augenscheinlich (also für den Beobachter sicheren) Wagen fahren. Dies betrifft in einem geschlossenen Wagen nicht die Ladung, aber die Türen und Verschlüsse; auf einem offenen (Holz-)Transporter aber auch die sichtbare Ladung. Bei Gefahrgut müssen noch weitere Kontrollen von geschultem Personal durchgeführt werden und der Wagen passend gekennzeichnet werden.
\subsubsection{Wagenwechsel}
Eine Ladekante ist meist für Einzelwagen oder kleine Gruppen (bis max. ca. 4 Wagen) gestaltet, so dass häufig Wagen an der Ladestelle getauscht werden müssen, bevor die Zustellfahrt erfolgt. Dies liegt unteranderem an der Seitenbeladung, für die viel Platz benötigt wird. Diese hat, gegenüber der Heckbeladung bei LKW, den Vorteil der Parallelisierbarkeit. LKW gehen nach der Beladung direkt in den Umlauf, sodass bei diesen das häufige Umsetzen nicht als Nachteil gesehen wird. \par
Bei Gefahrgut ist es auch üblich Weichen abzuschließen und diese vom Verlader öffnen zu lassen, wenn die Wagen fertig beladen und gesichert sind.\par
Diese Bewegung kann rangiert oder verschoben werden. \\
Findet die Wagenbewegung mit \gls{Lokrangierfuehrer} (\acrshort{LRF}) und Lok statt, wird rangiert; meist ohne Luftkupplung und entsprechend auch ohne \gls{Bremsprobe}. Ob dies möglich ist hängt von der Lokomotive und der Last, der Achsen und der Neigung des Ladegleises ab. Muss aufgrund dieser Eigenschaften mit Luft gefahren werden, ist auch eine vereinfachte \gls{Bremsprobe} notwendig. Häufig findet diese Rangierfahrt in der Bremsstellung G statt. \\
Verschoben werden die Wagen wenn die Wagenbewegung ohne Lok und Lokrangierführer stattfindet. Dies ist zum Beispiel mit Unimog im 2-Wege-Einsatz statt. Hier verschiebt ein angelernter und unterwiesener Verlademitarbeiter mit Sicherheitsposten. Die Anlernung und Unterweisung findet vom zuständigen \gls{Eisenbahnbetriebsleiter} (\acrshort{EBL}) statt. Im Allgemeinen werden hier einzelene Wagen oder kleinere Wagengruppen für nur wenige Meter verschoben.
\subsubsection{Transportdokumente}\label{sec:Transdoc}
Im LKW-Bereich entstehen bereits Lösungen zur papierlosen Transportabwicklung einschließlich Behandlung von Gefahrgutdokumenten. Bei Bahntransport herrscht die klassische Methode der Übergabe von Frachtdokumenten an das \acrshort{EVU} vor. Oft werden diese auch elektronisch übergeben. Dann sind diese später nur digital Verfügbar. Ausgenommen sind Gefahrgutscheine, diese werden als Papierzettel mitgeführt. Auch die Wagen werden, vorallem im Einzelwagenverkehr noch "`bezettelt"'\footnote{Siehe dazu auch VDV-Schrift 758 - Prüfen von Güterwagen im Eisenbahnbetrieb}. Diese Zettel sind Vereinfachungen des Frachtbriefes, die direkt auf dem Wagen mitgeführt werden. Auf diesen steht im Allgemeinen aus was die Ladung besteht, woher diese kommt und wohin sie geht. Auch weitere Besonderheiten werden hier vermerkt.
%Der Lkw wird meist in Längsrichtung durch das Heck an einem Tor / Vorsatzschleuse beladen. Für die eigentliche Beladung ist es  ungünstiger als die Seitenbeladung, es führt aber zu einer effizienten Flächennutzung in der Halle (Zeilenstruktur). Daher ist die Heckbeladung gerade bei Logistik/Verteilzentren und Speditionen heute die vorherrschende Methode.  
%Für spezielle Güter / Wagen ist Beladung mit Hallenkran von oben üblich (Papier, Stahlcoils)
\subsection{Abholen im Gleisanschluss}
Die Abholung von Wagen erfolgt entweder durch das \acrshort{EVU} unmittelbar an der Ladestelle oder -- wenn lokale Rangierhilfsmittel zur Verfügung stehen -- von einem Übergabegleis im Werksgelände. Die Bedienung des Anschlusses ist in der Regel Teil eines Umlaufs, in dem mehrere Anschlüsse nacheinander bedient werden.
\subsubsection{Luft- und mechanische Kupplung}\label{sec:LuftumechKup}
Der oder die Wagen stehen im festgelegten Zustand zur Abholung bereit. Die Luftbremse ist gewöhnlich außer Funktion, der Reserveluftbehälter ist leer. Nach dem Ansetzen der Lok bzw. des aktuell letzten Wagens der Rangierabteilung ist manuell zu kuppeln. Der Lokführer oder Rangierbegleiter "`taucht"' dazu unter den Puffern durch und verbindet zunächst manuell die Schraubenkupplung. Danach wird die Hauptluftleitung gekuppelt und die Absperrhähne der \acrshort{HL} geöffnet. %prüfen
\subsubsection{(Vereinfachte) Bremsprobe}\label{sec:vBremsprobe}
Im Anschluss erfolgt eine (vereinfachte) \gls{Bremsprobe}. Dazu wird zunächst am letzten Wagen der Gelöstzustand überprüft, dann der \acrshort{HL}-Druck abgesenkt und das Führerbremsventil abgesperrt. Die Bremsen müssen anlegen. Im Anschluss wird zur Durchgängigkeitsprüfung wieder die Fahrstellung eingenommen und das \acrshort{HL} Absperrventil des letzten Wagen für mindestens 15 Sekunden geöffnet. Die Bremsen müssen anlegen und wieder lösen.
\subsubsection{Technische Wagenbehandlung}\label{sec:tWb}
Vor Beginn der Rangier-/\gls{Sperrfahrt} ist eine technische Wagenbehandlung (\acrshort{twb}) der Stufe 1 (siehe dazu auch Anhang \ref{sec:ATWb}) durchzuführen. 
\subsubsection{Zustellfahrt}\label{sec:Zustellfahrt}
Erfolgt die Zustellfahrt über die freie Strecke, ist das Streckengleis durch das Stellwerk für \gls{Zugfahrt}en zu sperren. Wenn der Gleisanschluss selbst nicht als Bahnhof ausgebildet ist, also keine eigenen Ausfahrsignale besitzt, erfolgt die Fahrt als \gls{Sperrfahrt}. Wenn der \gls{Gleisanschluss} an ein Bahnhofsgleis angeschlossen ist, handelt es sich um eine \gls{Rangierfahrt}. In jedem Fall ist es eine Fahrt auf Sicht mit geringer Geschwindigkeit. Worst Case für die Nutzung der Strecke mit \gls{Zugfahrt}en wäre eine geschobene \gls{Sperrfahrt}, diese ist nicht nur mit geringer Geschwindigkeit sondern auch mit höherem Personalaufwand durch Besetzung der Spitze verbunden.

\subsection{Fahrt zum Knotenbahnhof und Zugbildung}
\subsubsection{Zugfahrt zum Satellitenbahnhof}\label{sec:Zugfahrt}
Satellitenbahnhöfe sind Bahnhöfe, in dem im Einzelwagenverkehr \gls{Zugfahrt}en enden oder beginnen. Sie sind in der Regel für die Übergabegruppe nur Durchgangsstation. Die Kombination der Übergabe mit weiteren Übergaben von anderen Anschlüssen/Satelliten erfolgt im \gls{Knotenbahnhof}. Weil die Fahrt vom Satelliten- zum \gls{Knotenbahnhof} eine \gls{Zugfahrt} (mit in der Regel Geschwindigkeiten $\ge 80 km/h$) darstellt, ist vor Beginn der Fahrt eine Berechnung des Bremsgewichts und eine volle \gls{Bremsprobe} durchzuführen.%erfragen
\subsubsection{Rangiervorgänge im Umsetzbetrieb}\label{sec:Rangierfahrt}
Knotenbahnhöfe können über \gls{Ablaufberg}e verfügen, in der Regel ist dies aber nicht der Fall und die Rangiervorgänge finden im so genannten Umsetzbetrieb statt. Dabei werden jeweils Wagengruppen angekuppelt (je nach Lok ist die Kupplung der Luft meist nicht notwendig) und durch eine Sägefahrt in ein anderes Gleis versetzt und dort gekuppelt. Das Stellen der Weichen für diese Vorgänge erfolgt meist vom Stellwerk des Bahnhofs aus oder durch eine \gls{EOW}\footnote{Elektrisch ortsgestellte Weichen}-Anlage.
\subsubsection{Übergabe der Wagen}\label{sec:UEdWagen}
Auch vor der Abfahrt des aus Übergaben zusammengestellten Zuges ist wieder ein \gls{Bremsprobe} und eine technische Wagenbehandlung fällig. Außerdem ist wieder das Bremsgewicht und die Bremshunderstel zu berechnen. Bei jeder Zugzusammenstellung sind Papiere zu prüfen und zu übergeben.

\subsection{Fahrt zum Knotenbahnhof des Zielbereichs über einen oder mehrere Rangierbahnhöfe}
\subsubsection{Rangiervorgänge im Knotenbahnhof}\label{sec:RangKnoten}
Ab dem \gls{Knotenbahnhof} sind die weiteren Fahrten/Umstellvorgänge wie folgt charakterisiert:
\begin{itemize}
    \item Züge haben idealerweise die volle Länge von 700 m
    \item Zugumstellungen erfolgen in den großen Rangierbahnhöfen
\end{itemize}
Wie bei einem Postverteilsystem haben die Rangierbahnhöfe die Aufgabe, hereinkommende Züge aufzulösen und die Wagen auf neue Züge umzustellen, die sie ihrem Ziel näher bringen. Dieser Abschnitt des Einzelwagenverkehrs ist hochautomatisiert und -produktiv. Dennoch verbleiben auch hier große Automatisierungslücken. 
\subsubsection{Vorprüfung}\label{sec:Vorpruefung}
Am Beispiel der Behandlung der Wagen in einem automatisierten \gls{Rangierbahnhof} wird dies deutlich:\par
Nach der Einfahrt des Zuges in die Einfahrgruppe wird durch den so genannten "`Vergleicher"' die Übereinstimmung der übermittelten Zugliste mit der tatsächlichen Reihung geprüft. Grund für diesen scheinbar überflüssigen Schritt ist die Tatsache, dass das Herausrangieren von fehlerhaften Wagen auf der Strecke (meist in der Folge einer Alarmmeldung einer Heißläuferortungsanlage) nicht in der Informationstechnik berücksichtigt wird.\par
Die weiteren Schritte bei der Vorbereitung sind wie folgt:
\begin{enumerate}
    \item Zugbremse anlegen, \acrshort{HL} entlüften, Lok abkuppeln
    \item Zug festlegen (i.d.R. durch \gls{Hemmschuh}e)
    \item Wagen/-gruppen vereinzeln. Dafür an den vorgesehenen Trennstellen nach Zerlegeliste Luft- und mechanische Schraubenkupplungen trennen und Luftbremse durch Ziehen am Lösezug lösen (Entlüften der A-Kammer/des Reserveluftbehälters). Je nach Rangierverfahren wird die Schraubenkupplung komplett ausgehängt (vorentkuppelt abdrücken) oder sie wird lose eingehängt gelassen und erst am \gls{Ablaufberg} durch den "`Stangler"' ausgeworfen.
\end{enumerate}
\subsubsection{Abdrücken am Ablaufberg}\label{sec:Abdruecken}
Wenn diese Vorbereitungen abgeschlossen sind, wird der Zug zum Abdrücken freigegeben. Die Abdrücklokomotive schiebt dann -- nach dem Entfernen der Hemmschuhe -- den Zug mit Geschwindigkeiten von 1,2 bis 3 m/s %ist das so?
über den \gls{Ablaufberg}. Hinter dem Gipfel vereinzeln sich die Wagen. Durch den Schwerkrafteinfluss, werden diese durch ein gestaffeltes System von mechanischen Gleisbremsen, auf die Eintrittsgeschwindigkeit im Richtungsgleis heruntergebremst und laufen langsam in die Richtungsgleise ein.  Klassisch ist das das Arbeitsfeld der Hemmschuhleger, die durch gezieltes Platzieren von Hemmschuhen die Wagen punktgenau vor dem letzten Wagen im Richtungsgleis abbremsen.
\subsubsection{Automatisiertes Abdrücken}\label{sec:automAbdruecken}
In den großen Rangierbahnhöfen sind die Fortschritte der Automatisierungstechnik sichtbar. In allen Anlagen gibt es automatische Steuerungen der Verteilweichen. In einigen Anlagen sind die Ablaufsteuerungen mit einer automatischen Steuerung der Abdrücklokomotiven gekuppelt, so das während des Abdrückvorgangs kein manueller Steuer- eingriff notwendig ist (die Rückfahrt zum nächsten Einsatz bleibt aber Handarbeit).In allen Anlagen werden die Bremsen (Bergbremse und Talbremse) geregelt betrieben, so dass unterschiedliche Laufeigenschaften der Wagen und Windeinflüsse automatisch kompensiert werden und in den meisten Anlagen ist das Legen von Hemmschuhen zur Zielbremsung ersetzt worden durch automatische Fördereinrichtungen (Räum- und Beidrückförderer).\par
Wenn dieser hochautomatisierte Prozess für den betrachteten Wagen im Richtungsgleis abgeschlossen ist, geht es wieder in Handarbeit weiter.
\subsubsection{Zugvorbereitung}\label{sec:Zugvorbereitung}
Nach dem Schließen des Richtungsgleises für weitere Abläufe werden die Wagen provisorisch miteinander gekuppelt und die Wagengruppe wird in ein Gleis für die Nachbehandlung gezogen. (In einigen Anlagen erfolgt die Nachbehandlung im Richtungsgleis). Dort werden die folgenden Schritte durchgeführt:
\begin{enumerate}
    \item Mechanisch kuppeln
    \item Luftleitungen kuppeln, \acrshort{HL} Absperrhähne öffnen
    \item \acrshort{HL} Absperrhahn am (neuen) \gls{Zugschluss} schließen, Zugschlusstafel stecken
    \item Bremsberechnung (Zuggewicht, Bremsgewicht, Bremseigenschaften der Lok)
    \item Bremsstellungswechsel je nach Bremsart der folgenden \gls{Zugfahrt}
    \item Volle \gls{Bremsprobe}, in der Regel mittels stationärer \gls{Bremsprobeanlage}
    \item Technische Wagenbehandlung, Stufe 3
\end{enumerate}
\subsubsection{Nachordnung}
Falls die örtlichen Verhältnisse in den Zielgleisanschlüssen und die Reihenfolge von deren Bedienung bekannt sind, erfolgt noch eine Nachordnung der Wagen nach Empfängern durch einen erneuten Berglauf oder in einer Nachordungsgruppe. Dieses ist effizienter als manuelle Umstellvorgänge mittels Sägefahrten in Knoten oder Satellitenbahnhöfen.

\subsection{Fahrt zum Satellitenbahnhof}
\subsubsection{Zugfahrt}
Der so vorbereitete Zug wird dann für den nächsten Abschnitt von der vorgesehenen Streckenlok übernommen. Nach einer vereinfachten \gls{Bremsprobe} kann die \gls{Zugfahrt} beginnen.\par
\subsubsection{Umsetzbetrieb}
Am Satellitenbahnhof angekommen wird der Zug wieder, wie oben beschrieben, im Umsetzbetrieb umgestellt. Dabei werden
jeweils Wagengruppen angekuppelt (je nach Lok ist die Kupplung der Luft meist nicht notwendig) und durch eine Sägefahrt in ein anderes Gleis versetzt und dort gekuppelt.
\subsubsection{Übergabe der Wagen/Zugvorbereitung}
Auch  vor  dieser  Abfahrt  wird der  zusammengestellte  Zug  ist  wieder mittels \gls{Bremsprobe} und technischer Wagenbehandlung geprüft. Ebenso die dazugehörigen Papiere.

\subsection{Fahrt zum Gleisanschluss/zur Ladestelle}
Wenn die Zustellfahrt über die freie Strecke erfolgt, ist das Streckengleis für Zugfahrten zu sperren. Diese Fahrt kann als Sperrfahrt oder Rangierfahrt stattfinden.
\subsubsection{Rausrangieren einzelnen Wagen}
Angekommen am \gls{Gleisanschluss} müssen einzelne Wagen wieder aus dem Zugverband rausgangiert oder zumindest abgekuppelt werden.
\subsubsection{Entladen}
Angekommen am \gls{Gleisanschluss} muss der Wagen noch zur Entladestelle gebracht werden und entladen werden. %\newpage
\section{Soll-Zustand}
Der Ist-Zustand im vorigen Kapitel hat gezeigt, dass viele manuelle Tätigkeiten für die Beladung und Abfertigung sowie die Rangiervorgänge von Güterwagen im Einzelwagenverkehr notwendig sind.\par
Dies verursacht hohe (Personal-)Kosten durch die benötigte Zeit und das benötigte Personal, sowie auch Kosten an der Verladestelle, wenn diese aufgrund von Kupplungs- und Rangiervorgängen still steht.\par
Dies Punkte sollen mit diesem Projekt angegangen werden. Dafür sollen verschiedene technische Stufen zu Ausstattungspaketen, die aufeinander aufbauen, zusammengestellt werden. Jedes Ausstattungspaket soll entweder bei gleicher Sicherheit manuelle Tätigkeiten abbauen oder vereinfachen, oder die Sicherheit erhöhen.\par
An der Bewegung der Wagen im Wagenverband an sich mit Hilfe einer Lok oder eines anderen Rangierhilfsmittel kann nichts groß automatisiert werden. Siehe dazu auch Kapitel \ref{sec:Zustellfahrt}, \ref{sec:Zugfahrt} und \ref{sec:Rangierfahrt}. Automatisierungen können aber selbst verständlich im Bereich der Kupplung, Bremse oder auch der informationstechnischen Prozesse stattfinden.\par
Zur besseren Einordnung sollen nun einige Komponenten definiert und erläutert werden aus denen sich verschiedene Stufen ergeben und aus denen sich wiederum später verschiedene Anforderungen ergeben.

\subsection{Bremse}
\subsubsection{Bremsprobe}
Die \gls{Bremsprobe} findet zur Vorbereitung einer Sperr-, Rangier- oder Zugfahrt statt und überprüft die Funktionsfähigkeit des Bremssystems im Zug- oder Wagenverbund.\par
\begin{figure}[htbp] 
    \begin{center}
            \includegraphics[width=9cm]{Bilder/bremsprobe.png}
            \caption{Schematische Übersicht der Bremsprobe}
            \label{fig:Bremsprobe}
    \end{center}
\end{figure} 
Sie wird von Bremsprobeberechtigten nach genau vorgeschriebenem Ablauf durchgeführt. Ja nach Zustand des Zuges und Fälligkeit der \gls{Bremsprobe} wird eine volle, vereinfachte, stationäre oder Führerraumbremsprobe durchgeführt. Der genaue Ablauf ist in der \acrshort{RIL} 915 oder der VDV-Schrift 757 geregelt und als grobe Übersicht in Abbildung \ref{fig:Bremsprobe} zu sehen.\par
\textit{Durch bekannte Vorprüfungen kann die vollständige \gls{Bremsprobe} vereinfacht werden. Zum Beispiel durch sogenannte vorgeprüfte Gruppen.\\
\textbf{Siehe auch DI, Luftventile\\
\ref{sec:vBremsprobe}, \ref{sec:UEdWagen}, \ref{sec:RangKnoten}}}
\subsubsection{technische Wagenbehandlung}

\subsubsection{Bremshundertstel/Bremsberechnung}
Die Bremshundertstel werden bisher, genau wie das bremsgewicht, händisch mittels Bremszettel berechnet. \textbf{BILD}. Auf diesem Vordruck trägt der Triebfahrzeugführer Achszahl, Zugmasse und Bremsgewichte des Zuges ein. Daraus berechnen sich die Bremshundertstel. Nach Vergleich dieser mit den Mindest-Bremshundertstel der Strecke, ergibt sich Höchstgeschwindigkeit und Bremsstellung. die Bremsstellung wird nach jeder neuen Zusammenstellung und vor Fahrtantritt, siehe Kapitel \ref{sec:UEdWagen}, neu berechnet.\par
Diese manuelle Berechnung ist für Triebfahrzeugführer tägliche Arbeit, dennnoch ist sie Fehleranfällig und kann durch Rechnereinsatz einfach automatisiert werden.

\subsection{Kupplung}
Wie in Kapitel \ref{sec:Personal} angedeutet und in Kapitel \ref{sec:LuftumechKup} beschrieben, ist das mechanische Kuppeln und das Kuppeln von Luft aufwändig, körperlich anstrengend und fehlerbehaftet. \par
Zur Automatisierung von mechanischen Kupplungen gibt es Kupplungsrobotoren\footnote{Zum Beispiel die Bahn-Kupplungs-Robotoren BaKuRo und EntKuRo}, dort muss aber weiter händisch Luft gekuppelt werden \textbf{Ist das so?}. Alternativ ist die Automatische Kupplung eine Lösung, aber auch nach deren Einführung im vorigen Jahrhundert ist eine flächendeckende Nutzung im Güterverkehr noch nicht realisiert.\par
\textbf{Irgendwas zu Luft und elektrischen Ventilen.}\\
\textit{siehe auch \ref{sec:Vorpruefung} und \ref{sec:Zugvorbereitung}}
\par
Aufgrund dieser Punkte ist es erst einmal sinnvoll mich der mechanischen Kupplung weiter zu machen und Lösungen für die Automatische Kupplung kompatibel zu halten. Vor allem muss aber die Luftkupplung vereinfacht wreden wo es geht \textbf{ SIEHE AUCH: Luftventile.}

\subsection{Bewegung des Wagens}
In Kapitel \ref{sec:BewdWagen} werden die verschiedenen Möglichkeiten der Wagenbewegung betrachtet. Im Allgemeinen werden dafür zusätzliche Fahrzeuge, Personal und eine Gleisanlage benötigt. Je nach Beschaffung der Gleisanlage und die in Kapitel \ref{sec:Fahrweg} angesprochene Einschränkung, werden zusätzliche Sägefahrten zur korrekten Einsortierung der Wagen auf verschiedene Gleise benötigt. \par
Ein eigener Antrieb auf jedem Wagen, der eine selbstständige Bewegung in geringer Geschwindigkeit zulässt wäre hier eine Lösung.\par
\textit{\textbf{Probleme:} Stromversorgung, Akkuaufladung, Antrieb, Sicherheit}
\subsubsection{Adrücken}
DAs Abdrücken am Ablaufberg, siehe Kapitel \ref{sec:Abdruecken} ist bereits auf großen Rangierbahnhöfen automatisiert, siehe Kapitel \ref{sec:automAbdruecken}, dennoch ist auch dies Handarbeit zur Vorbereitung. Die Wagen müssen vorentkuppelt werden, sowohl mit der Luftkupplung, als auch mechanisch, dann aber wieder, mit einem Hemmschuh, festgelegt werden und kurz vorm Ablaufberg vollständig entkuppelt. 
Hemmschuh, vereinzeln, vorentkuppelt, Geschwindigkeit
\subsection{Sicherheit/Zugintegrität}
\subsubsection{Personalschulung}
Neue Systeme benötigen immer Zeit zur Einführung und Schulung des Personlas. Das gilt auch für dieses Projekt, aber auch jetzt schon sind die Handgriffe, wie in Kapitel \ref{sec:Personal} angesprochen, kompliziert und fehleranfällig. Durch informationstechnische Prozesse \textbf{MEHR DAZU UNTER}, die zusätzlich im Hintergrund laufen und durch weniger Handgriffe im üblichen Betrieb, sollen diese Prozesse einfacher und sicherer werden. Zusätzlich sollen die neu benötigten Handgriffe intuitiv zu den alten Handgriffen passen oder leicht erlenrnbar sein. Zusätzliches Schulungsmaterial soll leicht verständlich und intuitiv gestaltet werden.\par
\textit{\textbf{Probleme:} Annahme der Systeme, Kosten für die Schulung}

\subsection{Informationstechnische Prozesse}
\subsubsection{Transportdokumente}
Wie bereits In Kapitel \ref{sec:Transdoc} beschrieben, gibt es die papierlose Transportabwicklung inklusive Gefahrgutdokumenten bisher im LKW-Bereich, allerdings nciht im Bahnsektor. Ein Grund dafür ist die fehlende Dateninfrastruktur.\par
Durch eine durchgängige Stromversorgung auf dem Wagen und einen bahntauglichen Rechner sowie entsprechende Datenverbindungen kann dieses Problem angegangen werden.\par
\textit{\textbf{Siehe dazu auch:} Stromversorgung, Rechner, Datenverbindung, DI (Digitale Identität)}
\subsubsection{Vergleicher}
\ref{sec:Vorpruefung}
\subsubsection{Zugvorbereitung}
\ref{sec:Vorpruefung} %\newpage
\section{Anforderungsbeschreibung}
\textbf{Einfügen: Wie lang sind Lösezeiten, Verwendung Handbremse, Umlegen Endabsperrhähne}\par
Zielgrupper - Rangierende, WagenmeisterInnen, Bremsprobeberechtige, Tf, ...\par
\textbf{Sicherheitsrelevanz, Bedienen, Beobachten, Anbringungspunkte von Bauteilen -- gefedert? -- Bahntauglichkeit}\par
\begin{figure}[htbp] 
    	\begin{center}
            		\begin{tikzpicture}[scale = 1]
		\draw[line width = .6cm, gray!30, -stealth] (-1.5,6.3) -- (-1.5,0);
                                \begin{axis}[ 
                                font = \small,
                                width = 10cm, height = 8cm,
                                xbar, xmin=0, xmax = 55,
                                xlabel={Zeit [min]},
                                symbolic y coords={%
                                {Freigabe prüfen},{Bremse freigeben},{Bremsung prüfen}, {Bremsen betätigen}, {Dichtigkeitsprüfung}, {Zustandsbewertung},{Füllen Bremsleitung},{Zugvorbereitung} },
                                ytick=data,
                                nodes near coords, 
                                %nodes near coords align={horizontal},
                                ytick=data,
                                % nodes near coords align={vertical},
                                bar width=7pt,
                %            legend style={
                   %             at={(1,1.1)},
                    %      	anchor=east},
                  %              legend columns = 2,
                                ]
                                \addplot coordinates {                          
                                (39.5,{Zugvorbereitung})
                                (40,{Füllen Bremsleitung})
                                (33.2,{Zustandsbewertung})
                                (1,{Dichtigkeitsprüfung})
                                (1,{Bremsen betätigen})
                                (33.2,{Bremsung prüfen})
                                (2,{Bremse freigeben})
                                (33.2,{Freigabe prüfen})
%                                (183.1,{Sum})
                                };
                                %\addlegendentry{Wagon $<$ 4.0}
                                \addplot coordinates {                          
                                (1,{Zugvorbereitung})
                                (10,{Füllen Bremsleitung})
                                (1,{Zustandsbewertung})
                                (1,{Dichtigkeitsprüfung})
                                (1,{Bremsen betätigen})
                                (1,{Bremsung prüfen})
                                (2,{Bremse freigeben})
                                (1,{Freigabe prüfen})
%                                (18,Sum)
                                };
                                %\addlegendentry{Wagon 4.0}
                                \end{axis}
                              \end{tikzpicture}
        		\end{center}
    \caption{Zeitvergleich Bremsprobe und automatisierte Bremsprobe nach \cite{Stephenson}}
    \label{fig:Zeitvergleich}
\end{figure} 
Abbildung \ref{fig:Zeitvergleich} zeigt einen Zeitvergleich zwischen konventioneller Bremsprobe (blau) und automatisierter Bremsprobe (rot). Hier sieht man vor allem, dass  bei Zugvorbereitung, Füllen der Bremsleitung, Zustandsbewertung der Wagen und Überprüfung ob die Bremsen angelegt und wieder gelöst haben viel Zeit eingespart werden kann. Dies ist in den vorigen Kapiteln beschrieben.\par
Anhand des Ist- und Soll-Zustandes sind fünf Ausbaustufen für den Güterwagen 4.0 geplant, die diese Zeiteinsparung bringen sollen. Hier eine Kurzbeschreibung der geplanten Ausbaustufen:\par
\textbf{Ausbaustufe 1: Stromversorgung, Telematik und Datenvernetzung}\par
\begin{figure}[hbp] 
    \includegraphics[width=\textwidth]{Bilder/Ausbaustufen_1.PNG}
    \caption{Klasse 1 mit Stromversorgung, Telematik und Datenvernetzung - angelehnt an \cite{Ausbaustufen} }
    \label{fig:Klasse1}
\end{figure} 
In der ersten Ausbaustufe, siehe Abbildung \ref{fig:Klasse1}, ist geplant Bordelektronik und eine entsprechende Spannungsversorgung anzubringen. Diese kann als Batterie mit Speisung durch einen Radsatzgenerator, Solarpanels oder ähnlichem realisiert werden oder auch als Pufferbatterie mit Speisung durch AK. Dazu kommen verschiedene Antennen und Kurzstreckenfunk zur Kommunikation mit anderen Wagen. Auch Sensoren zur Erfassung verschiedener Telematikfunktionen sind geplant.\par
In diesem Stadium ist der Wagen an sich noch nicht 'schlauer' als ein nicht ausgerüsteter Wagen, aber er kann sich mitteilen. Mitteilungen könne sein: Standort, Belandung, Laufleistung, (ungewöhnliche) Vibrationen (beispielsweise durch Falschstellen), Heißläuferdedektion, letzte Wartungsintervalle, Zustand der Bremse und vieles mehr.\par
\textbf{Ausbaustufe 2: Ausbaustufe 1 + Automatisierung der Bremsbedienung}\par
\begin{figure}[htbp] 
    \includegraphics[width=\textwidth]{Bilder/Ausbaustufen_2.PNG}
    \caption{Klasse 2 bestehend aus Klasse 1 und der Automatisierung der Bremsbedienung - angelehnt an \cite{Ausbaustufen}}
    \label{fig:Klasse2}
\end{figure} 
In der zweiten Ausbaustufe ist eine zusätzliche Aktorik für Endabsperrhähne und Handbremse geplant. Dadurch kann ein Teil der Bremsbedienung sow weit automatisiert werden, dass ein Einstellen der Bremsart anhand von anderen Wagen im Wagenzug, Gewicht und Bremsfähigkeit möglich ist. Außerdem ist die automatische Parkbremse realisiert.\par
\textbf{Ausbaustufe 3: Ausbaustufe 2 + ep-''light''-Bremsen}\par
\begin{figure}[htbp] 
    \includegraphics[width=\textwidth]{Bilder/Ausbaustufen_3.PNG}
    \caption{Klasse 3 bestehend aus Klasse 2 und der eingeführten ep-''light'-Bremse - angelehnt an \cite{Ausbaustufen}}
    \label{fig:Klasse3}
\end{figure} 
In der dritten Ausbaustufe kommt zusätzlich zur Bremsbedienung auch die Ep-''light''-Bremse hinzu. Diese sorgt für eine für kürzere Bremswege und/oder höhere Geschwindigkeiten.\par
\textbf{Ausbaustufe 4: Ausbaustufe 3 + automatisierter Zugschluss}\par
\begin{figure}[htbp] 
    \includegraphics[width=\textwidth]{Bilder/Ausbaustufen_4.PNG}
    \caption{Klasse 4 bestehend aus Klasse 3 und der Automatisierung des Zugschlusses - angelehnt an \cite{Ausbaustufen}}
    \label{fig:Klasse4}
\end{figure} 
In der vierten Ausbaustufe ist ein automatisierter Zugschluss geplant, neben der dann noch lästigen Aufgabe den kompletten Wagenzug langzulaufen um am letzen Wagen ein Zugschluss-Signal zu stecken soll mit dieser Funktion auch die Zugintigrität gewährleistet werden.\par
\textbf{Ausbaustufe 5: Ausbaustufe 4 + Rangierantrieb}\par
\begin{figure}[htbp] 
    \includegraphics[width=\textwidth]{Bilder/Ausbaustufen_5.PNG}
    \caption{Klasse 5 bestehend aus Klasse 4 und einem Rangierantrieb - angelehnt an \cite{Ausbaustufen}}
    \label{fig:Klasse5}
\end{figure} 
In der fünften Ausbaustufe kommt der Rangierantrieb hinzu. Damit dieser ohne Probleme funktioniert brauch er er zusätzlich eine weitere Batterie und Umrichter. Zur Speisung der zweiten Batterie wird auch ein zweiter Radsatzgenerator eingebaut.\par
In diesem Stadium kann von einem automatisierten Güterwagen gesprochen werden. Er kann selbstständig bei der Briefkastenbedienung assistieren und auf dem Werksgelände ohne Rangierlok verfahren.\par
Die ersten drei Ausbaustufen sollen in diesem Projekt stattfinden. Ausbaustufe 4 und 5 in Folgeprojekten. Eine Zulassung soll über mindestens gleichbleibende Sicherheit bei kompletter Abschaltung des Systems stattfinden. Diese in Schritten stattfindende Automatisierung führt zu einem hohen Mehrwert des Güterwagens, nicht nur, im Einzelwagenverkehr.
 %\newpage
\section{Anforderungen}
\textbf{Einfügen: Wie lang sind Lösezeiten, Verwendung Handbremse, Umlegen Endabsperrhähne}\par
\textbf{Sicherheitsrelevanz, Bedienen, Beobachten, Anbringungspunkte von Bauteilen -- gefedert? -- Bahntauglichkeit}\par
\subsection{Allgemeine und nicht funktionale Anforderungen}
\begin{feat}[Anf. X]
Umgebungsbedingungen nach \acrshort{DIN} \acrshort{EN} 50155
\end{feat}
\begin{feat}[Anf. X]
mindestens gleichbleibende Sicherheit des Systems
\end{feat}
\begin{feat}[Anf. X]
Rüttel- und Schüttelfestigkeit nach \acrshort{DIN} \acrshort{EN} 61373
\end{feat}
\begin{feat}
EMV-Verträglichkeit nach ?
\end{feat}
\begin{feat}
Brandschutznachweise nach \acrshort{DIN} \acrshort{EN} 45545 (gegen TSI WAG prüfen)
\end{feat}
Verstellkomponenten möglichst nah beieinander um möglichst wenig Kabel zu ziehen\\
Sensoren immer überwachen! Mustererkennung - Rohdaten auswerten - was ist wichtig?
\subsection{Bremse}
\begin{feat}[REQ. 9]
elektrisches Lösen der Handbremse möglich
\end{feat}
\begin{feat}[REQ. 10]
elektrisches Lösen der Bremse möglich
\end{feat}
\begin{feat}[REQ. 11]
Automatische Einstellung der Bremsstellung
\end{feat}
\begin{feat}[REQ. 12]
Automatische oder ferngesteuerte Betätigung der Endabsperrhähne
\end{feat}
\begin{feat}[REQ. 12]
Bremsstellung G und P werden unterstützt
\end{feat}
\begin{feat}[REQ. 13]
automatische Bremsprobe für Bedienfahrt
\end{feat}
\begin{rem}
XXX
\end{rem}
\begin{feat}
automatische Bremsprobe für Rangierfahrt
\end{feat}
\begin{feat}
automatische Bremsprobe für Sperrfahrt
\end{feat}
\begin{feat}
automatische Bremsprobe für Zugfahrt
\end{feat}
\begin{feat}
automatische Bremsberechnung anhand des Wagenzuges und der Lok möglich
\end{feat}

\subsection{Kupplung}
\begin{feat}[REQ. 14]
elektrische Vorbereitung von Kuppelstellen möglich
\end{feat}
\begin{feat}
elektrische Vorbereitung von Trennstellen möglich
\end{feat}
\begin{rem}
XXX
\end{rem}

\subsection{Rechnerbasierte Anforderungen}
\begin{feat}
digitales Bilden einer Wagengruppe möglich
\end{feat}
\begin{feat}
Betriebsparameter für Bedienfahrt
\end{feat}
\begin{feat}
Betriebsparameter für Rangierfahrt
\end{feat}
\begin{feat}
Betriebsparameter für Sperrfahrt
\end{feat}
\begin{feat}
Betriebsparameter für Zugfahrt	
\end{feat}
\begin{feat}
automatische Wagengruppenbildung möglich
\end{feat}
\begin{feat}
Überprüfung der Wagenreihung möglich
\end{feat}
\begin{feat}
Überprüfung des technischen Zustands aus vorheriger Fahrt möglich
\end{feat}
\begin{feat}
automatische Übermittlung von Transportunterlagen möglich
\end{feat}

\subsection{Ideenspeicher}
\begin{feat}
automatische Zugschlussanzeige
\end{feat}
\begin{feat}
Rangierantrieb
\end{feat}%\newpage
%\input{Kapitel/Schnittstellen} %\newpage
%\input{Kapitel/Inbetriebnahme} %\newpage
%\input{Kapitel/Qualitaetsanforderungen} %\newpage
%\input{Kapitel/Projektdurchfuehrung} %\newpage
%\bibliography{Literatur}
%Anhang
\appendix
\section{Abkürzungen}
Tabelle mit allen Abkürzungen hier anlegen\\
\begin{tabular}[c]{r|p{12cm}l}
	\hline
	AK		& 	Automatikkupplung	\\
	
	DIN		&	Deutsches Institut für Normung	\\
	
	EN		&	Europäische Normung	\\
	EVU		&	Eisenbahnverkehrsunternehmen\\
	
	FH AC 	& 	FH Aachen			\\
	
	GW40	&	Güterwagen 4.0		\\
	
	HL		&	Hauptluftleitung	\\
	
	ISO		&	International Organisation for Standardisation	(dt.: Internationale Organisation für Normung)\\
	
	RIL		&	Richtlinie			\\
	
	TwB		&	technische Wagenbehandlung	\\
	
	
	
	\hline
\end{tabular}
\section{Glossar}
Tabelle mit allen Begriffserklärungen hier anlegen.\\
\begin{tabular}[c]{r|p{12cm}l}
	\hline
	Ablaufberg			&	Der Ablaufberg ist ein in der Regel künstlich angelegter Hügel, über den ein Gleis verläuft.		\\
	
	Bedienfahrt			&		\\
	Bezetteln           &       \\
	Bremsprobe			&		\\
	
	EOW                 &       \\
	
	Gleisanschluss		&		\\
	
	Hemmschuh			&		\\
	
	Knotenbahnhof       &       \\
	
	Rangierbahnhof		&		\\
	Rangierfahrt        &       \\
	Rangiermittel		&		\\
	Rangiervorgang      &       \\
	
	Satellitenbahnhof	&		\\
	Sägefahrten         &       \\
	Schiebewandwagen	&		\\
	Sperrfahrt          &       \\
	
	Zugfahrt			&		\\
	Zugschluss			&		\\
	
	\hline
\end{tabular}
\end{document}